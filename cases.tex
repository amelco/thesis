%  C A S E S . S T Y             ver 1.0
%% Modified by Luciano R Silveira for plain TeX use, 18/04/2015
%
%  Copyright (C) 1993 by Donald Arseneau
%  These macros may be freely transmitted, reproduced, or modified 
%  provided that this notice is left intact.  Sub-equation numbering
%  is based on subeqn.sty by Stephen Gildea.
%
%  LaTeX environment "numcases" to produce multi-case equations with
%  a separate equation number for each case.  Also, "subnumcases"
%  gives each case numbered with the equation number plus a letter 
%  [8a, 8b, etc.]   The syntax is
%
%  \begin{numcases}{left_side}
%  case_1 & explanation \\
%  case_2 & explanation \\
%  ...
%  case_n & explanation
%  \end{numcases}
%
%  Each case is a math formula, and each explanation is a piece of
%  lr mode text (which may contain math mode).  The explanations are
%  optional.  Equation numbers are inserted automatically, just as for
%  the eqnarray environment.  In particular, the \nonumber command 
%  suppresses an equation number, and the \label command allows 
%  reference to a particular equation.  In a subnumcases environment, 
%  a \label in the left_side of the equation gives the overall equation 
%  number, without any letter.
%
% - - - - -   
%  A simple example is:
%    \begin{numcases}{|x|=}
%    x, & for $x \geq 0$\\
%    -x, & for $x < 0$
%    \end{numcases}
%
%                    / x   for x > 0              (1)
%             |x| = <            -
%                    \ -x  for x < 0              (2)
% 
% - - - - -   
%  Another:
%    \begin{subnumcases}{f(x)=\label{f(x)}}
%    1/3 & if $0 < x < 1$;\\
%    2/3 & if $3 < x < 4$;\label{.6666}\\
%    0   & elsewhere.
%    \end{subnumcases}
% 
%                   /  1/3   if 0 < x < 1;      (18.23a)
%                   |
%           f(x) = <   2/3   if 3 < x < 4;      (18.23b)   
%                   |
%                   \  0     elsewhere.         (18.23c)
%
%  and \ref{f(x)} will then give (18.23) while \ref{.6666} gives (18.23b)
%
%  - - -  begin definitions - - - 

% Sub number a equation
\newcount\dsubnum
\dsubnum=0
\def\thedsubnum{(\the\dnum.\the\dsubnum)}
\def\eqsubmark{%
\ifnum\dsubnum=0 \global\advance\dnum by1 \fi
\global\advance\dsubnum by1
\ifinner\else\eqno \fi
\wlabel\thedsubnum \thedsubnum
}%

\catcode`@=11
\newskip\@centering
\@centering = 0pt plus 1000pt
\newcount\@eqcnt
\def\@eqncr{& \eqsubmark\cr}
\def\@@eqncr{& \eqsubmark\cr}
\def\m@th{\mathsurround\z@}
\def\numcases#1{$$%
 \setbox\z@\hbox{ % local
  $\displaystyle {#1}\m@th\mskip\medmuskip$}%
%\numc@setsub
\dimen@ii\displaywidth
\setbox\tw@\vbox\bgroup
% \stepcounter{equation}\let\@currentlabel\theequation
% \global\@eqnswtrue\m@th
 \tabskip\@centering\let\\\@eqncr
 \halign to\dimen@ii
 \bgroup
   \kern\wd\z@ \kern13\p@ \global\@eqcnt\@ne$\displaystyle\tabskip\z@{##}$\hfil 
   &\global\@eqcnt\tw@ \quad ##\unskip \hfil\tabskip\@centering
   &\llap{##}\tabskip\z@\cr}

\def\endnumcases{\@@eqncr
 \egroup % end \halign, which does not contain first column or brace
% \global\advance\c@equation\m@ne 
% \unskip\unpenalty\unskip\unpenalty
 \setbox\z@\lastbox \nointerlineskip \copy\z@ % grab last line, then put it back
 \g@tboxedwidth\z@ % box \z@ destroyed, width of columns -> \dimen@i
\egroup% end \vbox (\box\tw@)
\hbox to\displaywidth{% assemble the whole equation
 \hskip\@centering \hbox to\dimen@i{$\displaystyle \box\z@ % parameter #1
  \dimen@\ht\tw@ \advance\dimen@\dp\tw@ % get size of brace
  \left\{\vcenter to\dimen@{\vfil}\right.\n@space % make brace
  $\hfil}\hskip\@centering % finished first part (filled whole line)
 \kern-\displaywidth $\vcenter{\box\tw@}$% overlay the alignment
}% end the \hbox
%\numc@resetsub
\global\dsubnum=0
$$}

% Get width of all boxes on a line, ignoring the glue between them.
% Return total width in \dimen@i (global).
\def\g@tboxedwidth#1{\setbox#1\hbox{\unhbox#1\global\dimen@i\z@ 
  \G@tBoxedWidth}}
\def\G@tBoxedWidth{\unskip\unskip\unskip \setbox\z@\lastbox 
 \ifvoid\z@\else \global\advance\dimen@i\wd\z@ \expandafter\G@tBoxedWidth \fi}
\catcode`@=12

% Send problem reports to asnd@Reg.TRIUMF.CA
%
% test integrity:
% brackets:  round, square, curly, angle:   () [] {} <>
% backslash, slash, vertical, at, dollar, and: \ / | @ $ &
% hat, grave, acute (apostrophe), quote, tilde, under:   ^ ` ' " ~ _

