\cleardoublepage
\chap CONCLUSION

\begitems
* A solute uptake model was incorporated in the microscopic model of \ref[liersolute].
* Additionally, it has the ability to partition active and passive contributions to the solute uptake, which can be uses in...
* The solute uptake can be predicted using linear and nonlinear approaches.
* It is better to use the nonlinear model since it is closer to the original MM equation, making its behaviour more close to reality (or our assumptions).
* The disadvantage in using NLU is that it is very sensitive to space and time steps, which can lead to stabilization problems.
* LU can be used in the case of...
* The model seems to overestimate the uptake in wet soil conditions.
* The model results seems to corroborate with the expected variations of soil, plant and atmospheric parameters.
* However, more tests has to be done, preferably with experimental data.
* The model is more sensitive to the parameters XXX, XXX... 
\enditems


%The proposed model simulates the solute flux and root uptake considering a soil concentration dependent uptake. There was no significant difference between linear and nonlinear solutions for the simulated scenarios. The results of uptake for the proposed model showed that the limiting potential is reached at a higher pressure head, increasing the period of potential transpiration. It also showed a second limiting condition that happens at the time when $C<C_{lim}$ caused by a insufficient supply of solute at the same rate of plant demand. The proposed model is also able to do a partition between active and passive uptake which will be important to simulate the plant stress due to ionic or osmotic components, according to the solute concentration inside the plant.
%Comparison between the numerical and the analytical solution proposed by Cushman is in process.

