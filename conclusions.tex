\cleardoublepage
\chap CONCLUSION

{\tred The importance of active and passive uptake partitioning are huge.
It gives light to the modelling of metabolic partitioning of energy spent by the plant, that can be used in a proper quantification of osmotic and toxic stresses.
The stressors are direct related to crop yield which can be also improve models crop predictions, and water and nutrients management.
Obviously, the model assumptions could be more complex and, as is known, the assumptions are extremelly important to the model predictions.
Poor assumptions leads to poor predictions and a complex system like this in study (SPA) are often far from reality.
Luckily, several computational/physical/mathematical models are being developed, each one with its own characteristics and limitations, in order to best approximate nature and its phenomena.
}



%The proposed model simulates the solute flux and root uptake considering a soil concentration dependent uptake. There was no significant difference between linear and nonlinear solutions for the simulated scenarios. The results of uptake for the proposed model showed that the limiting potential is reached at a higher pressure head, increasing the period of potential transpiration. It also showed a second limiting condition that happens at the time when $C<C_{lim}$ caused by a insufficient supply of solute at the same rate of plant demand. The proposed model is also able to do a partition between active and passive uptake which will be important to simulate the plant stress due to ionic or osmotic components, according to the solute concentration inside the plant.
%Comparison between the numerical and the analytical solution proposed by Cushman is in process.

