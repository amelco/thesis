\section{Conclusions}
\subsection{Conclusions}

\begin{frame}
\frametitle{Conclusions}
\begin{itemize}
\item Linear and non-linear uptake solutions show good agreement with an analytical solution which also considers a concentration dependent uptake as the boundary condition at the root surface.
They are significantly different only when comparing the concentration as a function of time for times where $C_0 < C_{lim}$ (NUP).

\item A second reduction in the $T_r$ may occur by a reduction of the solute uptake rate resulting in a reduction of water flux due to the decreasing value of pressure head needed to maintain the limiting value of $H=H_{lim}$.
It shows that the limiting value $C_{lim}$ can be an important parameter to determine changes in the combined water and osmotic stress in low concentration situations, suggesting it requires more investigation.

\item Soil hydraulic properties, root length density, initial concentration and potential transpiration are factors that change the time that the concentration at the root surface starts to decrease and the time that the active uptake is maximum.
\end{itemize}

\end{frame}

\begin{frame}
\frametitle{Conclusions}
\begin{itemize}
\item Quantities that require a careful parameterization are: $\theta_r$, $\theta_s$, $\alpha$, $I_m$ and $K_m$, affecting strongly the solute concentration at the root surface at completion of simulation, $\theta_s$ affecting the time at wich limiting values of solute concentration are reached, and $n$ which strongly affects all selected predictions, mainly $h_\pi$. 

\item The model showed to be able to quantify the active and passive contributions to the solute uptake, which can be used to distinguish osmotic and ionic stressors in further works.

\item The proposed model uses an implicit scheme for the numerical solution of the convection-dispersion, including variable space steps and diffusion coefficients.
A more detailed investigation of stability issues for this kind of model would benefit its applicability and is suggested as a future work.
\end{itemize}
\end{frame}
