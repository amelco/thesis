\section{Conclusões}
\subsection{Conclusões}

\begin{frame}
\frametitle{Conclusões}
\begin{itemize}
  \item A solução linear é preferível por ser de mais rápida execução, porém apenas nas predições $Ac(t)$ e $C(r)$.
%\item Linear and non-linear uptake solutions show good agreement with an analytical solution which also considers a concentration dependent uptake as the boundary condition at the root surface.
%They are significantly different only when comparing the concentration as a function of time for times where $C_0 < C_{lim}$ (NUP).
  \item Uma segunda redução na $T_r$ mostrou-se possível devido à redução do fluxo de água para se manter o valor de $H_{lim}$.
        $C_{lim}$ está diretamente associado com o fluxo de água e é importante na determinação do estresse combinado (hídrico e osmótico/iônico) em baixas concentrações de soluto no solo. 

%\item A second reduction in the $T_r$ may occur by a reduction of the solute uptake rate resulting in a reduction of water flux due to the decreasing value of pressure head needed to maintain the limiting value of $H=H_{lim}$.
%It shows that the limiting value $C_{lim}$ can be an important parameter to determine changes in the combined water and osmotic stress in low concentration situations, suggesting it requires more investigation.
  \item As propriedades hidráulicas do solo, densidade radicular, concentração inicial de soluto e transpiração potencial são fatores que afetam o tempo em que a concentração à superfície da raíz começa a diminuir e o tempo em que a extração ativa é máxima.

%\item Soil hydraulic properties, root length density, initial concentration and potential transpiration are factors that change the time that the concentration at the root surface starts to decrease and the time that the active uptake is maximum.
\end{itemize}

\end{frame}

\begin{frame}
\frametitle{Conclusões}
\begin{itemize}
  \item Os parâmetros mais sensíveis do modelo são: \\
    $\theta_r$, $\theta_s$, $\alpha$, $I_m$ e $K_m$ $\rightarrow$ afetam fortemente a concentração de soluto no solo\\
    $\theta_s$ $\rightarrow$ afeta o tempo em que os valores limitantes de concentração são alcançados\\
    $n$ $\rightarrow$ afeta todas as saídas, mas principalmente $h_\pi$

  \item O modelo quantifica as contribuições ativa e passiva da extração de soluto do solo, que podem ser utilizadas para discernir o estresse osmótico do iônico em trabalhos futuros.
        
%\item Quantities that require a careful parameterization are: $\theta_r$, $\theta_s$, $\alpha$, $I_m$ and $K_m$, affecting strongly the solute concentration at the root surface at completion of simulation, $\theta_s$ affecting the time at wich limiting values of solute concentration are reached, and $n$ which strongly affects all selected predictions, mainly $h_\pi$. 

%\item The model showed to be able to quantify the active and passive contributions to the solute uptake, which can be used to distinguish osmotic and ionic stressors in further works.

%\item The proposed model uses an implicit scheme for the numerical solution of the convection-dispersion, including variable space steps and diffusion coefficients.
%A more detailed investigation of stability issues for this kind of model would benefit its applicability and is suggested as a future work.
\end{itemize}
\end{frame}

\begin{frame}\frametitle{Trabalhos futuros}
  \begin{itemize}
    \item Modificações no método de discretização/resolução das equações de balanço de massa afim de se obter soluções estáveis e mais rápidas.
    \item Elaborar experimentos controlados com plantas sob situações conhecidas de estresse osmótico e iônico e comparar os resultados com as previsões do modelo. 
    \item Investigar os mecanismos que atuam e caracterizam a definição da concentração limitante ($C_{lim}$) aqui teorizada.
    \item Considerar a concentração do íon dentro da planta.
  \end{itemize}

\end{frame}
