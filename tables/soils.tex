% Table with soil types

% macro pmb to bold math symbols
\long\def\pmb#1{\setbox0=\hbox{#1}\copy0\kern-\wd0%
\kern0.02em\raise0.02ex\copy0\kern-\wd0\kern0.02em\box0}

{\tenpoint
\midinsert\medskip \label[soils]
\caption/t {Soil hydraulical parameters used in simulations}
\def\tabiteml{\hbox to 4pt{}}  % left material before each \table item
\def\tabitemr{\hbox to 4pt{}}
\ctable{lcccccccc}{
\crl\hfil 
{\bf Staring}	& {\bf Textural}& {\bf Reference}	& \pmb{$\theta_r$}	& \pmb{$\theta_s$}	& \pmb{$\alpha$}& {\bf l}	& {\bf n}	& \pmb{$K_s$}		\cr 
{\bf soil ID}	& {\bf class}	& {\bf in this paper}	& $\rm{m^3\,m^{-3}}$	& $\rm{m^3\,m^{-3}}$	& $\rm{m^{-1}}$	& --		& --		& $\rm{m\,d^{-1}}$	\crl \tskip4pt
B3		& Loamy sand	& Sand		 	& 0.02			& 0.46			& 1.44		& -0.215	& 1.534 	& 0.1542		\cr
B11		& Heavy clay	& Clay		 	& 0.01			& 0.59			& 1.95		& -5.901	& 1.109 	& 0.0453		\cr
B13		& Sandy loam	& Loam		 	& 0.01			& 0.42			& 0.84		& -1.497	& 1.441 	& 0.1298		\crl
\multispan2 {\tenpoint Source: \citeonline[wosten]} \hfill  & & & & & & \cr
}
\endinsert
}
