\section*{Abstract}

% 250 words MAXIMUM!!
A modification in an existing water uptake and solute transport numerical model was implemented in order to allow the model to simulate solute uptake by the roots.
The convection-dispersion equation (CDE) was solved numerically, using a complete implicit scheme, considering a transient state for water and solute fluxes and a soil solute concentration dependent boundary for the uptake at the root surface, based on the Michaelis-Menten (MM) equation.
Additionally, a linear approximation was developed for the MM equation such that the CDE has a linear and a non-linear solution.
A radial geometry was assumed, considering a single root with its surface acting as the uptake boundary and the outer boundary being the half distance between neighboring roots, a function of root density.
The proposed solute transport model includes active and passive solute uptake and predicts solute concentration as a function of time and distance from the root surface.
It also estimates the relative transpiration of the plant, on its turn directly affecting water and solute uptake and related to water and osmotic stress status of the plant.
Performed simulations show that the linear and non-linear solutions result in significantly different solute uptake predictions when the soil solute concentration is below a limiting value ($C_{lim}$).
This reduction in uptake at low concentrations may result in a further reduction in the relative transpiration.
The contributions of active and passive uptake vary with parameters related to the ion species, the plant, the atmosphere and the soil hydraulic properties. 
The model showed a good agreement with an analytical model that uses a linear concentration dependent equation as boundary condition for uptake at the root surface.
The advantage of the numerical model is it allows simulation of transient solute and water uptake and, therefore, can be used in a wider range of situations.
Simulation with different scenarios and comparison with experimental results are needed to verify model performance and possibly suggest improvements.

