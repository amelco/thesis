\app Finding \clim{}, \c2 and linearizing MM equation

According to the assumptions made over the MM equation (described in Chapter~\ref[methodology]), the diffusive and convective contributions for the uptake are driven by active and passive processes.
%We assume that the diffusive and convective parts of the original equation is similar to the active and passive uptakes of the MM equation, respectively, at root suurface. 
It is shown in Figure \ref[fig_MM] the two limiting concentrations (\clim{} and \c2), and the non-linear (Figure \ref[fig_MM]a) and linear (Figure \ref[fig_MM]b) uptake functions.

%proposed partitioning of active and passive uptakes, for a given water flux density. Figure 0a is linear, which is a simplification of MM equation to facilitate its use in the numerical solution. Figura 0a is the MM equation itself.

\medskip
\label[fig_MM]
\picw=10cm \cinspic I_and_Linear.pdf
\caption/f {Uptake (influx) rate as a function of concentration in soil water for [a] non-linear case and [b] linear case. \clim{} is the limiting concentration in which the uptake is limited by the solute flux and \c2{} is the concentration where the uptake is governed by convective flow only}
\medskip

%EXPLAIN ACTIVE AND PASSIVE
 
%Figure 0. Uptake rate (influx rate) as a function of concentration in soil water for [a] non-linear case and [b] linear case

In the linearized equation, the slope $\beta$ of the total uptake line (continuous line), for concentration values smaller than $C_{lim}$, can be found by the relation $I_m/C_{lim}$, since the line starts at the origin of the Cartesian coordinates system:
\label[eq_relation_beta]
$$
\beta = {I_m \over C_{lim}}. \eqmark
$$
%
According to our assumptions, at values smaller than $C_{lim}$ the solute uptake is concentration dependent and the uptake is smaller than the plant demand $I_m$.
Additionally, for values greater than $C_2$, the solute uptake occurs due to the mass transport by water flow only, {\it i.e.} active uptake is zero and the overall uptake is passive.

Therefore, \clim{} can be calculated by setting the value of solute flux density at the root surface ($q_{s0}$, Equation \ref[eq_case1]) to \im{} and $C_0$ to $C_{lim}$:
\label[eq_clim1]
$$
%I_m = {I_m C_0 \over K_m + C_0} + q_0 C_0 \,\,. \eqmark
I_m = {I_m C_{lim} \over K_m + C_{lim}} + q_0 C_{lim} \,\,. \eqmark
$$
%
Solving Equation \ref[eq_clim1] for \clim, we find it as the positive value of:
\label[eq_clim2]
$$
C_{lim} = -{K_m \pm \left( {K_m}^2 + 4{I_m K_m / q_0} \right)^{1/2} \over 2 } \eqmark
$$
%
Substitution of Equation \ref[eq_clim2] into \ref[eq_relation_beta], the slope of total uptake $\beta$ can be defined as the positive value of:
\label[eq_beta]
$$
\beta = { I_m \over C_{lim} } = -{2 I_m \over K_m \pm \left( {K_m}^2 + 4{I_m K_m / q_0} \right)^{1/2}}. \eqmark
$$

At concentration values greater than $C_2$, the solute uptake is driven only by mass flow of water and the active uptake is zero. 
Thus, the first term of the right-hand-side of Equation \ref[eq_case1] is zero and $C_2$ can be found by setting $q_{s0}$ to \im{} and $C_0$ to $C_2$:
\label[eq_c2]
$$
I_m= q_0 C_2 \Rightarrow C_2 = {I_m \over q_0} \eqmark
$$
%

%WRITE IN RADIAL COORDINATES??
%$$
%C_2 = {-I_m \over 2 \pi L r_0 q_0} \eqmark
%$$

The partitioning between active ($\alpha C_0$) and passive ($q_0 C_0$) uptake is calculated by difference, as the values of total uptake ($\beta C_0$) and passive uptake is always known:
$$
\eqalignno{
&q_{s0} = (\hbox{active slope} + \hbox{passive slope})\,C_0 = \beta\,C_0; \cr
&\hbox{passive slope} = q_0; \cr
&\hbox{active slope} = \beta - q_0 = \alpha; \cr
\label[eq_linear]
&q_{s0} = (\alpha + q_0)\,C_0. & \eqmark
}
$$

The equation \ref[eq_linear] is, therefore, the linearization of equation \ref[eq_case1], and it is used in the piecewise equation \ref[eq_MM_mod] for values of concentration smaller than $C_{lim}$ and greater than $C_2$ (with $\alpha=0$).


