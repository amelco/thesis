\cleardoublepage
\label[theory]
\chap THEORETICAL FRAMEWORK (base, foundations)
% Describe:
% * water flow (briefly)
% * solute transport equations that will be used in the comparisons
% * Michaelis-Menten equation

This chapter focuses on the theoretical aspects used in the methodology.
It briefly describes the Richards equation that is applied in water flow models and details the convection-dispersion equation for solute transport.
Both flow models are detailed in terms of a microscopic approach.
Also, a short explanation about the Michaelis-Menten kinetics for nutrient uptake is given.
%, which is fundamental to understand the work done in this thesis, since it is treated as a boundary condition for the convection-dispersion equation. 
For those who are familiarized with these theories, the reading starting from Chapter \ref[methodology] is sufficient and this chapter may be skipped, since it does not provide any particular information that is different from what is already known from literature.
Equations that were presented in this section, and are used further in this thesis, are properly referenced.

\label[sec_micro]
\sec Microscopic approach (modelling)

As mentioned in the last section, the models for water and solute uptake can be divided in macroscopic and microscopic. 
Since the model developed in this thesis is microscopic, we will present the basic theory behind it.

%Macroscopic models deal with flows in one, two or three spacial dimensions and the use a sink-source therm in the flow equations to deal with solute/water inputs and outputs in the domain. The domain is usually the root-zone but can also consider bigger regions. (maybe this is inside the literature review). According to the objective of this thesis, the developed model was inspired in a microscopic one and, thus, is also microscopic. So, detail in the microscopic approach for modelling water and solute uptakes are described.

Microscopic models consider a cylindrical root of radius $r_0$ \uhead with a extraction zone being represented by a concentric cylinder of radius $r_m$ (m) that bounds the the half-distance between roots. The height of both cylinders is $z$ \uhead and represents the rooted soil depth. 
The basic assumptions of this type of model is that the root density does not change with depth and there is no difference in extraction power along the root surface. Water and solute flows are axis-symmetric.

It is common to find in the literature values for the root length density $R$ (m m$^3$) and $r_0$. In this case, the equations used to find values for $r_m$ and root length $L$ \uhead are:
\label[eq_rm]
$$
r_m = {1 \over \sqrt{\pi R}} \eqmark
$$
\label[eq_L]
$$
L = {A_p z \over \pi {r_m}^2} \eqmark
$$
%
where $A_p$ (m$^2$) is the soil surface area occupied by the plant. For the case that there is no available data from literature, one can obtain the value of $L$ from relatively simple measurements of root and soil characteristics as soil mass ($m_s$, Kg) and density ($d_s$, Kg m$^{-3}$), and root average radius ($\overline{r_0}$, m)
\label[eq_L2]
$$
L = {d_s A_p z - m_s \over d_s \pi \overline{r_0}^2} \eqmark
$$
%
then estimate $r_m$ and $R$ by
\label[eq_rm2]
$$
r_m = \sqrt{A_p z \over \pi L} \eqmark
$$
\label[eq_R]
$$
R = {1 \over \pi {r_m}^2} \eqmark
$$

For a complete derivation of the Equations \ref[eq_rm] to \ref[eq_R], see Appendix \ref[ap_micro].


\sec Water flow equation

The equation for water flow for a homogeneous and isotropic soil, in saturated and non-saturated conditions, is given by the Richards equation. As microscopic models consider a radial geometry, it can be written as:
\label[eq_Richards]
$$
r{\partial \theta \over \partial t} = -{\partial q \over \partial r} \eqmark
$$
%
where $r$ \uhead is the distance from the root axial center, $\theta$ \uwc is the water content in soil and $t$ \utime is the time. The water flux density $q$ \uwatflux is given by the Darcy-Buckingham equation:
\label[eq_watflux]
$$
q = -K(\theta) {dH \over dr} \eqmark
$$
%
where $K$ \uK is the soil hydraulic conductivity and $H$ is the total hydraulic potential. This equation describes the laminar water flux for (non)saturated soil in any direction of space. 
%The role of $K$ is detailed in Section \ref[soil_prop].

Equation \ref[eq_Richards] is a second-order partial differential equation and, therefore, needs initial and boundary conditions to result in a particular solution. 
The most common initial condition is of constant water content or pressure head along the radial distance, although a function of water content (or pressure head) over distance can also be used. 
In analytical solutions, steady-state condition in relation to water flow is often used to solve the equation. The high non-linearity of the hydraulic functions (see Section \ref[soil_prop]) makes a transient solution rather complex and, sometimes, impractical.
In numerical solutions, on the other hand, it is easier to define a transient water flow condition, which gives a more realistic treatment to the equation. 
Boundaries for both steady-state and transient solutions can be either of prescribed pressure head or water flux, as shown in Equations \ref[wat_bc1] and \ref[wat_bc2] respectively:
\label[wat_bc1]
$$
h(r_i,t) = f(t) \eqmark
$$
\label[wat_bc2]
$$
\left. K(\theta) {\partial h \over \partial r} \right|_{r=r_i} = g(t) \eqmark
$$
%
where $f(t)$ and $g(t)$ can be either a constant or a time variable function of pressure head and water flux, respectively; and $r_i$ is the distance from axial center to the specific boundary. In the case of microscopic (and usually radial) models, $r_i$ can assume values of distance from axial center to root surface (inner boundary) and to the end of the domain (outer boundary). Macroscopic models use the Cartesian system and, for one-dimension, represents the domain along the $z$ axis (depth) instead of $r$. Thus, $z_i$ values would be of the distance from the soil surface (top boundary) to a certain depth, {\it e.g.} root depth (bottom boundary).

In most of microscopic (single root) models, the boundary conditions are of flux type, according to Equation \ref[wat_bc2]. At root surface, it is equal to the transpiration rate, and at the outer boundary, it is often of zero flux --- meaning inter-root competition for water. Therefore, the only water exit is at root surface through the transpiration stream.
In macroscopic models, as they deal with the entire root zone, both 
%boundaries are located at the soil surface and at a given depth (usually, root depth). 
%Both 
boundary types are equally found, depending on the simulation scenario to consider (soil surface evaporation, irrigation or rain, presence of water table, drainage, water root uptake, etc.). 
A sink-source term is then added to Equation \ref[eq_Richards] to deal with such water inputs and outputs (as seen in the previous chapter).
%can be equal to the evaporation rate and of free drainage in the bottom (root depth). The water extraction by plant roots is equal to the plant transpiration encapsulated in a sink term in Richards equation.

The hydraulic potential $H$ is the sum of pressure ($h$) and elevation ($h_g$) heads in models that do not consider solute flow. 
In order to deal with solutes, the osmotic head ($h_\pi$) must be added to $H$ and it will serve as a `link' to the solute transport equation, detailed in Section \ref[sol_transp]. 
Moreover, $h_g$ can be neglected when this component is of minor relevance (as of in microscopic models).

%When no solute is dealt in the solution of Richards equation, $H$ turns out to be pressure head $h$ and, occasionally, the sum of pressure head an gravitational component $h_g$, neglecting the osmotic head $h_\pi$. The link between water movement and solute transport in soil appears when $h_\pi$ is considered in the Richards equation. Researchers try to separate the mechanisms that deals with both transports although they occur sinergically in nature. The flow of water interferes in the solute flow and vice-versa and a link between them should exist. In Section \ref[sol_transp] we deal with transport of solute in more details.

\label[soil_prop]
\sec Soil hydraulic functions

The soil hydraulic properties $K$, $\theta$ and $h$ are interdependent and, as mentioned in the previous section, highly nonlinear. Equations \ref[eq_theta] and \ref[eq_K] show their interdependence and their noticeable nonlinearity. Among all models to describe the water retention curve mentioned in Chapter \ref[literature], the most used is the one of van Genuchten below:

%As mentioned in the previous section, $K$, $\theta$ and $h$ are inter-related soil properties. Their interdependence allows to infer one property in relation to the other two. A common practice is to measure $\theta$ for several $h$ values and make use of a model to know the so called characteristic soil curve or water retention curve. As the function of $K$ is dependent of $\theta$ or $h$, this relation is automatically obtained. Among several $K$--$\theta$--$h$ relationship models, the van Genuchten relations is one of the most used:

\label[eq_theta]
$$
\theta(h) = \theta_r + {\theta_s-\theta_r \over [1+|\alpha h|^n]^{1-(1/n)}} \eqmark
$$

\label[eq_K]
$$
K(\theta) = K_{s} \Theta^\lambda [1-(1-\Theta^{n/(n-1)})^{(1-(1/n)})]^2  \eqmark
$$
%
where $\Theta$ (--) is the effective saturation defined by ${(\theta - \theta_r) \over (\theta_s-\theta_r)}$; $\theta_s$ \uwc and $\theta_r$ \uwc are the saturated and residual water contents, respectively; and $\alpha$ (m$^{-1}$), $\lambda$ (--) and $n$ (--) are empirical parameters.

Luckily, the relationship between the hydraulic functions allows to determine a property as a function of another. Thus, by measuring two of them, it is possible to fit the parameters of Equations \ref[eq_theta] and \ref[eq_K]. A common practice is to measure $\theta$ at prescribed values of $h$ and adjust the Equation \ref[eq_theta] parameters with the data, consequently obtaining the $K$ function.

%BRIEF EXPLANATION OF THE PARAMETERS (one paragraph)

%BRIEF EXPLANATION OF K(O)

%BRIEF EXPLANATION ABOUT THE NONLINEARITY OF THE FUNCTIONS
%Experiments have shown that the relationship between $\theta$ and $h$ is extremely nonlinear. Besides that, there is hysteresis in the process that makes the function different for drying or wetting patterns.

%TALK BRIEFLY ABOUT REDUCTION FUNCTIONS

\label[sol_transp]
\sec Solute transport

The solute transport in soil occurs by diffusion and convection. The equation that involves those mechanisms is the convection-diffusion equation. In radial coordinates, it can be written as:

\label[eq_sol]
$$
r{\partial (\theta C) \over \partial t} = -{\partial q_s \over \partial r} \eqmark
$$
%
where $C$ \uconc is the concentration of solute in soil solution. The solute flux density $q_s$ \usolflux is given by:

\label[eq_solflux]
$$
q_s = -D(\theta) {dC \over dr} + qC \eqmark
$$
%
where $D$ \udisp is the effective diffusion-dispersion coefficient. This equation describes the solute flux in the soil solution.
The convective flux is accounted by the second term on the right-hand side of the equation \ref[eq_solflux] and the diffusive-dispersive flux by the first term. 

%convection
Transport by convection occurs due to movement of diluted solutes carried by mass flow of water proportional to the solute concentration in soil solution. 
It is noticeable that the convective contribution to solute flux reduces as water flux $q$ becomes small, therefore the convective flux is highly dependent on water content gradient and hydraulic conductivity. 
Convection also affects the solute diffusivity due to a water velocity gradient originated in the micropore space of the soil. 
As solutes are carried with different velocities, a concentration gradient is originated, allowing a diffusive movement within this pore space.
This microscale process is called mechanical dispersion and can be expressed by macroscopic variables as follows:

\label[eq_Ds]
$$
D_s = \tau {q \over \theta} \eqmark
$$
%
where $D_s$ \udisp is the mechanical dispersion coefficient, $\tau$ \uhead is the dispersivity and the relation $q/\theta$ \uwatflux accounts for the average pore water velocity.

%dispersion-diffusion
Transport by diffusion is the movement of solutes caused by a concentration gradient in soil solution. It is proportional to the effective diffusion coefficient $D$ which is given by:

$$
D = D_m + D_s \eqmark
$$
%
where $D_m$ \udisp is the molecular diffusion coefficient that, as $D_s$, is also a microscale process resulted from the averaged random motion of the molecules in the soil solution in response to the concentration gradient. It can be expressed by macroscopic variables as follows:

\label[eq_Dm]
$$
D_m=\xi D^0 = {\theta^{10 \over 3} \over \theta^2_s} D^0 \eqmark
$$
%
where $\xi$ (--) is the tortuosity factor and $D^0$  \udisp is the diffusion coefficient in water. Values of $D^0$ for ions or molecules can be easily measured and estimated by analytical equations for free water. Soil water content is often below its saturation point, thus, diffusion is often different from that of free water. 
Equations \ref [eq_Ds] to \ref[eq_Dm] are one of the several existing models to estimate the effective diffusion-dispersion coefficient. 
Chapter \ref[literature] has a list of the available models.


%boundary conditions
The boundary conditions types are the same as of water, except that they are in respect to concentration instead of pressure head. 
In microscopic models, common boundary conditions are of zero, constant or variable solute flux at root surface. 
Choosing one over another depends on how the solute uptake by the plant will be considered in the model (details in Section \ref[sol_uptake]).
The outer boundary depends on the distance in which it is considered. A zero solute flux condition works well to simulate inter-root competition for solutes in a distance ranging from XXX to xxx m while constant concentration --- equal to the initial concentration --- can be used in a situation that the boundary is `far enough' from root surface, {\it i.e.} there is no competition between neighbour roots.
In macroscopic models, a sink-source term is added to Equation \ref[eq_sol] to consider different inputs and outputs of solute in the system ({\it e.g.} plant uptake, mineralization, degradation, etc.), and the top and bottom boundaries are treated according to the scenario (similar to what is done with the water flow equation).

%accumulation or depletion of solute at root surface dependent on chosen boundaries
%Focusing in microscopic modelling, 
%Different boundary conditions result in distinct solutions for the partial differential equation and, consequently, lead to different results. 
%For a situation of zero solute flux at root surface, the solute would accumulate there due to the transport by convection. 
%A consequent diffusion of solutes away from the roots would occur due to a difference of gradient until it achieves an equilibrium. 
%On the other hand, considering a constant or a variable solute flux at root surface, an accumulation of solute would occur uniquely if the root uptake is less than the solute flux arriving at root surface. 
%The opposite would cause a decrease in concentration at root surface until a zero concentration value. 
%What differs from constant to variable solute flux boundary condition is the rate of solute uptake.

%Explanation of the 3 models

%Simplifications can be made in the solutions for equation \ref[eq_sol], considering $D$ not dependent on $\theta$, for example.


\label[sol_uptake]
\sec Solute uptake by plant roots
%Describes zero, constant and variable solute uptake. Detail the three models.
%In literature review, do a simple description and refer to here for more details.

%The root-soil interface is the boundary that deals with solute uptake by the root.
The transition from fluxes of water and solutes in the soil to physiological mechanisms of uptake by the plant is located at the root-soil interface.
This boundary is of the outermost importance and a correct mathematical treatment here is crucial.
Different conditions in this boundary result in distinct solutions of the convection-dispersion equation and, consequently, lead to different results. 
%talk about concentration dependent boundary vs zero and constant uptak
For a situation of zero solute flux at root surface, the solute would accumulate 
%there 
in that region
due to the transport by convection. 
A consequent diffusion of solutes away from the root would occur due to the formed concentration gradient until an equilibrium between those fluxes is achieved. 
On the other hand, considering a constant or a variable solute flux at root surface, an accumulation of solute would occur uniquely if the root uptake is less than the solute flux arriving at root surface. 
The opposite situation would cause a decrease in concentration at root surface until a zero (or limiting) concentration value. 
%What differs from constant to variable solute flux boundary condition is the rate of solute uptake.

%In this section, only boundary conditions at root surface for microscopic models will be treated. 
%More specifically, the boundaries of the models used in this thesis. They are of zero and constant solute flow. 
%The boundary of concentration dependent solute flow is also described and is one of the objectives of this thesis. 
%Its application is detailed in Chapter \ref[methodology].

Writing the convection-diffusion equation in its full form, by substituting Equation \ref[eq_solflux] in Equation \ref[eq_sol] as

$$
r{\partial (\theta C) \over \partial t} = {\partial \over \partial r} \left( D(\theta) {\partial C \over \partial r} - qC\right) \eqmark
$$
%
%being $r_0$ \uhead the distance from root center to the root surface and knowing that solute flux is represented by Equation \ref[eq_solflux], the boundary condition at root surface can be written in the form:
and with a flux type boundary condition at root surface of 
$$
\left. -D(\theta) {\partial C \over \partial r} \right|_{r=r_0} + qC = F\;{\rm ,} \eqmark
$$
%
all conditions of solute uptake rate can be represented by setting to $F$ a specific uptake function. A zero solute uptake condition would be $F = 0$, a constant uptake condition would be $F = k$ and a concentration dependent uptake function would be $F = \alpha(C) k$, where $k$ \usolflux is a constant that represents the uptake rate and $\alpha(C)$ (--) a shape function for the uptake. 

The rate of solute uptake by plant roots can be referred to be similar to that between enzyme and its substrate, described by the Michaelis-Menten (MM) equation \cite[michaelis]. Thus the uptake shape function $\alpha(C)$ can follow the concentration dependent MM kinetics, and $k$ being equal to $I_m$ leads to the following:

$$
\alpha(C) = {C \over K_m+C} \Rightarrow F = {C \over K_m+C}\, I_m \eqmark
$$
%
where $I_m$ is the maximum uptake rate, $C$ is the solute concentration in soil solution and $K_m$ the Michaelis-Menten constant. 
$I_m$ is found experimentally and $K_m$ is adjusted as the concentration at which $I_m$ assumes half of its value, being interpreted as the affinity of the plant for the solute. 

The uptake rate following the MM kinetics is an asymptote which saturates with increasing external concentration (Figure \ref[orig_MM]). The equation shows that, for very low concentration values ($C \ll K_m$), the uptake rate is proportional to concentration whereas, for $C \gg K_m$, the solute uptake rate is maximal and independent of concentration.

\medskip
\label[orig_MM]
\picw=13cm \cinspic orig_MM.pdf
\caption/f {Solute uptake rate as a function of external concentration following Michaelis-Menten kinetics with its more common parameters}
\medskip

The $K_m$ coefficient represents the affinity of the plant for the solute. For $K_m$ values higher than 50 ${\rm \mu M}$, it is considered that the plant has a low affinity for the solute whereas for values less than 10 ${\rm \mu M}$, the plant has a high-affinity for it.


%The zero solute uptake condition is straightforward in the mathematics. 
%The condition of constant solute uptake assumes that the plant demand for solutes is constant. Considering the plant demand $I_m$ \usolflux, the uptake for an individual root, in cylindrical coordinates, would be as:
%
%$$
%F = {I_m \over 2 \pi L_{rv} L r_0}
%$$
%%
%where $L_{rv}$ m m$^{-3}$ is the root density and $L$ \uhead the root length.



