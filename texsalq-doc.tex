% The documentation of the usage of TeXSALQ -- the template for
% typessetting thesis by plain\TeX at ESALQ - USP
% ---------------------------------------------------------------------
% Luciano R Silveira  Jun.-Jul. 2014
% lrsilveira@gmail.com
%
% Based on ctustyle-doc.tex by Petr Olsak

\input macros/texsalq   % The template is included here.

\worktype [D/EN] % Type: M = master, D = Ph.D., O = other
                 % / the language: PT = Portuguese, EN = English

\title    {Mechanistic numerical modeling of solute uptake by plant roots}
\author     {Andre Herman Freire Bezerra}
\authorinfo {Bachelor in Agronomy}
\city       {Piracicaba}
\date       {2015}
\supervisor {Prof. Dr. \bf\uppercase{Quirijn de Jong van Lier}}  % One or more supervisors
\workname   {} % Used only if \worktype [O/*] (Other)

            % Title / Subtitle in minor language:
\titlePT      {Modelagem num�rica de extra��o de solutos pelas ra�zes}
\titarea    {Ci�ncias}
\concarea   {Engenharia de Sistemas Agr�colas}
\concareaEN {Agricultural Systems Engineering}
\pagetwo    {}  % The text printed on the page 2 at the bottom.
\dedication {      %  Optional. Use main language here
     To the past, \par\vskip 0.3cm
    to the present, and \par\vskip 0.3cm
    to the future \par\vskip 1.3cm
    %With love%, \bfit DEDICO\par
}
\thanks     {      %  Optional. Use main language here
%Agrade�o aos meus amigos e colegas de p�s-gradua��o...
%
%Ao Conselho Nacional de Densenvolvimento Cient�fico e Tecnol�gico (CNPq) pela bolsa de doutorado e � Coordena��o de Aperfei�oamento de Pessoal de N�vel Superior (CAPES) pela bolsa de est�gio {\it sandwich}.
%
%Ao meu orientador Quirijn de Jong van Lier por...
%
%I also would like to thank my coleagues at the Wageningen University for all the support that, even without knowing, they gave me and specially to my advisors Sjoerd van der Zee (official), Jos van Dam and Peter de Willigen to every contribution for this work. 
%Wihtout your support, this thesis would be still a project.
    }
\epigraph   {       %  Optional. Use main language here
   {\sl
Some catch phrase that everyone expect to read.
   }

}
\abstractEN {%
A modification in a water uptake and solute transport numerical model was implemented in order to take into account the solute uptake by the roots.
The convection-dispersion equation (CDE) was solved numerically, using a complete implicit scheme, considering a transient state for water and solute fluxes and a soil solute concentration dependent boundary for the uptake at the root surface, based on the Michaelis-Menten (MM) equation.
Additionally, a linear approximation was developed for the MM equation such that the CDE has a linear and a non-linear solution.
A radial geometry was assumed, considering a single root with its surface acting as the uptake boundary and the outer boundary being the half distance between neighbouring roots, a function of root density.
The proposed solute transport model includes active and passive solute uptake and predicts solute concentration as a function of time and distance from the root surface.
It also estimates the relative transpiration of the plant, on its turn directly affecting water and solute uptake and related to water and osmotic stress status of the plant.
{\localcolor\Blue Performed simulations show that the linear and non-linear solutions result in significantly different solute uptake predictions when the soil solute concentration is below a limiting value ($C_{lim}$). {\it Verify this conclusion. There is or there is not difference?}}
This reduction in uptake at low concentrations may result in a further reduction in the relative transpiration, {\localcolor\Red shortening the time at which the plant ceases the uptake. {\it QM: This is not clear. Change words}}
The contributions of active and passive uptake vary with parameters related to the ion species, the plant, the atmosphere and the soil hydraulic properties. 
%{\localcolor\Red As the model has the ability to partition active and passive uptake, it can be improved to differentiate osmotic and ionic stress by including a solute concentration inside the plant variable and compare it with a threshold value of plant concentration, which is regulated by plant age and species, for example.}
The model showed a good agreement with an analytical model that uses a linear concentration dependent equation as boundary condition for uptake at the root surface.
The advantage of the numerical model is it allows simulation of transient solute and water uptake and, therefore, can be used in a wider range of situations.
Simulation with different scenarios and comparison with experimental results is needed to verify model performance and possibly suggest improvements.
}
\abstractPT {%
 Aqui o resumo  
}

\keywordsEN {%
Solute transport; Michaelis-Menten; transient solute flux
}
\keywordsPT {%

}


%%%%% <--   % The place for your own macros is here.
\def\emph#1{{\it #1}}
\def\ttb{\tt\char`\\} %  sequence control printing of Tables
\def\asp#1{``#1''} % facilitates the use of quotation marks
\addprotect\ref

%\draft     % Uncomment this if the version of your document is working only.
%\linespacing=1.7  % uncomment this if you need more spaces between lines
                   % Warning: this works only when \draft is activated!
%\savetoner        % Turns off the lightBlue or lightGrey backround of
                   % verbatims, only for \draft version.
%\blackwhite       % Use this if you need really Black+White thesis.

\makefront  % Mandatory command. Makes title page, acknowledgment, contents etc.

% Files where the source of the document is prepared.
% Full name is: body.tex, appendicies.tex, the suffix can be omitted.

%\input epsf % Works onty if pdfoutput = 0
%\input body

\input introduction

\input literature

%\input literature_old

\input theory

\input methodology

\input results

\input conclusions

{\makebib{mybib}}

\input appendicies

\bye
