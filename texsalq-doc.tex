% The documentation of the usage of TeXSALQ -- the template for
% typessetting thesis by plain\TeX at ESALQ - USP
% ---------------------------------------------------------------------
% Luciano R Silveira  Jun.-Jul. 2014
% lrsilveira@gmail.com
%
% Based on ctustyle-doc.tex by Petr Olsak

\input macros/texsalq   % The template is included here.

\worktype [D/EN] % Type: M = master, D = Ph.D., O = other
                 % / the language: PT = Portuguese, EN = English

\title    {Mechanistic numerical modeling of solute uptake by plant roots}
\author     {Andre Herman Freire Bezerra}
\authorinfo {Engenheiro Agron�mo}
\city       {Piracicaba}
\date       {2014}
\supervisor {Prof. Dr. \bf\uppercase{Quirijn de Jong van Lier}}  % One or more supervisors
\workname   {} % Used only if \worktype [O/*] (Other)

            % Title / Subtitle in minor language:
\titlePT      {Modelagem num�rica de extra��o de solutos pelas ra�zes}
\titarea    {Ci�ncias}
\concarea   {Engenharia de Sistemas Agr�colas}
\concareaEN {Agricultural Systems Engineering}
\pagetwo    {}  % The text printed on the page 2 at the bottom.
\dedication {      %  Optional. Use main language here
     Ao passado, \par\vskip 0.3cm
    ao presente e \par\vskip 0.3cm
    ao futuro \par\vskip 1.3cm
    Com amor, \bfit DEDICO\par
}
\thanks     {      %  Optional. Use main language here

    }
\epigraph   {       %  Optional. Use main language here
   {\sl
   }

}
\abstractEN {%
   
}
\abstractPT {%
   
}

\keywordsEN {%

}
\keywordsPT {%

}


%%%%% <--   % The place for your own macros is here.
\def\emph#1{{\it #1}}
\def\ttb{\tt\char`\\} %  sequence control printing of Tables
\def\asp#1{``#1''} % facilitates the use of quotation marks

%\draft     % Uncomment this if the version of your document is working only.
%\linespacing=1.7  % uncomment this if you need more spaces between lines
                   % Warning: this works only when \draft is activated!
%\savetoner        % Turns off the lightBlue or lightGrey backround of
                   % verbatims, only for \draft version.
%\blackwhite       % Use this if you need really Black+White thesis.

\makefront  % Mandatory command. Makes title page, acknowledgment, contents etc.

% Files where the source of the document is prepared.
% Full name is: body.tex, appendicies.tex, the suffix can be omitted.

%\input epsf % Works onty if pdfoutput = 0
\input body

{\makebib{mybib}}

%\input appendicies

\bye
