% The documentation of the usage of TeXSALQ -- the template for
% typessetting thesis by plain\TeX at ESALQ - USP
% ---------------------------------------------------------------------
% Luciano R Silveira  Jun.-Jul. 2014
% lrsilveira@gmail.com
%
% Based on ctustyle-doc.tex by Petr Olsak

\input macros/texsalq   % The template is included here.

\worktype [D/EN] % Type: M = master, D = Ph.D., O = other
                 % / the language: PT = Portuguese, EN = English

\title    {Mechanistic numerical modeling of solute uptake by plant roots}
\author     {Andre Herman Freire Bezerra}
\authorinfo {Bachelor in Agronomy}
\city       {Piracicaba}
\date       {2015}
\supervisor {Prof. Dr. \bf\uppercase{Quirijn de Jong van Lier}}  % One or more supervisors
\workname   {} % Used only if \worktype [O/*] (Other)

            % Title / Subtitle in minor language:
\titlePT      {Modelagem num�rica de extra��o de solutos pelas ra�zes}
\titarea    {Ci�ncias}
\concarea   {Engenharia de Sistemas Agr�colas}
\concareaEN {Agricultural Systems Engineering}
\pagetwo    {}  % The text printed on the page 2 at the bottom.
\dedication {      %  Optional. Use main language here
     Ao passado, \par\vskip 0.3cm
    ao presente e \par\vskip 0.3cm
    ao futuro \par\vskip 1.3cm
    Com amor, \bfit DEDICO\par
}
\thanks     {      %  Optional. Use main language here

    }
\epigraph   {       %  Optional. Use main language here
   {\sl
   }

}
\abstractEN {%
A modification in a water uptake and solute transport numerical model was implemented in order to take into account the solute uptake by the roots.
The convection-dispersion equation (CDE) was solved numerically, using a complete implicit scheme, considering a transient state for water and solute fluxes and a soil solute concentration dependent boundary for the uptake at the root surface, based on the Michaelis-Menten (MM) equation.
Additionally, a linear approximation was developed to the MM equation such that the CDE has a linear and a nonlinear solution.
It was assumed a radial geometry, considering a single root with its surface acting as the uptake boundary and the outer boundary as being the half distance between neighbours roots which depends on root density.
The proposed solute transport model includes active and passive solute uptake and can predict the solute concentration in soil at any time and distance from the root surface.
It also estimates the relative transpiration of the plant, that directly affects water and solute uptake and is related to water and osmotic stress status of the plant.
The simulations has shown that the linear and nonlinear solutions have significant differences in uptake when the concentration in the soil is below a limiting value ($C_{lim}$).
Consequently, this reduction in uptake at low concentrations can produce a second reduction in the relative transpiration, shortening the time at which the plant ceases the uptake. 
{\localcolor\Red Active and passive uptake contributions change with all tested input parameters and the simulations has shown that, even with the same ion type, a whole set of contributions might happen.
As the model has the ability to partition active and passive uptake, it can be improved to differentiate osmotic and ionic stress by including a solute concentration inside the plant variable and compare it with a threshold value of plant concentration, which is regulated by plant age and species, for example.}
The model showed a good agreement with a analytical model that uses a linear concentration dependent equation as the boundary condition for uptake at the root surface, with the advantage to deal with transient solute and water uptake and, therefore, can be used in a wider range of situations.
}
\abstractPT {%
 Aqui o resumo  
}

\keywordsEN {%
Solute transport; Michaelis-Menten; transient solute flux
}
\keywordsPT {%

}


%%%%% <--   % The place for your own macros is here.
\def\emph#1{{\it #1}}
\def\ttb{\tt\char`\\} %  sequence control printing of Tables
\def\asp#1{``#1''} % facilitates the use of quotation marks
\addprotect\ref

%\draft     % Uncomment this if the version of your document is working only.
%\linespacing=1.7  % uncomment this if you need more spaces between lines
                   % Warning: this works only when \draft is activated!
%\savetoner        % Turns off the lightBlue or lightGrey backround of
                   % verbatims, only for \draft version.
%\blackwhite       % Use this if you need really Black+White thesis.

\makefront  % Mandatory command. Makes title page, acknowledgment, contents etc.

% Files where the source of the document is prepared.
% Full name is: body.tex, appendicies.tex, the suffix can be omitted.

%\input epsf % Works onty if pdfoutput = 0
%\input body

\input introduction

\input literature

%\input literature_old

\input theory

\input methodology

\input results

\input conclusions

{\makebib{mybib}}

\input appendicies

\bye
