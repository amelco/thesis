\cleardoublepage
\label[literature]
\chap LITERATURE REVIEW

\sec Modeling solute transport and uptake by roots

%The processes that involves solute uptake by the plants have been subject of study of many researchers since the beginning of the last century. 
%Solute flow in the soil is directly related to the water flow and uptake by the plants, and they are indissociable.
%Therefore, studying water flow is a obligated requirement to understand solute flow, which makes pioneer works such as of GARDNER crucial to the further development of the theories. He described the processes of fluxes from soil to plant as a set of resistances that should be overpassed to the flow to happen

Solute transport in soil and solute uptake by the plants have extensively been studied in the last half century.
As a result, several simulation models have been developed to predict water and solute flows within the vadose zone. 
They are usually classified as microscopic or macroscopic, according to the considered scale \cite[feddes]. 
Microscopic models \cite[barber74,cushman,willigen1,roose] consider a single cylindrical root, of uniform radius and absorption properties \cite[gardner], that extracts water and solute by an axisymmetric flow according to the defined boundary condition at the root surface.
The water flow is described using the Richards equation and the solute flow using the convection-dispersion equation (CDE), both formulated in radial coordinates.
The flow towards the root is driven by hydraulic head and concentration gradients between the root and the surrounding soil, proportional to the hydraulic conductivity and solute diffusivity. 
One of the advantages of the microscopic approach is that it implicitly simulates water uptake compensation, as the computed local water potential gradients control the uptake for the whole root system \cite[simunek]. 
Despite the more realistic simulation of soil-root interactions, microscopic models require a large computational effort for the simulations and, therefore, are limited to applications of relatively small scale of a single plant.
Chapter \ref[theory] shows, in more details, the theory of microscopic models.

Macroscopic models \cite[simunekHYDRUS,somma,vandam] consider the whole root zone as a single uniform extraction component whereby the potential transpiration is distributed over the root zone and it is proportional to the rooting density. 
This approach neglects the effects of root geometry and flow pathways around individual roots, and considers water and solute uptakes as a sink term added to their respective mass balance equations. 
On the other hand, macroscopic models are able to simulate processes at the plot or field scale. They are sometimes incorporated in complete hydrological models that also deal with drainage, crop growth and productivity.

Another classification of models, relates to the experimental approach of the phenomenon, distinguishing between empirical and mechanistic models. 
Empirical models \cite[chanter,ross,yerokun] are based on the response of a set of experiments and attempt to describe the observed phenomenon without hypothesizing about underlying mechanisms. 
They are usually of stochastic nature and have the advantage of being relatively simple, allowing to predict the results of the processes with a good certainty. 
The main disadvantage of this kind of model is that the model parameters do not have a clear physical, physiological or biological meaning, limiting interpretation and substantial model improvement. 
Furthermore, the application of these models is limited to boundary conditions corresponding to the experiments/observations that they were developed from, and they will probably lead to incorrect or inaccurate predictions in scenarios that were not used for their development.
Mechanistic models \cite[barber74,cushman,willigen1,roose,simunekHYDRUS,somma,vandam] seek to explain the physical mechanisms that drive the phenomenon, requiring a better understanding and mathematical description of the underlying processes. 
Although the parameters may be difficult to measure, they have a physical meaning -- they can be explained by the processes and their relationships, and they can be measured independently -- and, therefore, can be adapted to be used in a broad variety of scenarios.
Good reviews of available root solute uptake models can be found in \citeonline[rengel], \citeonline[feddes] and \citeonline[silberbush2013].

\sec Solute mobility in soils

Solute mobility in soil is described by two processes: 
1) convective transport by water mass flow through the transpiration stream, and 
2) movement driven by diffusion that depends on the solute concentration gradient in the soil caused by depletion of solutes due to root uptake \cite[barber62]. 
The first to recognize that solutes move towards the root due to the uptake process was \citeonline[bray].
Since then, the efforts to describe the solute movement in soils were directed to develop analytical models and, when more computational means became available, numerical models.
%
The earlier analytical solutions for the CDE were formulated considering a steady-state condition to the water flow and a solute uptake governed by the solute concentration in the soil solution \cite[barber74,cushman,nye] or determined by a constant plant demand \cite[willigen81]. 
The concentration limiting (or supply driven) approach may overestimate the uptake in scenarios where solute supply to the root is not limiting \cite[barraclough] whilst the constant plant demand (or demand driven) formulation may overestimate the uptake when the soil is very dry or at low solute concentration at the root surface, when the diffusive flow prevails.
%(cite papers about MM experiments -- the difference between the constant uptake and MM equation at low concentrations)
Later, \citeonline[roosephd] formulated analytical solutions for different conditions of uptake, considering a so called `pseudo-steady-state' condition for water flow and the interference of root hairs and mycorrhizae in the uptake process, offering also an analytical solution to upscaling from the single root to a three-dimensional root architecture uptake model.

%Numerical solutions

%Talk about Solute extraction models (MM and others)
Solute uptake can be described by the Michaelis-Menten (MM) equation \cite[barber,barber81,schroder,simunek]
according to which uptake increases with increasing solute concentration asymptotically approaching maximum uptake (further details in Section \ref[sol_uptake]). 
%Change this. pasted from rengel!!!!!!!!!
The necessary parameters are the MM constant ($K_m$), the maximum solute uptake ($I_m$) and the minimum solution concentration at which uptake ceases ($C_{min}$) \cite[barber]. 
%Kinetic parameters $I_m$ and $K_m$ vary with plant species, genotype, plant age, soil temperature, and nutritional status of the plant \cite[jungk,teo]. 
%Therefore, they would have to be determined for each experimental situation. 
%my annotations about MM here
\citeonline[epstein52] were the first to use the MM equation to represent the uptake of a solute and its concentration in external medium (for potassium and sodium solutions in barley roots) and it has been frequently used since then. 
The MM equation is supposed to describe well the solute uptake for both anions \cite[epstein72,siddiqi,wang] and cations \cite[broadley,kelly,kochian,lux,sadana] in the low concentration range and, adding a linear component ($k$) to the equation, it can properly estimates the uptake rate also for higher concentrations \cite[borstlap,broadley,epstein72,kochian,vallejo,wang].
Many authors agree that for low concentration in external medium, the uptake is driven by an active plant mechanism, as it occurs contrary the solute gradient between root and soil (Epstein's mechanism~I). 
For the high concentration range, solutes are freely transported from soil to roots by diffusion and occasional convection. This passive transport is known as Epstein's mechanism~II \cite[kochian,siddiqi]. 
Moreover, experiments have shown that a minimum concentration ($C_{min}$) bellow which the uptake ceases is commonly found \cite[dudal,machado,mouat].
Details on Epstein's mechanisms and its physiological mechanisms, as well as on active and passive uptake, are found in \citeonline[epstein60] and \citeonline[fried].

The values of MM parameters are strongly dependent on the experimental methods used and vary with plant species, plant age, plant nutritional status, soil temperature and pH \cite[barber,shi]. 
Therefore, they have to be determined for each particular experimental scenario.
Some types of experiments to determine the kinetic parameters $I_m$, $K_m$ and $C_{min}$ include hydroponically-grown plants \cite[barber] and the use of radioisotopes to estimate them directly from soil \cite[nye77]. 
The latter is more realistic since there is a large difference between a stirred nutrient solution and the complex and dynamic soil medium.
Measuring $C_{min}$ is particularly difficult \cite[lambers,seeling] because it occurs at very low concentration levels that may be hard to be accurately measured. 
\citeonline[seeling] show that $C_{min}$ can be neglected for the cases of high $K_m$ values.

%{\localcolor\Red
%IMPROVE
%
%%Define active and passive uptake and its importance in the quantification of plant ionic and osmotic stress.
%The plant can extract solute from the soil solution by two mechanisms: active and passive uptake. 
%The former is done against the root-soil concentration gradient and, therefore, demands metabolic energy while the later is energy-free since it happens through the process of convection (solute carried by mass flow of water through the transpiration stream) \cite[epstein72]. 
%The MM equation can be considered to deal with both uptake mechanisms and it has been used in many models with
%%Solute uptake kinetics using the MM equation have been used in numerical \cite[bechtold,nye,silberbush,simunek] and analytical \cite[barber,cushman,roose] models.
%%Other authors also makes a separation of active and passive as being the diffusive and convective part, respectively, of the convection-dispersion equation, such as
%a separation of active and passive uptake as being respectively the diffusive and convective part of the convection-dispersion equation 
%%Silberbush, Ben-Asher and Ephrath (2005) 
%\cite[silberbush,schroder,ungs,raij,simunek].
%Authors do not seem to agree on the use of MM equation.  
%Depending on the assumptions made in the model, some authors consider that it is valid for active uptake only while others consider that it deals with both active and passive uptake \cite[shi].
%The partitioning of the uptake in active and passive mechanisms can lead to discussions on how the plant respond to osmotic and ionic stress, therefore, being possible to separate these two stressors. 
%A starting point would be to consider that the plant would not take up solute by the active mechanism once it has already achieved a limiting concetration. 
%As passive uptake is not regulated by the plant, it would continue the uptake, making the concentration go higher than the limiting value causing the ionic stress (or toxicity) within the plant.
%%But the plant concentration inside the plant has to be considered in the model.
%}
%
%Talk about other uptake functions
There are other alternatives to describe the solute uptake rate function apart from the MM equation. 
\citeonline[dalton] proposed a physical-mathematical model that includes active uptake, mass flow and diffusion, expressing solute uptake as being proportional to root water uptake. 
\citeonline[bouldin] described the uptake from low and high concentration solutions by two linear equations, thus simplifying the process of solute uptake at the root surface to two diffusion components \cite[rengel].
\citeonline[tinker] determined the uptake rate as a linear component followed by a constant rate phase where the threshold value is dependent on solute concentration inside the plant.
Nevertheless, the MM equation is the most frequently used equation in root solute uptake models for being physically grounded and for its good agreement with experimental data.


%Talk about transpiration reduction functions (water, solute and water+solute). Talk about the quirijn's no uptake model and how it affects the combined water and solute reduction function without assume an additive or multiplicative interaction
Combined water and solute uptake models compute averages or local pressure and osmotic heads, allowing estimating plant water and osmotic stress. 
%\citeonline[feddes78] developed an empirical water stress model relating the actual transpiration to pressure head originating a transpiration reduction function. 
\citeonline[feddes78] developed an empirical water stress model relating the actual transpiration to pressure head, originating a piecewise transpiration reduction function in which the potential transpiration decreases linearly according to threshold values of pressure head that account for excess or shortage of soil water.
%It is a piecewise function in which the potential transpiration is linearly reduced according to threshold values of pressure head that accounts for excess and shortage of water in soil.
As the presence of solutes in the soil solution reduces the total hydraulic head -- affecting root water uptake -- the compound effect in transpiration reduction by water and salt stress are commonly supposed to be either additive 
%\cite[vangenuchten87] 
or multiplicative.
%\cite[vangenuchten87] 
However, crops seem to respond to water and salt stress differently. 

\citeonline[shalhevet] found that transpiration of cotton and pepper plants reduces more strongly under low pressure head than under an equivalent low osmotic head. 
Similar results have already been found by \citeonline[sepaskhah] for wheat, and by \citeonline[parra] for beans.
Thus, empirical weighting factors have been introduced in the transpiration reduction function to account for the different response of a crop to osmotic and pressure heads.

Some models treat the combined water and salt stress neither in an additive or in a multiplicative way. \citeonline[homaee] developed a piecewise reduction function similar to the \citeonline[feddes78] and concluded that the function fitted his experimental data better than any additive or multiplicative model.
\citeonline[liersolute] proposed a microscopic analytical model to estimate relative transpiration as a function of soil and plant parameters. 
Their mechanistic approach reduces the need of empirical parameters and does not require any explicit assumption (additive or multiplicative) to derive the combined water and salt stress. 
Results show good agreement with experimental data, although some discrepancies exist, especially under wet conditions. 
Their model does not consider root solute uptake.
Considering root solute uptake would interfere in solute concentrations estimative, leading to more realistic simulations and to a better prediction of the crop response to the combined water and osmotic stress.
%One of their conclusions was that the relative transpiration reduces more gradually with combined water and osmotic stress than in a scenario with water stress only


%disadvantages of this


%reduced according to four threshold pressure heads values ($h_1$, $h_2$, $h_3$ and $h_4$). The actual transpiration reduces, linearly, between $h_3$ and $h_4$, and between $h_1$ and $h_2$ due to soil water shortage and soil oxygen shortage, respectively. For pressure heads values greater than $h_1$ and lower than $h_4$, the transpiration is zero, and between $h_2$ and $h_3$, the transpiration is potential.
%Figura feddes

%Talk about specific models: van Dam (water), quirijn (solute+water), cushman and willigen. All microscopic. Tell the limitations of them.

%POTENTIAL GOOD CITATIONS
%Other authors also makes a separation of active and passive as the diffusive and convective part of eq. \ref[eq_solute], such as Silberbush, Ben-Asher and Ephrath (2005) \cite[silberbush] 102.pdf on calcium uptake in a soilless culture, Schröder et al. (2012) \cite[schroder] 48.pdf, Ungs, Boersma and Akratanakul (1982) \cite[ungs] 11.pdf, Raij et al. (2013) 43.pdf \cite[raij] and Simunek and Hopmans (2009) \cite[simunek] 68.pdf just to cite some.
