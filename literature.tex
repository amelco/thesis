\cleardoublepage
\label[literature]
\chap LITERATURE REVIEW

%The processes that involves solute uptake by the plants have been subject of study of many researchers since the beginning of the last century. 
%Solute flow in the soil is directly related to the water flow and uptake by the plants, and they are indissociable.
%Therefore, studying water flow is a obligated requirement to understand solute flow, which makes pioneer works such as of GARDNER crucial to the further development of the theories. He described the processes of fluxes from soil to plant as a set of resistances that should be overpassed to the flow to happen

Solute transport in soil and uptake by the plants are subjects that have extensively been studied in the last half century and, as a result, several simulation models have been developed to predict water and solute flows within the vadose zone. 
They are usually classified as being microscopic or macroscopic models, according to the scale that the processes are considered \cite[feddes]. 
Microscopic models \cite[barber74,cushman,willigen1,roose] consider a single cylindrical root, of uniform radius and absorption properties \cite[gardner], that extracts water and solute by an axis-symmetric flow according to the defined boundary condition at root surface.
The water flow is described using the Richards equation and the solute flow using the convection-dispersion equation (CDE), both formulated in radial coordinates.
The flow towards the root is driven by water and concentration gradients between the root and the surrounding soil, proportional to the hydraulic conductivity and solute diffusivity. 
The main advantage of the microscopic approach is that it automatically allows water uptake compensation mechanism, as the computed local water potential gradients control the uptake for the whole root system \cite[simunek]. 
Despite the more realistic simulation of soil-root interactions, microscopic models require a large computational effort for the simulations and, therefore, are limited to applications of relatively small scale of a single plant.
Chapter \ref[theory] shows, in more details, the theory of microscopic models.

Macroscopic models \cite[simunekHYDRUS,somma,vandam] consider the whole root zone as a single extraction component whereby the potential transpiration is distributed over the root zone and it is proportional to the root density. 
This approach neglects the effects of root geometry and flow pathways around the roots, and considers water and solute uptakes as a sink term added to their respective mass balance equations. 
On the other hand, macroscopic models are able to simulate processes at plot or field scales. They are usually more complete hydrological models that deals also with drainage, crop growth and productivity.

Another classification, related to how the phenomena is addressed, is of empirical and mechanistic models. 
Empirical models \cite[chanter,ross,yerokun] are based on the response of a set of experiments and attempt to describe the observed phenomena without hypothesizing how they happen. 
They are usually stochastic and have the advantage of being relatively simple, allowing to predict the results of the processes with a good certainty. 
The main disadvantage of this approach is that the resulting parameters from the model development has no physical, physiological or biological meaning, limiting its improvement possibilities. The application of those models are limited to the dataset of experiments/observations that they were designed, which may cause incorrect or inaccurate predictions in scenarios that were not considered.
Mechanistic models \cite[barber74,cushman,willigen1,roose,simunekHYDRUS,somma,vandam] seek to explain how the phenomena have happened, which requires a better understanding and mathematical description of the underlying processes. 
Although the parameters may have a difficult measurement, they have physical meaning -- they can be explained by the processes and their relationships -- and, therefore, can be adapted to be used in a broad variety of scenarios.
Good reviews of solute uptake models can be found in \citeonline[rengel] and \citeonline[feddes].

Solute mobility in soil is described by two main processes: 
1) convective transport by water mass flow through the transpiration stream, and 
2) movement driven by diffusion that depends on the solute concentration gradient in the soil caused by depletion of solutes due to root uptake \cite[barber62]. 
The first to recognize that solutes move towards the root due to the uptake process was \citeonline[bray].
Since then, the efforts to describe the solute movement in soils were directed to formulate analytical and, with the improvement of computational power, numerical models.
%
Analytical solutions for the CDE were extensively formulated (citations) but their applications fall into a few set of scenarios. 
Those models considers steady-state condition for water \cite[cushman,willigen1] or solute \cite[liersolute] and simplified boundary conditions as constant uptake by the roots \cite[willigen1] or non-competitive roots absorption \cite[cushman]. 
Despite their limitations, analytical solutions have more flexibility to derive other functions as of total uptake and solute accumulation in the plant, time at which the concentration at root surface reaches a certain limiting value an so on.
Numerical models can easier deal with the non-linearities of the processes %talk about water functions (cited) and also solutes


%Talk about specific models: van Dam (water), quirijn (solute+water), cushman and willigen. All microscopic. Tell the limitations of them.

%Talk about transpiration reduction functions (water, solute and water+solute)



%Talk about Solute extraction models (MM and others)
%PASTED FROM RENGEL. MODIFY!!
In some cases nutrient uptake can be described by the Michaelis-Menten equation (Barber, 1984; Barber and Cushman, 1981; Claassen and Barber, 1976; Claassen et al., 1986) according to which uptake increases with increasing nutrient concentration in a curvilinear fashion asymptotically approaching maximum uptake. 
In such a case, necessary model parameters are Michaelis-Menten constant (Km), maximum nutrient uptake (Imax) and minimum solution concentration at which uptake ceases (Barber, 1984). 
In contrast, Bouldin (1989) described nutrient uptake by two linear equations corresponding to uptake from solutions of low and high nutrient concentrations, thus over-simplifying the uptake process at the root surface to two diffusion components only.
Kinetic parameters Imax and K~ vary with plant species, genotype, plant age, soil temperature, and nutritional status of the plant (Jungk and Claassen, 1989; Teo et al., 1992). 
Therefore, these parameters would have to be determined separately for each experimental situation. 

%\cite[rengel]
