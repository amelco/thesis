\cleardoublepage
\chap LITERATURE REVIEW

%In this work, a numerical solution for the equation of convection--dispersion is proposed, assuming a soil concentration dependent solute uptake as the boundary condition at root surface given by the Michaelis-Menten equation. The proposed model is compared with a no solute uptake \citeonline[liersolute] and a constant solute uptake \citeonline[willigen1] numerical models, as well as with an analytical model which uses steady-state condition for water content \citeonline[cushman]. The numerical models consider a transient water flow, based on the work of \citeonline[lierwater]. There are several numerical (\cite[nye], \cite[simunek]) and analytical (\cite[roose], \cite[cushman], \cite[willigen1]) models that describe water and solute uptake by the roots, each one with their own particularities, such as steady-state water flux solutions and different boundaries conditions at root surface. \cite[feddes] and \cite[raats] give reviews of soil water uptake modeling including effects of salinity.

%EXPLAIN THE ABILITY OF THE MODEL TO CALCULATE ACTIVE AND PASSIVE UPTAKES
%EXPLAIN THE REASON TO USE MM EQUATION

%Nyer and Marriot (1969): qs0=D dc/dr + qC ~ kC0/(1+kC0/Im) but they considered Im and K independent of q0. I just considered Im independent of q0, but k, alpha in my case, is depedent on concentration
%Other authors also makes a separation of active and passive as the diffusive and convective part of eq. (1), such as Silberbush, Ben-Asher and Ephrath (2005) [7] 102.pdf on calcium uptake in a soilless culture, Schrder et al. (2012) [6] 48.pdf, Ungs, Boersma and Akratanakul (1982) [9] 11.pdf, Raij et al. (2013) 43.pdf [4] and Simunek and Hopmans (2009) [8] 68.pdf just to cite some.

%The determination of water and salt stress in field crops is a major problem in agriculture. Stress of water is caused by a lower amount of water in the soil, which reduces the potential transpiration of the plant and leads to a yield decrease. Similarly, higher solute concentrations in the soil solution can lead to stress caused by salinity which, in turn, reduce the potential transpiration (osmotic effect) and interfere negatively in the plant metabolism (ionic stress or toxicity).
%Works of Maas anf Hoffman (1977) XXX determines threshold values in which the plant cannot recover from saline stress by a reducion of yield. Some other works (Katerji et al, 2000) test their values and try to make a better approximation of its empirical values by taking into account other characteristics like water stress as well 
%Modeling of water and solute extraction by plant roots can be used to predict the transpiration rates together with soil water and solute movement
%One of the greatest problems in agriculture is to predict the actual transpiration rates in order to calculate the amount of water to be applied in the field crops. The determination of this variable depends on a great number of properties such as climatic data and parameters of soil and plant.

%It needs to be clear that active and passive uptake only makes sense when it means movement of solute at the root surface. Then, active is the diffusive part and passive is the convective part. The whole transport in the soil profile is by convection and diffusion only.

%Analytical solutions are possible only for a few very simple systems, tend to be complicated and often are not understood by who needs to use them. Numerical solutions have the advantage that hey are not restricted to simple boundary conditions and a numerical technique converts a differential equation into a set of algebraic expressions that are easier to solve \cite[campbell].

\sec Transport of solute in the soil-plant system

%Plants need to uptake water and nutrients from soil to

The rate of solute uptake by plant roots can be referred to be similar to that between enzyme and its substrate, described by the Michaelis-Menten (MM) equation \cite[michaelis]. The most well-known form of the equation is the following:
\label[eq_MM]
$$
I = {I_m C \over K_m+C} \eqmark
$$
%
where $I$ the rate of solute uptake, $I_m$ is the maximum uptake rate, $C$ is the solute concentration in external medium and $K_m$ the Michelis-Menten constant. 
$I_m$ is found experimentally and $K_m$ is adjusted as the concentration at which $I_m$ assumes half of its value, and represents the affinity of the plant for the solute. The uptake rate following the MM kinetics is an asymptote which saturates with increasing external concentration (Figure \ref[orig_MM]).
\citeonline[johnson] revisited the original MM paper and, using modern computer techniques (inexistent by the time the equation was developed), they could get the same results as of the original paper---with no simplifying assumptions---showing how precise and careful was the measurements and equation development. They also showed that the constant found by Michaelis and Menten is $I_m/K_m$ rather than the widely referred $K_m$.

\medskip
\label[orig_MM]
\picw=13cm \cinspic orig_MM.pdf
\caption/f {Solute uptake rate as a function of external concentration following Michaelis-Menten kinetics with its more common parameters}
\medskip

\citeonline[epstein52] were the first to use the MM equation to represent the uptake of a solute and its concentration in external medium, for potassium and sodium solutions in barley roots. The equation shows that, for very low concentration values ($C \ll K_m$), the uptake rate is proportional to concentration. MM equation describes well the solute uptake for both anions \cite[epstein72,siddiqi,wang] and cations \cite[kochian,kelly,sadana,broadley,lux] in the low concentration range. Several authors agree that for low concentration in external medium, the uptake is given by an active mechanism since the uptake goes against the root-soil gradient and this is called Epstein's mechanism I \cite[kochian,siddiqi]. Also some experiments have shown that a minimum concentration value ($C_{min}$) in which the uptake ceases to occur is often found \cite[mouat,dudal,machado]. Measuring $C_{min}$ is experimentally difficult \cite[seeling,lambers] due to occur in very low concentrations, being hard to be detected. \citeonline[seeling] shows that $C_{min}$ can be negleted for the cases of high $K_m$ values.

For $C \gg K_m$ the solute uptake rate is maximal and independent of concentration. Nevertheless, experiments have shown that the uptake rate can present a linear increase at high concentrations and a linear component must be added on MM equation \cite[epstein72,kochian,borstlap,wang,vallejo,broadley]. The solute can go inside the plant freely by diffusion and convection, so it is a passive mechanism called Epstein's mechanism II \cite[kochian,siddiqi].

The MM equation modified by the addition of the other two terms of minimum concentration and the linear component, and used in many recent models is:
\label[eq_MM_plus]
$$
I = {I_m (C-C_{min}) \over K_m+C-C_{min}} + k (C-C_{min}) \eqmark
$$

The separation active and passive is due to many physiological proccesses within the plant and is explained in more details in \cite[marschner].

\medskip
\label[MM_plus]
\picw=13cm \cinspic MM_plus.pdf
\caption/f {Solute uptake rate as a function of external concentration. The saturable component follows Eq. \ref[eq_MM], the linear component is $k (C-C_{min})$ and the total uptake follows Eq. \ref[eq_MM_plus]}
\medskip




%\secc Numerical and analytical models

%\sec Solute transport

%\secc Michaelis-Menten equation

