\cleardoublepage
\label[literature]
\chap LITERATURE REVIEW

%The processes that involves solute uptake by the plants have been subject of study of many researchers since the beginning of the last century. 
%Solute flow in the soil is directly related to the water flow and uptake by the plants, and they are indissociable.
%Therefore, studying water flow is a obligated requirement to understand solute flow, which makes pioneer works such as of GARDNER crucial to the further development of the theories. He described the processes of fluxes from soil to plant as a set of resistances that should be overpassed to the flow to happen

Solute transport in soil and uptake by the plants are subjects that have extensively been studied in the last half century and, as a result, several simulation models have been developed to predict water and solute flows within the vadose zone. 
They are usually classified as being microscopic or macroscopic models, according to the scale that the processes are considered \cite[feddes]. 
Microscopic models \cite[barber74,cushman,willigen1,roose] consider a single cylindrical root, of uniform radius and absorption properties \cite[gardner], that extracts water and solute by an axis-symmetric flow according to the defined boundary condition at root surface.
The water flow is described using the Richards equation and the solute flow using the convection-dispersion equation (CDE), both formulated in radial coordinates.
The flow towards the root is driven by water and concentration gradients between the root and the surrounding soil, proportional to the hydraulic conductivity and solute diffusivity. 
The main advantage of the microscopic approach is that it automatically allows water uptake compensation mechanism, as the computed local water potential gradients control the uptake for the whole root system \cite[simunek]. 
Despite the more realistic simulation of soil-root interactions, microscopic models require a large computational effort for the simulations and, therefore, are limited to applications of relatively small scale of a single plant.
Chapter \ref[theory] shows, in more details, the theory of microscopic models.

Macroscopic models \cite[simunekHYDRUS,somma,vandam] consider the whole root zone as a single extraction component whereby the potential transpiration is distributed over the root zone and it is proportional to the root density. 
This approach neglects the effects of root geometry and flow pathways around the roots, and considers water and solute uptakes as a sink term added to their respective mass balance equations. 
On the other hand, macroscopic models are able to simulate processes at plot or field scales. They are usually more complete hydrological models that deals also with drainage, crop growth and productivity.

Another classification, related to how the phenomena is addressed, is of empirical and mechanistic models. 
Empirical models \cite[chanter,ross,yerokun] are based on the response of a set of experiments and attempt to describe the observed phenomena without hypothesizing how they happen. 
They are usually stochastic and have the advantage of being relatively simple, allowing to predict the results of the processes with a good certainty. 
The main disadvantage of this approach is that the resulting parameters from the model development has no physical, physiological or biological meaning, limiting its improvement possibilities. Besides, the application of those models are limited to the dataset of experiments/observations that they were designed for, which may cause incorrect or inaccurate predictions in scenarios that were not considered.
Mechanistic models \cite[barber74,cushman,willigen1,roose,simunekHYDRUS,somma,vandam] seek to explain how the phenomena have happened, which requires a better understanding and mathematical description of the underlying processes. 
Although the parameters may have a difficult measurement, they have physical meaning -- they can be explained by the processes and their relationships -- and, therefore, can be adapted to be used in a broad variety of scenarios.
Good reviews of solute uptake models can be found in \citeonline[rengel], \citeonline[feddes] and \citeonline[silberbush2013].

Solute mobility in soil is described by two main processes: 
1) convective transport by water mass flow through the transpiration stream, and 
2) movement driven by diffusion that depends on the solute concentration gradient in the soil caused by depletion of solutes due to root uptake \cite[barber62]. 
The first to recognize that solutes move towards the root due to the uptake process was \citeonline[bray].
Since then, the efforts to describe the solute movement in soils were directed to develop analytical and, with the improvement of computational power, numerical models.
%
The earlier analytical solutions for the CDE were formulated considering a steady-state condition to the water flow and solute uptake governed by concentration of the soil solution \cite[barber74,cushman,nye] or determined by a constant plant demand \cite[willigen81]. 
The concentration limiting approach may overestimates the uptake in situations where solute supply to the root is not limiting \cite[barraclough] whilst the constant plant demand formulation may overestimate the uptake in situations of very dry soil conditions and low concentration at root surface, where the diffusive flow prevails (cite papers about MM experiments -- the difference between the constant uptake and MM equation at low concentrations). 
Later, \citeonline[roosephd] formulated analytical solutions for different conditions of uptake, considering a so called `pseudo-steady-state' condition for water flow and the interference of root hairs and mycorrhizae in the uptake process, offering also an analytical solution to upscaling from the single root to a three-dimensional root architecture uptake model.

%Numerical solutions

%Talk about Solute extraction models (MM and others)
Solute uptake can be described by the Michaelis-Menten (MM) equation \cite[barber,barber81,schroder,simunek]
according to which uptake increases with increasing solute concentration asymptotically approaching maximum uptake (details in Section \ref[sol_uptake]). 
%Change this. pasted from rengel!!!!!!!!!
The necessary parameters are the MM constant ($K_m$), the maximum solute uptake ($I_m$) and the minimum solution concentration at which uptake ceases ($C_{min}$) \cite[barber]. 
%Kinetic parameters $I_m$ and $K_m$ vary with plant species, genotype, plant age, soil temperature, and nutritional status of the plant \cite[jungk,teo]. 
%Therefore, they would have to be determined for each experimental situation. 
%my annotations about MM here
\citeonline[epstein52] were the first to use the MM equation to represent the uptake of a solute and its concentration in external medium (for potassium and sodium solutions in barley roots) and it has been frequently used since then. 
It describes well the solute uptake for both anions \cite[epstein72,siddiqi,wang] and cations \cite[kochian,kelly,sadana,broadley,lux] in the low concentration range and, adding a linear component ($k$) to the equation, it can properly estimates the uptake rate also for high concentrations \cite[epstein72,kochian,borstlap,wang,vallejo,broadley].
Many authors agree that for low concentration in external medium, the uptake is driven by an active mechanism of the plant, as it occurs against the solute gradient between root and soil (Epstein's mechanism~I). 
For the high concentration range, solutes are freely transported from soil to roots by diffusion and occasional convection. This passive transport is known as Epstein's mechanism~II \cite[kochian,siddiqi]. 
Moreover, experiments have shown that a minimum concentration value ($C_{min}$) in which the uptake ceases to occur is often found \cite[mouat,dudal,machado].
Details on Epestein's mechanisms and its physiological processes, and on active and passive uptake, are found on \citeonline[epstein60] and \citeonline[fried].

The values of MM parameters are strongly dependent on the experimental methods used and vary with plant species, plant age, plant nutritional status, soil temperature and pH \cite[barber,shi]. Therefore, they have to be determined for each particular experimental situation.
Some types of experiments to determine the kinetic parameters $I_m$, $K_m$ and $C_{min}$ include hydroponically-grow plants \cite[barber] and the use of radioisotopes to estimate them directly from soil \cite[nye77]. 
The latter is more realistic since there is a large difference between a stirred nutrient solution and the complex and dynamic soil medium.
Measuring $C_{min}$ is particularly difficult \cite[seeling,lambers] because it occurs at very low concentration levels, being hard to be detected. \citeonline[seeling] shows that $C_{min}$ can be neglected for the cases of high $K_m$ values.

%Talk about other uptake functions
There are other alternatives to describe the solute uptake rate function apart from MM equation. 
\citeonline[dalton] proposed a physical-mathematical model that include active uptake, mass flow and diffusion, and express solute uptake as being proportional to root water uptake. 
\citeonline[bouldin] described the uptake from low and high concentrations solutions by two linear equations, thus simplifying the process of solute uptake at root surface to two diffusion components \cite[rengel].
\citeonline[tinker] determined the uptake rate as a linear component followed by a constant rate phase where the threshold value is dependent on solute concentration inside the plant.
Nevertheless, MM equation is yet the mostly used equation in root solute uptake models for being physically grounded and for its good agreement with experimental data.


%Talk about transpiration reduction functions (water, solute and water+solute). Talk about the quirijn's no uptake model and how it affects the combined water and solute reduction function without assume an additive or multiplicative interaction
Combined water and solute uptake models compute averages or locals pressure and osmotic heads, being able to estimate plant water and osmotic stress. 
%\citeonline[feddes78] developed an empirical water stress model relating the actual transpiration to pressure head originating a transpiration reduction function. 
\citeonline[feddes78] developed an empirical water stress model relating the actual transpiration to pressure head, originating a piecewise transpiration reduction function in which the potential transpiration is linearly reduced according to threshold values of pressure head that accounts for excess and shortage of water in soil.
%It is a piecewise function in which the potential transpiration is linearly reduced according to threshold values of pressure head that accounts for excess and shortage of water in soil.
As solute presence in soil solution contributes to reduce the total hydraulic head -- decreasing root water uptake -- the effect in transpiration reduction by water and salt stress are commonly related to be either additive \cite[vangenuchten87] or multiplicative \cite[vangenuchten87] in many models. 
However, crops seem to respond to water and salt stress differently. 

\citeonline[shalhevet] found that transpiration of cotton and pepper plants reduces more strongly under water stress then under an equivalent osmotic stress. 
Similar results have already been found by \citeonline[sepaskhah] for wheat, and by \citeonline[parra] for beans.
Thus, empirical weighting factors has been introduced in the transpiration reduction function to account for different response of a crop to osmotic and pressure heads.

Some models treat the combined water and salt stress as being neither additive or multiplicative. \citeonline[homaee] developed a piecewise reduction function similar to the \citeonline[feddes78] and concluded that the function fitted his experimental data better than any additive or multiplicative models.
\citeonline[liersolute] proposed a microscopic analytical model to estimate relative transpiration as a function of soil and plant parameters. 
Their mechanistic approach reduces the need of empirical parameters an does not require any explicit assumption (additive or multiplicative) to derive the combined water and salt stress. 
Despite the results are in good agreement with experimental data, it shows some discrepancies, especially under wet conditions. 
Their model do not consider root solute uptake.
As being mechanistic, considering it would lead to more realistic simulations and interfere in the solute concentration estimative, leading to different responses of the crop to the combined water and osmotic stress.
%One of their conclusions was that the relative transpiration reduces more gradually with combined water and osmotic stress than in a scenario with water stress only


%disadvantages of this


%reduced according to four threshold pressure heads values ($h_1$, $h_2$, $h_3$ and $h_4$). The actual transpiration reduces, linearly, between $h_3$ and $h_4$, and between $h_1$ and $h_2$ due to soil water shortage and soil oxygen shortage, respectively. For pressure heads values greater than $h_1$ and lower than $h_4$, the transpiration is zero, and between $h_2$ and $h_3$, the transpiration is potential.
%Figura feddes

%Talk about specific models: van Dam (water), quirijn (solute+water), cushman and willigen. All microscopic. Tell the limitations of them.
