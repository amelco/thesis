%% fontch.tex Macros for easily changing fonts and sizes in plain TeX %%
%% V2.2  2010/04/12 
%% Copyright 2010 Rodrigo Medina rmedina@ivic.gob.ve
%% Suggested location: .../texmf/tex/plain/fontch/
%% Licence: LPPL
%%
%%%%%% Components
%% The fontch V2.2 package is composed of the following files
%% README         -- Explanation file (has essentially this information)
%% fontch.tex     -- main macros
%% bsymbols.tex   -- macros for boldface  symbols
%% TS1mac.tex     -- macros for the TS1 companion symbols
%% DSmac.tex      -- auxiliary file for the DStroke option
%% fontch.pdf     -- manual to fontch
%% fontch_doc.tex -- source of manual
%%%%%% Description
%%
%%  Fontch implements a mechanism for changing type and size of fonts.  One
%% can write text and math at main sizes of 8, 10, 12, 14, 20 and 24 points.
%% Fonts of sizes 5, 6, 7 and 9 points are used in subscripts and subscripts of %% subscripts of other main sizes.
%% A single command makes all the changes needed for changing font smoothly.
%% In addition fontch gives support to:
%%   - boldface math-italic and boldface math symbols,
%%   - T1 Latin Modern fonts,
%%   - TS1 companion symbol fonts.
%%   - AMS fonts for boldface math,
%%   - Double Stroke fonts for blackboard bold symbols
%% 
%%%%%% Usage
%%
%% For using fontch.tex with the original TeX OT1 fonts (Computer Modern)
%% just put at the beginning of the document:
%% \input fontch.tex
%%
%% The file bsymbols.tex provides macro definitions for boldface versions
%% of math symbols. For using it just input the file where you need it.
%%
%% The fontch package has four options for handling different kinds of fonts:
%%
%%  -  LMTone     for Latin Modern T1 fonts.
%%  -  LMTSone    for the TS1 companion symbol font.
%%  -  AMSfont    for AMS fonts used for boldface math.
%%  -  DStroke    for Doublestroke fonts providing blackboard-boldface symbols.
%%
%% The four options are independent. They are activated by setting the
%% corresponding variable before calling  fontch.tex. For example for
%% activating all the four options and using the macros for bold math
%% put at the beginning of the document:
%%
%% \let\AMSfont\relax
%% \let\DStroke\relax 
%% \let\LMTone\relax
%% \let\LMTSone\relax
%% \input fontch.tex
%% \input bsymbols.tex
%%
%%
%%%%%% NOTES:
%% (1) There is one modification of the standard behavior of plain TeX. Plain
%%     TeX defines for the font #3 the same font (tenex) for text, script and
%%     scriptscript. That is awfull. The fontch package uses sevenex for
%%     script and fiveex for scriptscript, as for the other fonts.
%%     
%% (2) The use of files il1-t1.tcx or il2-t1.tcx is recommended together
%%     with the LMT1 fonts.
%%
%% (3) Only the cmcs, cmmib, cmbsy and cmex AMS fonts are supported. These
%%     are used for small caps and for boldface math-italic and boldface
%%     greek symbols. The other AMS fonts like the Euler, Cyrillic and extra
%%     symbols are not supported.
%%
%% (4) Only the roman Doublestroke font is supported. The sans serif version
%%     is not supported.
%%
%% (5) There are two main uses of boldface math. One case is the use of
%%     isolated bold  characters or symbols inside a formula. For this case
%%     fontch provides bold versions of math synbols and the macro \mb for bold
%%     math-italic characters.  The plain TeX macro "\bf" yields roman-bold
%%     characters in math mode.  The name of the bold version of a math symbol
%%     is obtained adding the prefix "bf" to the name of the normal symbol. For
%%     example a boldface italic "a" is {\mb a}, a boldface roman "P" is
%%     {\bf P}, a boldface \Sigma is \bfSigma, a boldface \iff is \bfiff.
%%
%%     The other case is when one wants to write a complete formula in
%%     boldface, for example inside a title. For this case fontch has the
%%     macro \setmathbold that changes the font families 1 (cmmi) and 2 (cmsy)
%%     to the bold versions cmmib and cmmbsy. The macro  \unsetmathbold resets
%%     the families 1 and 2 to their original values. These macros should be
%%     put before and after the formula.
%%
%% (6) Usually the macro \setmathbold does not change the whole formula to
%%     boldface. This is due to the fact that in math mode some symbols
%%     come from family #0 (cmr) or #3 (cmex) such as "+" or \int. Fontch
%%     provides macros for the symbols that come from family #0, For
%%     example the bold version of "(" is \bflparen.
%%
%% (7) Math symbols of family #3 (cmex) do not have bold versions. In particular
%%     \int, \sum and \prod. Nevertheless \smallint is of cmsy and does have
%%     a bold version. 
%%
%%%%%%%%% Commands of fontch %%%%%%
%%
%%%%%% General Commands
%  \rm  -- Roman
%  \sl  -- Slanted
%  \it  -- Italic
%  \bf  -- Boldface
%  \tt  -- Teletype
%  \sf  -- Sans Serif
%  \sc  -- Small Caps
%  \cal -- Calligraphic
%  \mit -- Math Italic
%  \mb  -- Math Boldface
%  \bcal          -- Boldface calligraphic
%  \oldstyle      -- Old style digits
%  \boldstyle     -- Boldface old style digits
%  \setmathbold   -- set families 1 and 2 to cmmib and cmmbsy
%  \unsetmathbold -- reset families 1 and 2 to cmmi and cmmsy
%  \eightpoint    -- Change to small type
%  \tenpoint      -- Change to normal type
%  \twelvepoint   -- Change to large type
%  \fourteenpoint -- Change to very large type
%  \twentypoint   -- Change to huge type
%  \twentyfourpoint -- Change to immense type
%  \doublespace   -- Double space for ten & twelve points
%
%%%%%%   Macros of bold symbols
%  Symbols that have a keyword 
%   bold_keyword = bf + normal_keyword
%
%  Symbols that do not have a keyword
%   \bfexcl        -- !
%   \bflparen      -- (
%   \bfrparen      -- )
%   \bfplus        -- +
%   \bfcomma       -- ,
%   \bfcolon       -- :
%   \bfsemicolon   -- ;
%   \bfequal       -- =
%   \bflbraket     -- [
%   \bfrbraket     -- ]
%   \bflt          -- <
%   \bfslash       -- /
%   \bfgt          -- >
%   \bfminus       -- -
%   \bfvert        -- |
%   \bfVert        -- \|
%
%%%%%%   Command of the DStroke option
%  \ds  -- Double Stroke
%
%%%%%%   Commands of the LMT1 option
%% Redefined Plain TeX Commands
%  \i , \j , \ae , \AE, \oe , \OE , \o , \O ,
%  \ss , \SS , \aa , \AA , \l , \L ,
%  \Gamma , \Delta , \Theta , \Lambda , \Xi , \Pi ,
%  \Sigma , \Upsilon , \Phi , \Psi , \Omega, \hbar
%%
%% New T1 Commands for non-latin1 characters
%  \lh  -- l'
%  \Lh  -- L'
%  \dh  -- d'
%  \th  -- t'
%  \dbar -- d slash
%  \Edh
%  \edh
%  \ij
%  \IJ
%  \nj
%  \NJ
%  \thorn
%  \Thorn
%  \smallzero
%  \lguille   -- left guillemot (does not appear in il1-t1.tcx)
%  \rguille   -- right guillemot (does not appear in il1-t1.tcx)
%  \vispace   -- visible space
%%
%% Redefined plain TeX accents
%  \` , \' , \^ , \~ , \" , \H , \v , \u , \= , \. , \b , \c
%% New T1 accents
%  \r   -- ring accent
%  \k   -- ogonek
%%
%% Redefined plain TeX math-accents
% \acute,\grave,\ddot,\tilde,\bar,\breve,\check,\hat,\dot
%% New T1 math-accent
%  \ring
%%
%%%%% Commands of the LMTS1 option %%
%% Text companion symbols TS1 %%
%  \tcchar{#1} -- Character of TS1 table
%% Redefined plain TeX commands
%  \t{#1} --  tie-after-accent
%  \P     --  pilcrow
%  \S     --  section
%  \dag   --  dagger
%  \ddag  --  double dagger
%  \copyright
%% Most TS1 symbols (in increasing charcode order)
%  \arrowl      -- left arrow
%  \arrowr      -- right arrow
%  \blank
%  \dollar
%  \caster      -- centered asterisk
%  \sequals     -- short equals
%  \anglel
%  \angler
%  \mho
%  \bigcircle
%  \Ohm
%  \bbracketl   -- left double square bracket
%  \bbracketr   -- right double square bracket
%  \arrowu      -- up arrow
%  \arrowd      -- down arrow
%  \textstar    -- star
%  \born
%  \died
%  \leaf
%  \married
%  \music
%  \doublevert  -- Double vertical bar
%  \perthousand
%  \bull        -- bullet (is in manmac)
%  \centigrade
%  \olddollar
%  \oldcent
%  \florin
%  \Colon
%  \won
%  \naira
%  \guarani
%  \peso
%  \lira
%  \recipe
%  \interrobang -- Question+Exclamation
%  \gnaborretni -- Question+Exclamation upsidedown
%  \dong
%  \TM
%  \pertenth    -- per ten thousand
%  \npilcrow    -- narrow pilcrow
%  \baht
%  \numero
%  \abzueglich
%  \aestimated  -- Big e
%  \openbull
%  \SM
%  \qbrackl     -- left quill bracket
%  \qbrackr     -- right quill bracket
%  \cent
%  \sterling
%  \currency
%  \yen
%  \brokenvert  -- broken vertical bar
%  \feminine
%  \copyleft
%  \lognot      -- logical not
%  \circledP
%  \registered
%  \degree
%  \plusminus
%  \twoup
%  \threeup
%  \micro
%  \centereddot
%  \reference
%  \oneup
%  \masculine
%  \root
%  \onequarter
%  \onehalf
%  \threequarters
%  \euro
%  \texttimes
%  \textdiv
%
%%%%%%%%%%%%%%%%%%%%%%%%%%%%%%%%%%%%%%%%%%%%%%%%%%%%%%%%
%%
%% Modifications from version 2.1
%% Version 2.2 is a minor bug correction version.
%% The main change is the inclusion of the fontch.pdf file.
%%
%% Modifications from version 2.0
%% Version 2.1 is a bug correction version
%% The main change is to conform with the new naming conventions
%% of the LM fonts files. Fontch-2.1 is compatible with LM fonts
%% of version 1.000 or larger.
%%
%% Modifications from version 1.3
%%
%% Version 2.0 is a mayor revision.
%%
%% There are many improvements such as
%% -- Proper handling of smallcaps
%% -- Support for bold-math
%% -- Support for AMS fonts
%% -- Support for blackboard boldface
%% -- Correction of bugs and scaling.
%%
%%  The behavior should be compatible with V1.3 except for
%%  case explained in Note (1).
%%%%%%%%%%%%%%%%%%%%%%%%%%%%%%%%%%%%%%%%%%%%%%%%%%%%%%%%
\wlog{File: fontch.tex
        [v2.2 (2010/04/12) font change package for plain TeX]}
%
%% Math fonts, valid for any option
%% five points fonts (defined in plain.tex) %%
%% six points fonts %%
\font\sixi=cmmi6
\font\sixsy=cmsy6
\skewchar\sixi='177
\skewchar\sixsy='60
%% seven points fonts (defined in plain.tex) %%
%% eight points fonts %%
\font\eighti=cmmi8
\font\eightsy=cmsy8
\skewchar\eighti='177
\skewchar\eightsy='60
%% nine points fonts %%
\font\ninei=cmmi9
\font\ninesy=cmsy9
\skewchar\ninei='177
\skewchar\ninesy='60
%% ten points fonts %%
\font\tenmib=cmmib10
\font\tenbsy=cmbsy10
\skewchar\tenmib='177
\skewchar\tenbsy='60
%% twelve points fonts %%
\font\twelvei=cmmi12
\font\twelvesy=cmsy10 scaled 1200
\font\twelveex=cmex10 scaled 1200
\font\twelvemib=cmmib10 scaled 1200
\font\twelvebsy=cmbsy10 scaled 1200
\skewchar\twelvei='177
\skewchar\twelvemib='177
\skewchar\twelvesy='60
\skewchar\twelvebsy='60
%% fourteen points fonts %%
\font\fourteeni=cmmi12 scaled 1167
\font\fourteensy=cmsy10 scaled 1400
\font\fourteenex=cmex10 scaled 1400
\font\fourteenmib=cmmib10 scaled 1400
\font\fourteenbsy=cmbsy10 scaled 1400
\skewchar\fourteeni='177
\skewchar\fourteenmib='177
\skewchar\fourteensy='60
\skewchar\fourteenbsy='60
%% twenty points fonts %%
\font\twentyi=cmmi12 scaled 1667
\font\twentysy=cmsy10 scaled 2000
\font\twentyex=cmex10 scaled 2000
\font\twentymib=cmmib10 scaled 2000
\font\twentybsy=cmbsy10 scaled 2000
\skewchar\twentyi='177
\skewchar\twentymib='177
\skewchar\twentysy='60
\skewchar\twentybsy='60
%% twentyfour  points fonts %%
\font\twentyfouri=cmmi12 scaled 2000
\font\twentyfoursy=cmsy10 scaled 2400
\font\twentyfourex=cmex10 scaled 2400
\font\twentyfourmib=cmmib10 scaled 2400
\font\twentyfourbsy=cmsy10 scaled 2400
\skewchar\twentyfouri='177
\skewchar\twentyfourmib='177
\skewchar\twentyfoursy='60
\skewchar\twentyfourbsy='60
%%
%% big... operators
\newdimen\bigsize
\newdimen\Bigsize
\newdimen\biggsize
\newdimen\Biggsize
%% Make @ visible
\catcode`\@=11
\def\big#1{{\hbox{$\left#1\vbox to \bigsize{}\right.\n@space$}}}
\def\Big#1{{\hbox{$\left#1\vbox to \Bigsize{}\right.\n@space$}}}
\def\bigg#1{{\hbox{$\left#1\vbox to \biggsize{}\right.\n@space$}}}
\def\Bigg#1{{\hbox{$\left#1\vbox to \Biggsize{}\right.\n@space$}}}
%%
\ifx\AMSfont\@undefined%
{}
\else
\let\ifAMS=\relax
\fi
%%
\ifx\DStroke\@undefined%
{}
\else
\let\ifDS=\relax
\fi
%%
\ifx\LMTone\@undefined%
{}
\else
\let\ifLMT=\relax
\fi
\ifx\LMTSone\@undefined%
{}
\else
\let\ifLMTS=\relax
\fi
%%
%% Make @ invisible
\catcode`\@=12
%%
\ifx\ifAMS\relax%
%% AMS option activated %%
\immediate\write16{*** AMS option activated. AMS fonts are used ***}
%% extension font
\font\sevenex=cmex7
\font\eightex=cmex8
\font\nineex=cmex9
\font\fiveex=cmex7 scaled 714
\font\sixex=cmex7 scaled 857
%% math-italic boldface font
\font\fivemib=cmmib5
\font\sixmib=cmmib6
\font\sevenmib=cmmib7
\font\eightmib=cmmib8
\font\ninemib=cmmib9
%% boldface symbols fonts
\font\fivebsy=cmbsy5
\font\sixbsy=cmbsy6
\font\sevenbsy=cmbsy7
\font\eightbsy=cmbsy8
\font\ninebsy=cmbsy9
\else
%% extension font
\font\fiveex=cmex10 scaled 500
\font\sixex=cmex10 scaled 600
\font\sevenex=cmex10 scaled 700
\font\eightex=cmex10 scaled 800
\font\nineex=cmex10 scaled 900
%% math-italic boldface font
\font\fivemib=cmmib10 scaled 500
\font\sixmib=cmmib10 scaled 600
\font\sevenmib=cmmib10 scaled 700
\font\eightmib=cmmib10 scaled 800
\font\ninemib=cmmib10 scaled 900
%% boldface symbols fonts
\font\fivebsy=cmbsy10 scaled 500
\font\sixbsy=cmbsy10 scaled 600
\font\sevenbsy=cmbsy10 scaled 700
\font\eightbsy=cmbsy10 scaled 800
\font\ninebsy=cmbsy10 scaled 900
\fi
%% End of AMS option
\skewchar\fivemib='177
\skewchar\sixmib='177
\skewchar\sevenmib='177
\skewchar\eightmib='177
\skewchar\ninemib='177
\skewchar\fivebsy='60
\skewchar\sixbsy='60
\skewchar\sevenbsy='60
\skewchar\eightbsy='60
\skewchar\ninebsy='60
%%
\ifx\ifLMT\relax%
%% T1 option activated %
\immediate\write16{*** T1 option activated. cork-LM fonts are used ***}
%% Replace CM text fonts %%
%% ten points fonts %%
\font\tenrm=ec-lmr10
\font\tenbf=ec-lmbx10
\font\tensl=ec-lmro10
\font\tentt=ec-lmtt10
\font\tenit=ec-lmri10
\font\tenss=ec-lmss10
\font\tencsc=ec-lmcsc10
%% eight points fonts %%
\font\eightrm=ec-lmr8
\font\eightbf=ec-lmbx8
\font\eightsl=ec-lmro8
\font\eighttt=ec-lmtt8
\font\eightit=ec-lmri8
\font\eightss=ec-lmss8
\font\eightcsc=ec-lmcsc10 at 8pt
%% five points fonts %%
\font\fiverm=ec-lmr5
\font\fivebf=ec-lmbx5
\font\fivess=ec-lmss8 scaled 625
%% six points fonts %%
\font\sixrm=ec-lmr6
\font\sixbf=ec-lmbx6
\font\sixss=ec-lmss8 scaled 750
%% seven points fonts %%
\font\sevenrm=ec-lmr7
\font\sevenbf=ec-lmbx7
\font\sevenss=ec-lmss8 scaled 875
%% nine points fonts %%
\font\ninerm=ec-lmr9
\font\ninebf=ec-lmbx9
\font\niness=ec-lmss9
%% twelve points fonts %%
\font\twelverm=ec-lmr12
\font\twelvebf=ec-lmbx12
\font\twelvesl=ec-lmro12
\font\twelvett=ec-lmtt12
\font\twelveit=ec-lmri12
\font\twelvess=ec-lmss12
\font\twelvecsc=ec-lmcsc10 scaled 1200
%% fourteen points fonts %%
\font\fourteenrm=ec-lmr12 scaled 1167
\font\fourteenbf=ec-lmbx12 scaled 1167
\font\fourteensl=ec-lmro12 scaled 1167
\font\fourteentt=ec-lmtt12 scaled 1167
\font\fourteenit=ec-lmri12 scaled 1167
\font\fourteenss=ec-lmss12 scaled 1167
\font\fourteencsc=ec-lmcsc10 scaled 1400
%% twenty points fonts %%
\font\twentyrm=ec-lmr12 scaled 1667
\font\twentybf=ec-lmbx12 scaled 1667
\font\twentysl=ec-lmro12 scaled 1667
\font\twentytt=ec-lmtt12 scaled 1667
\font\twentyit=ec-lmri12 scaled 1667
\font\twentyss=ec-lmss12 scaled 1667
\font\twentycsc=ec-lmcsc10 scaled 2000
%% twentyfour  points fonts %%
\font\twentyfourrm=ec-lmr12 scaled 2000
\font\twentyfourbf=ec-lmbx12 scaled 2000
\font\twentyfoursl=ec-lmro12 scaled 2000
\font\twentyfourtt=ec-lmtt12 scaled 2000
\font\twentyfourit=ec-lmri12 scaled 2000
\font\twentyfourss=ec-lmss12 scaled 2000
\font\twentyfourcsc=ec-lmcsc10 scaled 2400
%%
%% Redefinition of character commands %%
\chardef\i="19 \chardef\j="1A
\chardef\ae="E6 \chardef\AE="C6
\chardef\oe="F7 \chardef\OE="D7
\chardef\o="F8 \chardef\O="D8
\chardef\ss="FF \chardef\SS="DF
\chardef\aa="E5 \chardef\AA="C5
\chardef\l="AA \chardef\L="8A
\mathchardef\Gamma="0100
\mathchardef\Delta="0101
\mathchardef\Theta="0102
\mathchardef\Lambda="0103
\mathchardef\Xi="0104
\mathchardef\Pi="0105
\mathchardef\Sigma="0106
\mathchardef\Upsilon="0107
\mathchardef\Phi="0108
\mathchardef\Psi="0109
\mathchardef\Omega="010A
\def\hbar{{\mathchar"7009\mkern-9muh}}

%% New character commands %%
\chardef\lh="A9 \chardef\Lh="89
\chardef\dh="A4 \chardef\th="B4
\chardef\dbar="9E
\chardef\Edh="D0 \chardef\edh="F0
\chardef\ij="BC \chardef\IJ="9C
\chardef\nj="AD \chardef\NJ="8D
\chardef\thorn="FE \chardef\Thorn="DE
\chardef\smallzero="18
\chardef\lguille="E \chardef\rguille="F
\chardef\vispace="20

%% Redefinition of accents and new accents%%
\def\`#1{{\accent0 #1}}
\def\'#1{{\accent1 #1}}
\def\^#1{{\accent2 #1}}
\def\~#1{{\accent3 #1}}
\def\"#1{{\accent4 #1}}
\def\H#1{{\accent5 #1}}
%New ring accent
\def\r#1{{\accent6 #1}}
\def\v#1{{\accent7 #1}}
\def\u#1{{\accent8 #1}}
\def\=#1{{\accent9 #1}}
\def\.#1{{\accent10 #1}}
\def\b#1{\oalign{#1\crcr\hidewidth
    \vbox to.2ex{\hbox{\char9}\vss}\hidewidth}}
\def\c#1{\setbox0=\hbox{#1}\ifdim\ht0=1ex \accent"B #1%
     \else{\ooalign{\unhbox0\crcr\hidewidth\char"B\hidewidth}}\fi}
%New ogonek accent
\def\k#1{\setbox0=\hbox{#1}\ifdim\ht0=1ex \accent"C #1%
     \else{\ooalign{\unhbox0\crcr\hidewidth\char"C\hidewidth}}\fi}
\def\textdiv{{\tcchar{"F6}}}
%
%% Redefinition of math-accents %%
\def\acute{\mathaccent"7001 }
\def\grave{\mathaccent"7000 }
\def\ddot{\mathaccent"7004 }
\def\tilde{\mathaccent"7003 }
\def\bar{\mathaccent"7009 }
\def\breve{\mathaccent"7008 }
\def\check{\mathaccent"7007 }
\def\hat{\mathaccent"7002 }
\def\dot{\mathaccent"700A }
%New ring math-accent
\def\ring{\mathaccent"7006 }
%
%% Activate non-ascii letters %%
% upper case 
\catcode`\^^80=11 \uccode`\^^80=`\^^80 \lccode`\^^80=`\^^a0
\catcode`\^^81=11 \uccode`\^^81=`\^^81 \lccode`\^^81=`\^^a1
\catcode`\^^82=11 \uccode`\^^82=`\^^82 \lccode`\^^82=`\^^a2
\catcode`\^^83=11 \uccode`\^^83=`\^^83 \lccode`\^^83=`\^^a3
\catcode`\^^84=11 \uccode`\^^84=`\^^84 \lccode`\^^84=`\^^a4
\catcode`\^^85=11 \uccode`\^^85=`\^^85 \lccode`\^^85=`\^^a5
\catcode`\^^86=11 \uccode`\^^86=`\^^86 \lccode`\^^86=`\^^a6
\catcode`\^^87=11 \uccode`\^^87=`\^^87 \lccode`\^^87=`\^^a7
\catcode`\^^88=11 \uccode`\^^88=`\^^88 \lccode`\^^88=`\^^a8
\catcode`\^^89=11 \uccode`\^^89=`\^^89 \lccode`\^^89=`\^^a9
\catcode`\^^8a=11 \uccode`\^^8a=`\^^8a \lccode`\^^8a=`\^^aa
\catcode`\^^8b=11 \uccode`\^^8b=`\^^8b \lccode`\^^8b=`\^^ab
\catcode`\^^8c=11 \uccode`\^^8c=`\^^8c \lccode`\^^8c=`\^^ac
\catcode`\^^8d=11 \uccode`\^^8d=`\^^8d \lccode`\^^8d=`\^^ad
\catcode`\^^8e=11 \uccode`\^^8e=`\^^8e \lccode`\^^8e=`\^^ae
\catcode`\^^8f=11 \uccode`\^^8f=`\^^8f \lccode`\^^8f=`\^^af
\catcode`\^^90=11 \uccode`\^^90=`\^^90 \lccode`\^^90=`\^^b0
\catcode`\^^91=11 \uccode`\^^91=`\^^91 \lccode`\^^91=`\^^b1
\catcode`\^^92=11 \uccode`\^^92=`\^^92 \lccode`\^^92=`\^^b2
\catcode`\^^93=11 \uccode`\^^93=`\^^93 \lccode`\^^93=`\^^b3
\catcode`\^^94=11 \uccode`\^^94=`\^^94 \lccode`\^^94=`\^^b4
\catcode`\^^95=11 \uccode`\^^95=`\^^95 \lccode`\^^95=`\^^b5
\catcode`\^^96=11 \uccode`\^^96=`\^^96 \lccode`\^^96=`\^^b6
\catcode`\^^97=11 \uccode`\^^97=`\^^97 \lccode`\^^97=`\^^b7
\catcode`\^^98=11 \uccode`\^^98=`\^^98 \lccode`\^^98=`\^^b8
\catcode`\^^99=11 \uccode`\^^99=`\^^99 \lccode`\^^99=`\^^b9
\catcode`\^^9a=11 \uccode`\^^9a=`\^^9a \lccode`\^^9a=`\^^ba
\catcode`\^^9b=11 \uccode`\^^9b=`\^^9b \lccode`\^^9b=`\^^bb
\catcode`\^^9c=11 \uccode`\^^9c=`\^^9c \lccode`\^^9c=`\^^bc
\catcode`\^^c0=11 \uccode`\^^c0=`\^^c0 \lccode`\^^c0=`\^^e0
\catcode`\^^c1=11 \uccode`\^^c1=`\^^c1 \lccode`\^^c1=`\^^e1
\catcode`\^^c2=11 \uccode`\^^c2=`\^^c2 \lccode`\^^c2=`\^^e2
\catcode`\^^c3=11 \uccode`\^^c3=`\^^c3 \lccode`\^^c3=`\^^e3
\catcode`\^^c4=11 \uccode`\^^c4=`\^^c4 \lccode`\^^c4=`\^^e4
\catcode`\^^c5=11 \uccode`\^^c5=`\^^c5 \lccode`\^^c5=`\^^e5
\catcode`\^^c6=11 \uccode`\^^c6=`\^^c6 \lccode`\^^c6=`\^^e6
\catcode`\^^c7=11 \uccode`\^^c7=`\^^c7 \lccode`\^^c7=`\^^e7
\catcode`\^^c8=11 \uccode`\^^c8=`\^^c8 \lccode`\^^c8=`\^^e8
\catcode`\^^c9=11 \uccode`\^^c9=`\^^c9 \lccode`\^^c9=`\^^e9
\catcode`\^^ca=11 \uccode`\^^ca=`\^^ca \lccode`\^^ca=`\^^ea
\catcode`\^^cb=11 \uccode`\^^cb=`\^^cb \lccode`\^^cb=`\^^eb
\catcode`\^^cc=11 \uccode`\^^cc=`\^^cc \lccode`\^^cc=`\^^ec
\catcode`\^^cd=11 \uccode`\^^cd=`\^^cd \lccode`\^^cd=`\^^ed
\catcode`\^^ce=11 \uccode`\^^ce=`\^^ce \lccode`\^^ce=`\^^ee
\catcode`\^^cf=11 \uccode`\^^cf=`\^^cf \lccode`\^^cf=`\^^ef
\catcode`\^^d0=11 \uccode`\^^d0=`\^^d0 \lccode`\^^d0=`\^^f0
\catcode`\^^d1=11 \uccode`\^^d1=`\^^d1 \lccode`\^^d1=`\^^f1
\catcode`\^^d2=11 \uccode`\^^d2=`\^^d2 \lccode`\^^d2=`\^^f2
\catcode`\^^d3=11 \uccode`\^^d3=`\^^d3 \lccode`\^^d3=`\^^f3
\catcode`\^^d4=11 \uccode`\^^d4=`\^^d4 \lccode`\^^d4=`\^^f4
\catcode`\^^d5=11 \uccode`\^^d5=`\^^d5 \lccode`\^^d5=`\^^f5
\catcode`\^^d6=11 \uccode`\^^d6=`\^^d6 \lccode`\^^d6=`\^^f6
\catcode`\^^d7=11 \uccode`\^^d7=`\^^d7 \lccode`\^^d7=`\^^f7
\catcode`\^^d8=11 \uccode`\^^d8=`\^^d8 \lccode`\^^d8=`\^^f8
\catcode`\^^d9=11 \uccode`\^^d9=`\^^d9 \lccode`\^^d9=`\^^f9
\catcode`\^^da=11 \uccode`\^^da=`\^^da \lccode`\^^da=`\^^fa
\catcode`\^^db=11 \uccode`\^^db=`\^^db \lccode`\^^db=`\^^fb
\catcode`\^^dc=11 \uccode`\^^dc=`\^^dc \lccode`\^^dc=`\^^fc
\catcode`\^^dd=11 \uccode`\^^dd=`\^^dd \lccode`\^^dd=`\^^fd
\catcode`\^^de=11 \uccode`\^^de=`\^^de \lccode`\^^de=`\^^fe
\catcode`\^^df=11 \uccode`\^^df=`\^^df \lccode`\^^df=`\^^ff
% Lower case
\catcode`\^^a0=11 \uccode`\^^a0=`\^^80 \lccode`\^^a0=`\^^a0
\catcode`\^^a1=11 \uccode`\^^a1=`\^^81 \lccode`\^^a1=`\^^a1
\catcode`\^^a2=11 \uccode`\^^a2=`\^^82 \lccode`\^^a2=`\^^a2
\catcode`\^^a3=11 \uccode`\^^a3=`\^^83 \lccode`\^^a3=`\^^a3
\catcode`\^^a4=11 \uccode`\^^a4=`\^^84 \lccode`\^^a4=`\^^a4
\catcode`\^^a5=11 \uccode`\^^a5=`\^^85 \lccode`\^^a5=`\^^a5
\catcode`\^^a6=11 \uccode`\^^a6=`\^^86 \lccode`\^^a6=`\^^a6
\catcode`\^^a7=11 \uccode`\^^a7=`\^^87 \lccode`\^^a7=`\^^a7
\catcode`\^^a8=11 \uccode`\^^a8=`\^^88 \lccode`\^^a8=`\^^a8
\catcode`\^^a9=11 \uccode`\^^a9=`\^^89 \lccode`\^^a9=`\^^a9
\catcode`\^^aa=11 \uccode`\^^aa=`\^^8a \lccode`\^^aa=`\^^aa
\catcode`\^^ab=11 \uccode`\^^ab=`\^^8b \lccode`\^^ab=`\^^ab
\catcode`\^^ac=11 \uccode`\^^ac=`\^^8c \lccode`\^^ac=`\^^ac
\catcode`\^^ad=11 \uccode`\^^ad=`\^^8d \lccode`\^^ad=`\^^ad
\catcode`\^^ae=11 \uccode`\^^ae=`\^^8e \lccode`\^^ae=`\^^ae
\catcode`\^^af=11 \uccode`\^^af=`\^^8f \lccode`\^^af=`\^^af
\catcode`\^^b0=11 \uccode`\^^b0=`\^^90 \lccode`\^^b0=`\^^b0
\catcode`\^^b1=11 \uccode`\^^b1=`\^^91 \lccode`\^^b1=`\^^b1
\catcode`\^^b2=11 \uccode`\^^b2=`\^^92 \lccode`\^^b2=`\^^b2
\catcode`\^^b3=11 \uccode`\^^b3=`\^^93 \lccode`\^^b3=`\^^b3
\catcode`\^^b4=11 \uccode`\^^b4=`\^^94 \lccode`\^^b4=`\^^b4
\catcode`\^^b5=11 \uccode`\^^b5=`\^^95 \lccode`\^^b5=`\^^b5
\catcode`\^^b6=11 \uccode`\^^b6=`\^^96 \lccode`\^^b6=`\^^b6
\catcode`\^^b7=11 \uccode`\^^b7=`\^^97 \lccode`\^^b7=`\^^b7
\catcode`\^^b8=11 \uccode`\^^b8=`\^^98 \lccode`\^^b8=`\^^b8
\catcode`\^^b9=11 \uccode`\^^b9=`\^^99 \lccode`\^^b9=`\^^b9
\catcode`\^^ba=11 \uccode`\^^ba=`\^^9a \lccode`\^^ba=`\^^ba
\catcode`\^^bb=11 \uccode`\^^bb=`\^^9b \lccode`\^^bb=`\^^bb
\catcode`\^^e0=11 \uccode`\^^e0=`\^^c0 \lccode`\^^e0=`\^^e0
\catcode`\^^e1=11 \uccode`\^^e1=`\^^c1 \lccode`\^^e1=`\^^e1
\catcode`\^^e2=11 \uccode`\^^e2=`\^^c2 \lccode`\^^e2=`\^^e2
\catcode`\^^e3=11 \uccode`\^^e3=`\^^c3 \lccode`\^^e3=`\^^e3
\catcode`\^^e4=11 \uccode`\^^e4=`\^^c4 \lccode`\^^e4=`\^^e4
\catcode`\^^e5=11 \uccode`\^^e5=`\^^c5 \lccode`\^^e5=`\^^e5
\catcode`\^^e6=11 \uccode`\^^e6=`\^^c6 \lccode`\^^e6=`\^^e6
\catcode`\^^e7=11 \uccode`\^^e7=`\^^c7 \lccode`\^^e7=`\^^e7
\catcode`\^^e8=11 \uccode`\^^e8=`\^^c8 \lccode`\^^e8=`\^^e8
\catcode`\^^e9=11 \uccode`\^^e9=`\^^c9 \lccode`\^^e9=`\^^e9
\catcode`\^^ea=11 \uccode`\^^ea=`\^^ca \lccode`\^^ea=`\^^ea
\catcode`\^^eb=11 \uccode`\^^eb=`\^^cb \lccode`\^^eb=`\^^eb
\catcode`\^^ec=11 \uccode`\^^ec=`\^^cc \lccode`\^^ec=`\^^ec
\catcode`\^^ed=11 \uccode`\^^ed=`\^^cd \lccode`\^^ed=`\^^ed
\catcode`\^^ee=11 \uccode`\^^ee=`\^^ce \lccode`\^^ee=`\^^ee
\catcode`\^^ef=11 \uccode`\^^ef=`\^^cf \lccode`\^^ef=`\^^ef
\catcode`\^^f0=11 \uccode`\^^f0=`\^^d0 \lccode`\^^f0=`\^^f0
\catcode`\^^f1=11 \uccode`\^^f1=`\^^d1 \lccode`\^^f1=`\^^f1
\catcode`\^^f2=11 \uccode`\^^f2=`\^^d2 \lccode`\^^f2=`\^^f2
\catcode`\^^f3=11 \uccode`\^^f3=`\^^d3 \lccode`\^^f3=`\^^f3
\catcode`\^^f4=11 \uccode`\^^f4=`\^^d4 \lccode`\^^f4=`\^^f4
\catcode`\^^f5=11 \uccode`\^^f5=`\^^d5 \lccode`\^^f5=`\^^f5
\catcode`\^^f6=11 \uccode`\^^f6=`\^^d6 \lccode`\^^f6=`\^^f6
\catcode`\^^f7=11 \uccode`\^^f7=`\^^d7 \lccode`\^^f7=`\^^f7
\catcode`\^^f8=11 \uccode`\^^f8=`\^^d8 \lccode`\^^f8=`\^^f8
\catcode`\^^f9=11 \uccode`\^^f9=`\^^d9 \lccode`\^^f9=`\^^f9
\catcode`\^^fa=11 \uccode`\^^fa=`\^^da \lccode`\^^fa=`\^^fa
\catcode`\^^fb=11 \uccode`\^^fb=`\^^db \lccode`\^^fb=`\^^fb
\catcode`\^^fc=11 \uccode`\^^fc=`\^^dc \lccode`\^^fc=`\^^fc
\catcode`\^^fd=11 \uccode`\^^fd=`\^^dd \lccode`\^^fd=`\^^fd
\catcode`\^^fe=11 \uccode`\^^fe=`\^^de \lccode`\^^fe=`\^^fe
\catcode`\^^ff=11 \uccode`\^^ff=`\^^df \lccode`\^^ff=`\^^ff
%%End of LMTone option %%
\else
%%
%% Other CM fonts
%% eight points fonts %%
\font\eightrm=cmr8
\font\eightbf=cmbx8
\font\eightsl=cmsl8
\font\eighttt=cmtt8
\font\eightit=cmti8
\font\eightss=cmss8
%% five points fonts %%
\font\fivess=cmss8 scaled 625
%% six points fonts %%
\font\sixrm=cmr6
\font\sixbf=cmbx6
\font\sixss=cmss8 scaled 750
%% seven points fonts %%
\font\sevenss=cmss8 scaled 875
%% nine points fonts %%
\font\ninerm=cmr9
\font\ninebf=cmbx9
\font\niness=cmss9
%% ten points fonts (not defined in plain.tex)%%
\font\tenss=cmss10
\font\tencsc=cmcsc10
%% twelve points fonts %%
\font\twelverm=cmr12
\font\twelvebf=cmbx12
\font\twelvesl=cmsl12
\font\twelvett=cmtt12
\font\twelveit=cmti12
\font\twelvess=cmss12
\font\twelvecsc=cmcsc10 scaled 1200
%% fourteen points fonts %%
\font\fourteenrm=cmr12 scaled 1167
\font\fourteenbf=cmbx12 scaled 1167
\font\fourteensl=cmsl12 scaled 1167
\font\fourteentt=cmtt12 scaled 1167
\font\fourteenit=cmti12 scaled 1167
\font\fourteenss=cmss12 scaled 1167
\font\fourteencsc=cmcsc10 scaled 1400
%% twenty points fonts %%
\font\twentyrm=cmr12 scaled 1667
\font\twentybf=cmbx12 scaled 1667
\font\twentysl=cmsl12 scaled 1667
\font\twentytt=cmtt12 scaled 1667
\font\twentyit=cmti12 scaled 1667
\font\twentyss=cmss12 scaled 1667
\font\twentycsc=cmcsc10 scaled 2000
%% twentyfour  points fonts %%
\font\twentyfourrm=cmr12 scaled 2000
\font\twentyfourbf=cmbx12 scaled 2000
\font\twentyfoursl=cmsl12 scaled 2000
\font\twentyfourtt=cmtt12 scaled 2000
\font\twentyfourit=cmti12 scaled 2000
\font\twentyfourss=cmss12 scaled 2000
\font\twentyfourcsc=cmcsc10 scaled 2400
%% AMS activated
\ifx\ifAMS\relax%
%% small caps font
\font\eightcsc=cmcsc8
\else
%% small caps font
\font\eightcsc=cmcsc10 scaled 800
\fi
%% end of AMS option
\fi
%% end of LMT1 option
%% Support of TS1 fonts
\ifx\ifLMTS\relax%
\immediate\write16{*** TS1 option activated. ts1-LM fonts are used ***}
\input TS1mac.tex
\fi
%% End of LMTSone option %%
%% DStroke option %%
\ifx\ifDS\relax%
\immediate\write16{*** Doublestroke option activated. DS fonts used ***}
\font\fiveds=dsrom8 scaled 625
\font\sixds=dsrom8 scaled 750
\font\sevends=dsrom8 scaled 875
\font\eightds=dsrom8
\font\nineds=dsrom10 scaled 900
\font\tends=dsrom10
\font\twelveds=dsrom12
\font\fourteends=dsrom12 scaled 1167
\font\twentyds=dsrom12 scaled 1667
\font\twentyfourds=dsrom12 scaled 2000
\fi
%% End of DStroke option %%
%%
%% Families defined in plain.tex %%
%% fam0 rm roman
%% fam1 i  math-italic
%% fam2 sy symbols
%% fam3 ex extension fonts
%% fam4 itfam text-italic
%% fam5 slfam slanted
%% fam6 bffam roman boldface
%% fam7 ttfam teletype
%%
%% New family math-italic bold. Must be fam8 %%
\newfam\mibfam
%%
%% New family symbols boldface. Must be fam9 %%
\newfam\bsyfam
%%
%% New family sans serif  %%
\newfam\ssfam
%%
%% New family small caps %%
\newfam\cscfam
%%
%% New family doublestroke %%
\ifx\ifDS\relax%
\input DSmac.tex
\fi
\def\ds{}
%%
%% Change to tenpoint fonts %%
%% Correction: added a % at the end of each line and delete space before \def 
%% to prevent spaces in command running %% added 2014-10-28, Luciano R S
\def\tenpoint{%
\textfont0=\tenrm \scriptfont0=\sevenrm \scriptscriptfont0=\fiverm%
\textfont1=\teni \scriptfont1=\seveni \scriptscriptfont1=\fivei%
\def\mit{\fam1}\def\oldstyle{\fam1 \teni}%
\textfont2=\tensy \scriptfont2=\sevensy \scriptscriptfont2=\fivesy%
\def\cal{\fam2}%
\textfont3=\tenex \scriptfont3=\sevenex \scriptscriptfont3=\fiveex%
\textfont\itfam=\tenit%
\textfont\slfam=\tensl%
\textfont\ttfam=\tentt%
\textfont\ssfam=\tenss%
\scriptfont\ssfam=\sevenss \scriptscriptfont\ssfam=\fivess%
\textfont\bffam=\tenbf%
\scriptfont\bffam=\sevenbf \scriptscriptfont\bffam=\fivebf%
\textfont\cscfam=\tencsc%
\textfont\mibfam=\tenmib%
\scriptfont\mibfam=\sevenmib \scriptscriptfont\bffam=\fivemib%
\textfont\bsyfam=\tenbsy%
\scriptfont\bsyfam=\sevenbsy \scriptscriptfont\bffam=\fivebsy%
\ifx\ifDS\relax%
\textfont\dsfam=\tends%
\scriptfont\dsfam=\sevends \scriptscriptfont\dsfam=\fiveds%
\def\ds{\fam\dsfam\tends}%
\else%
{}%
\fi%
\ifx\ifLMTS\relax%
\def\rm{\fam0 \tenrm \let\tcfont=\tencrm}%
\def\it{\fam\itfam\tenit \let\tcfont=\tencit}%
\def\sl{\fam\slfam\tensl \let\tcfont=\tencsl}%
\def\tt{\fam\ttfam\tentt \let\tcfont=\tenctt}%
\def\sf{\fam\ssfam\tenss \let\tcfont=\tencss}%
\def\bf{\fam\bffam\tenbf \let\tcfont=\tencbf}%
\else%
\def\rm{\fam0 \tenrm}%
\def\it{\fam\itfam\tenit}%
\def\sl{\fam\slfam\tensl}%
\def\tt{\fam\ttfam\tentt}%
\def\sf{\fam\ssfam\tenss}%
\def\bf{\fam\bffam\tenbf}%
\fi%
\def\sc{\fam\cscfam\tencsc}%
\def\mb{\fam\mibfam}\def\boldstyle{\fam\mibfam\tenmib}%
\def\bcal{\fam\bsyfam\tenbsy}%
\def\unsetmathbold{%
\textfont1=\teni \scriptfont1=\seveni \scriptscriptfont1=\fivei%
\textfont2=\tensy \scriptfont2=\sevensy \scriptscriptfont2=\fivesy%
\def\oldstyle{\fam1 \teni} }%
\def\setmathbold{%
\textfont1=\tenmib \scriptfont1=\sevenmib \scriptscriptfont1=\fivemib%
\textfont2=\tenbsy \scriptfont2=\sevenbsy \scriptscriptfont2=\fivebsy%
\def\oldstyle{\fam1 \tenmib}}%
\normalbaselineskip=12pt minus 1pt%
\setbox\strutbox=\hbox{\vrule height8.5pt depth3.5pt width0pt}%
\bigsize=8.5pt%
\Bigsize=11.5pt%
\biggsize=14.5pt%
\Biggsize=17.5pt%
\normalbaselines\rm}%
%%
%% Change to twelve points fonts %%
%% Correction: added a % at the end of each line and delete space before \def 
%% to prevent spaces in command running %% added 2014-10-28, Luciano R S
\def\twelvepoint{%
\textfont0=\twelverm \scriptfont0=\eightrm \scriptscriptfont0=\sixrm%
\textfont1=\twelvei \scriptfont1=\eighti \scriptscriptfont1=\sixi%
\def\mit{\fam1 }\def\oldstyle{\fam1 \twelvei}%
\textfont2=\twelvesy \scriptfont2=\eightsy \scriptscriptfont2=\sixsy%
\def\cal{\fam2 }%
\textfont3=\twelveex \scriptfont3=\eightex \scriptscriptfont3=\sixex%
\textfont\itfam=\twelveit%
\textfont\slfam=\twelvesl%
\textfont\ttfam=\twelvett%
\textfont\ssfam=\twelvess%
\scriptfont\ssfam=\eightss \scriptscriptfont\ssfam=\sixss%
\textfont\bffam=\twelvebf%
\scriptfont\bffam=\eightbf \scriptscriptfont\bffam=\sixbf%
\textfont\cscfam=\twelvecsc%
\textfont\mibfam=\twelvemib%
\scriptfont\mibfam=\eightmib \scriptscriptfont\mibfam=\sixmib%
\textfont\bsyfam=\twelvebsy%
\scriptfont\bsyfam=\eightbsy \scriptscriptfont\bsyfam=\sixbsy%
\ifx\ifDS\relax%
\textfont\dsfam=\twelveds%
\scriptfont\dsfam=\eightds \scriptscriptfont\dsfam=\sixds%
\def\ds{\fam\dsfam\twelveds}%
\else%
{}%
\fi%
\ifx\ifLMTS\relax%
\def\rm{\fam0 \twelverm \let\tcfont=\twelvecrm}%
\def\it{\fam\itfam\twelveit \let\tcfont=\twelvecit}%
\def\sl{\fam\slfam\twelvesl \let\tcfont=\twelvecsl}%
\def\tt{\fam\ttfam\twelvett \let\tcfont=\twelvectt}%
\def\sf{\fam\ssfam\twelvess \let\tcfont=\twelvecss}%
\def\bf{\fam\bffam\twelvebf \let\tcfont=\twelvecss}%
\else%
\def\rm{\fam0 \twelverm}%
\def\it{\fam\itfam\twelveit}%
\def\sl{\fam\slfam\twelvesl}%
\def\tt{\fam\ttfam\twelvett}%
\def\sf{\fam\ssfam\twelvess}%
\def\bf{\fam\bffam\twelvebf}%
\fi%
\def\sc{\fam\cscfam\twelvecsc}%
\def\mb{\fam\mibfam}\def\boldstyle{\fam\mibfam\twelvemib}%
\def\bcal{\fam\bsyfam}%
\def\unsetmathbold{%
\textfont1=\twelvei \scriptfont1=\eighti \scriptscriptfont1=\sixi%
\textfont2=\twelvesy \scriptfont2=\eightsy \scriptscriptfont2=\sixsy%
\def\oldstyle{\fam1 \twelvei} }%
\def\setmathbold{%
\textfont1=\twelvemib \scriptfont1=\eightmib \scriptscriptfont1=\sixmib%
\textfont2=\twelvebsy \scriptfont2=\eightbsy \scriptscriptfont2=\sixbsy%
\def\oldstyle{\fam1 \twelvemib} }%
\normalbaselineskip=14pt minus 1pt%
\setbox\strutbox=\hbox{\vrule height10pt depth4pt width0pt}%
\bigsize=10.2pt%
\Bigsize=13.8pt%
\biggsize=17.4pt%
\Biggsize=21pt%
\normalbaselines\rm}%
%%
%% Change to eight points fonts %%
\def\eightpoint{
\textfont0=\eightrm \scriptfont0=\sixrm \scriptscriptfont0=\fiverm
\textfont1=\eighti \scriptfont1=\sixi \scriptscriptfont1=\fivei
\def\mit{\fam1 } \def\oldstyle{\fam1 \eighti}
\textfont2=\eightsy \scriptfont2=\sixsy \scriptscriptfont2=\fivesy
\def\cal{\fam2 }
\textfont3=\eightex \scriptfont3=\sixex \scriptscriptfont3=\fiveex
\textfont\itfam=\eightit
\textfont\slfam=\eightsl
\textfont\ttfam=\eighttt
\textfont\ssfam=\eightss
\scriptfont\ssfam=\sixss \scriptscriptfont\ssfam=\fivess
\textfont\bffam=\eightbf
\scriptfont\bffam=\sixbf \scriptscriptfont\bffam=\fivebf
\textfont\cscfam=\eightcsc
\textfont\mibfam=\eightmib
\scriptfont\mibfam=\sixmib \scriptscriptfont\mibfam=\fivemib
\textfont\bsyfam=\eightbsy
\scriptfont\bsyfam=\sixbsy \scriptscriptfont\bsyfam=\fivebsy
\ifx\ifDS\relax%
\textfont\dsfam=\eightds
\scriptfont\dsfam=\sixds \scriptscriptfont\dsfam=\fiveds
\def\ds{\fam\dsfam\eightds}
\else
{}
\fi
\ifx\ifLMTS\relax%
\def\rm{\fam0 \eightrm \let\tcfont=\eightcrm}
\def\it{\fam\itfam\eightit \let\tcfont=\eightcit}
\def\sl{\fam\slfam\eightsl \let\tcfont=\eightcsl}
\def\tt{\fam\ttfam\eighttt \let\tcfont=\eightctt}
\def\sf{\fam\ssfam\eightss \let\tcfont=\eightcss}
\def\bf{\fam\bffam\eightbf \let\tcfont=\eightcbf}
\else
\def\rm{\fam0 \eightrm}
\def\it{\fam\itfam\eightit}
\def\sl{\fam\slfam\eightsl}
\def\tt{\fam\ttfam\eighttt}
\def\sf{\fam\ssfam\eightss}
\def\bf{\fam\bffam\eightbf}
\fi
\def\sc{\fam\cscfam\eightcsc}
\def\mb{\fam\mibfam} \def\boldstyle{\fam\mibfam\eightmib}
\def\bcal{\fam\bsyfam}
\def\unsetmathbold{
\textfont1=\eighti \scriptfont1=\sixi \scriptscriptfont1=\fivei
\textfont2=\eightsy \scriptfont2=\sixsy \scriptscriptfont2=\fivesy
\def\oldstyle{\fam1 \eighti} }
\def\setmathbold{
\textfont1=\eightmib \scriptfont1=\sixmib \scriptscriptfont1=\fivemib
\textfont2=\eightbsy \scriptfont2=\sixbsy \scriptscriptfont2=\fivebsy
\def\oldstyle{\fam1 \eightmib} }
\normalbaselineskip=9pt
\setbox\strutbox=\hbox{\vrule height7pt depth2pt width0pt}
\bigsize=6.8pt
\Bigsize=9.2pt
\biggsize=11.6pt
\Biggsize=14pt
\normalbaselines\rm}
%%
%% Change to fourteen points fonts %%
\def\fourteenpoint{
\textfont0=\fourteenrm \scriptfont0=\ninerm \scriptscriptfont0=\sevenrm
\textfont1=\fourteeni \scriptfont1=\ninei \scriptscriptfont1=\seveni
\def\mit{\fam1 } \def\oldstyle{\fam1 \fourteeni}
\textfont2=\fourteensy \scriptfont2=\ninesy \scriptscriptfont2=\sevensy
\def\cal{\fam2 }
\textfont3=\fourteenex \scriptfont3=\nineex \scriptscriptfont3=\sevenex
\textfont\itfam=\fourteenit
\textfont\slfam=\fourteensl
\textfont\ttfam=\fourteentt
\textfont\ssfam=\fourteenss
\scriptfont\ssfam=\niness \scriptscriptfont\ssfam=\sevenss
\textfont\bffam=\fourteenbf
\scriptfont\bffam=\ninebf \scriptscriptfont\bffam=\sevenbf
\textfont\cscfam=\fourteencsc
\textfont\mibfam=\fourteenmib
\scriptfont\mibfam=\ninemib \scriptscriptfont\mibfam=\sevenmib
\textfont\bsyfam=\fourteenbsy
\scriptfont\bsyfam=\ninebsy \scriptscriptfont\bsyfam=\sevenbsy
\ifx\ifDS\relax%
\textfont\dsfam=\fourteends
\scriptfont\dsfam=\nineds \scriptscriptfont\dsfam=\sevends
\def\ds{\fam\dsfam\fourteends}
\else
{}
\fi
\ifx\ifLMTS\relax%
\def\rm{\fam0 \fourteenrm \let\tcfont=\fourteencrm}
\def\it{\fam\itfam\fourteenit \let\tcfont=\fourteencit}
\def\sl{\fam\slfam\fourteensl \let\tcfont=\fourteencsl}
\def\tt{\fam\ttfam\fourteentt \let\tcfont=\fourteenctt}
\def\bf{\fam\bffam\fourteenbf \let\tcfont=\fourteencbf}
\def\sf{\fam\ssfam\fourteenss \let\tcfont=\fourteencss}
\else
\def\rm{\fam0 \fourteenrm}
\def\it{\fam\itfam\fourteenit}
\def\sl{\fam\slfam\fourteensl}
\def\tt{\fam\ttfam\fourteentt}
\def\bf{\fam\bffam\fourteenbf}
\def\sf{\fam\ssfam\fourteenss}
\fi
\def\sc{\fam\cscfam\fourteencsc}
\def\mb{\fam\mibfam} \def\boldstyle{\fam\mibfam\fourteenmib}
\def\bcal{\fam\bsyfam}
\def\unsetmathbold{
\textfont1=\fourteeni \scriptfont1=\ninei \scriptscriptfont1=\seveni
\textfont2=\fourteensy \scriptfont2=\ninesy \scriptscriptfont2=\sevensy
\def\oldstyle{\fam1 \fourteeni} }
\def\setmathbold{
\textfont1=\fourteenmib \scriptfont1=\ninemib \scriptscriptfont1=\sevenmib
\textfont2=\fourteenbsy \scriptfont2=\ninebsy \scriptscriptfont2=\sevenbsy
\def\oldstyle{\fam1 \fourteenmib} }
\normalbaselineskip=17pt minus 1pt
\setbox\strutbox=\hbox{\vrule height12pt depth5pt width0pt}
\bigsize=11.9pt
\Bigsize=16.1pt
\biggsize=20.3pt
\Biggsize=24.5pt
\normalbaselines\rm}
%%
%% Change to twentypoint fonts %%
\def\twentypoint{
\textfont0=\twentyrm \scriptfont0=\fourteenrm \scriptscriptfont0=\tenrm
\textfont1=\twentyi \scriptfont1=\fourteeni \scriptscriptfont1=\teni
\def\mit{\fam1 } \def\oldstyle{\fam1 \twentyi}
\textfont2=\twentysy \scriptfont2=\fourteensy \scriptscriptfont2=\tensy
\def\cal{\fam2 }
\textfont3=\twentyex \scriptfont3=\fourteenex \scriptscriptfont3=\tenex
\textfont\itfam=\twentyit
\textfont\slfam=\twentysl
\textfont\ttfam=\twentytt
\textfont\ssfam=\twentyss
\scriptfont\ssfam=\fourteenss \scriptscriptfont\ssfam=\tenss
\textfont\bffam=\twentybf
\scriptfont\bffam=\fourteenbf \scriptscriptfont\bffam=\tenbf
\textfont\cscfam=\twentycsc
\textfont\mibfam=\twentymib
\scriptfont\mibfam=\fourteenmib \scriptscriptfont\mibfam=\tenmib
\textfont\bsyfam=\twentybsy
\scriptfont\bsyfam=\fourteenbsy \scriptscriptfont\bsyfam=\tenbsy
\ifx\ifDS\relax%
\textfont\dsfam=\twentyds
\scriptfont\dsfam=\fourteends \scriptscriptfont\dsfam=\tends
\def\ds{\fam\dsfam\twentyds}
\else
{}
\fi
\ifx\ifLMTS\relax%
\def\rm{\fam0 \twentyrm \let\tcfont=\twentycrm}
\def\it{\fam\itfam\twentyit \let\tcfont=\twentycit}
\def\sl{\fam\slfam\twentysl \let\tcfont=\twentycsl}
\def\tt{\fam\ttfam\twentytt \let\tcfont=\twentyctt}
\def\sf{\fam\ssfam\twentyss \let\tcfont=\twentycss}
\def\bf{\fam\bffam\twentybf \let\tcfont=\twentycbf}
\else
\def\rm{\fam0 \twentyrm}
\def\it{\fam\itfam\twentyit}
\def\sl{\fam\slfam\twentysl}
\def\tt{\fam\ttfam\twentytt}
\def\sf{\fam\ssfam\twentyss}
\def\bf{\fam\bffam\twentybf}
\fi
\def\sc{\fam\cscfam\twentycsc}
\def\mb{\fam\mibfam} \def\boldstyle{\fam\mibfam\twentymib}
\def\unsetmathbold{
\textfont1=\twentyi \scriptfont1=\fourteeni \scriptscriptfont1=\teni
\textfont2=\twentysy \scriptfont2=\fourteensy \scriptscriptfont2=\tensy
\def\oldstyle{\fam1 \twentyi} }
\def\setmathbold{
\textfont1=\twentymib \scriptfont1=\fourteenmib \scriptscriptfont1=\tenmib
\textfont2=\twentybsy \scriptfont2=\fourteenbsy \scriptscriptfont2=\tenbsy
\def\oldstyle{\fam1 \twentymib} }
\normalbaselineskip=24pt minus 2pt
\setbox\strutbox=\hbox{\vrule height17pt depth7pt width0pt}
\bigsize=17pt
\Bigsize=23pt
\biggsize=29pt
\Biggsize=35pt
\normalbaselines\rm}
%%
%% Change to twentyfour points fonts %%
\def\twentyfourpoint{
\textfont0=\twentyfourrm \scriptfont0=\fourteenrm \scriptscriptfont0=\twelverm
\textfont1=\twentyfouri \scriptfont1=\fourteeni \scriptscriptfont1=\twelvei
\def\mit{\fam1 } \def\oldstyle{\fam1 \twentyfouri}
\textfont2=\twentyfoursy \scriptfont2=\fourteensy \scriptscriptfont2=\twelvesy
\def\cal{\fam2 }
\textfont3=\twentyfourex \scriptfont3=\fourteenex \scriptscriptfont3=\twelveex
\textfont\itfam=\twentyfourit
\textfont\slfam=\twentyfoursl
\textfont\ttfam=\twentyfourtt
\textfont\ssfam=\twentyfourss
\scriptfont\ssfam=\fourteenss \scriptscriptfont\ssfam=\twelvess
\textfont\bffam=\twentyfourbf
\scriptfont\bffam=\fourteenbf \scriptscriptfont\bffam=\twelvebf
\textfont\cscfam=\twentyfourcsc
\textfont\mibfam=\twentyfourmib
\scriptfont\mibfam=\fourteenmib \scriptscriptfont\mibfam=\twelvemib
\textfont\bsyfam=\twentyfourbsy
\scriptfont\bsyfam=\fourteenbsy \scriptscriptfont\bsyfam=\twelvebsy
\ifx\ifDS\relax%
\textfont\dsfam=\twentyfourds
\scriptfont\dsfam=\fourteends \scriptscriptfont\dsfam=\twelveds
\def\ds{\fam\dsfam\twentyfourds}
\else
{}
\fi
\ifx\ifLMTS\relax%
\def\rm{\fam0 \twentyfourrm \let\tcfont=\twentyfourcrm}
\def\it{\fam\itfam\twentyfourit \let\tcfont=\twentyfourcit}
\def\sl{\fam\itfam\twentyfoursl \let\tcfont=\twentyfourcsl}
\def\tt{\fam\ttfam\twentyfourtt \let\tcfont=\twentyfourctt}
\def\sf{\fam\ssfam\twentyfourss \let\tcfont=\twentyfourcss}
\def\bf{\fam\bffam\twentyfourbf \let\tcfont=\twentyfourcbf}
\else
\def\rm{\fam0 \twentyfourrm}
\def\it{\fam\itfam\twentyfourit}
\def\sl{\fam\itfam\twentyfoursl}
\def\tt{\fam\ttfam\twentyfourtt}
\def\sf{\fam\ssfam\twentyfourss}
\def\bf{\fam\bffam\twentyfourbf}
\fi
\def\sc{\fam\cscfam\twentyfourcsc}
\def\mb{\fam\mibfam} \def\boldstyle{\fam\mibfam\twentyfourmib}
\def\bcal{\fam\bsyfam}
\def\unsetmathbold{
\textfont1=\twentyfouri \scriptfont1=\fourteeni \scriptscriptfont1=\twelvei
\textfont2=\twentyfoursy \scriptfont2=\fourteensy \scriptscriptfont2=\twelvesy
\def\oldstyle{\fam1 \twentyfouri} }
\def\setmathbold{
\textfont1=\twentyfourmib \scriptfont1=\fourteenmib
\scriptscriptfont1=\twelvemib
\textfont2=\twentyfourbsy \scriptfont2=\fourteenbsy
\scriptscriptfont2=\twelvebsy
\def\oldstyle{\fam1 \twentyfourmib} }
\normalbaselineskip=28pt minus 2pt
\setbox\strutbox=\hbox{\vrule height20pt depth8pt width0pt}
\bigsize=20.4pt
\Bigsize=27.6pt
\biggsize=34.8pt
\Biggsize=42pt
\normalbaselines\rm}
%%
%% double space
\def\doublespace{
\normalbaselineskip=24pt minus 1pt
\setbox\strutbox=\hbox{\vrule height17pt depth 7pt width0pt}
\normalbaselines}
%%
%% Initial set at ten points
\tenpoint

