\cleardoublepage
\chap LITERATURE REVIEW

%In this work, a numerical solution for the equation of convection--dispersion is proposed, assuming a soil concentration dependent solute uptake as the boundary condition at root surface given by the Michaelis-Menten equation. The proposed model is compared with a no solute uptake \citeonline[liersolute] and a constant solute uptake \citeonline[willigen1] numerical models, as well as with an analytical model which uses steady-state condition for water content \citeonline[cushman]. The numerical models consider a transient water flow, based on the work of \citeonline[lierwater]. There are several numerical (\cite[nye], \cite[simunek]) and analytical (\cite[roose], \cite[cushman], \cite[willigen1]) models that describe water and solute uptake by the roots, each one with their own particularities, such as steady-state water flux solutions and different boundaries conditions at root surface. \cite[feddes] and \cite[raats] give reviews of soil water uptake modeling including effects of salinity.

%EXPLAIN THE ABILITY OF THE MODEL TO CALCULATE ACTIVE AND PASSIVE UPTAKES
%EXPLAIN THE REASON TO USE MM EQUATION

%Nyer and Marriot (1969): qs0=D dc/dr + qC ~ kC0/(1+kC0/Im) but they considered Im and K independent of q0. I just considered Im independent of q0, but k, alpha in my case, is depedent on concentration
%Other authors also makes a separation of active and passive as the diffusive and convective part of eq. (1), such as Silberbush, Ben-Asher and Ephrath (2005) [7] 102.pdf on calcium uptake in a soilless culture, Schrder et al. (2012) [6] 48.pdf, Ungs, Boersma and Akratanakul (1982) [9] 11.pdf, Raij et al. (2013) 43.pdf [4] and Simunek and Hopmans (2009) [8] 68.pdf just to cite some.

%The determination of water and salt stress in field crops is a major problem in agriculture. Stress of water is caused by a lower amount of water in the soil, which reduces the potential transpiration of the plant and leads to a yield decrease. Similarly, higher solute concentrations in the soil solution can lead to stress caused by salinity which, in turn, reduce the potential transpiration (osmotic effect) and interfere negatively in the plant metabolism (ionic stress or toxicity).
%Works of Maas anf Hoffman (1977) XXX determines threshold values in which the plant cannot recover from saline stress by a reducion of yield. Some other works (Katerji et al, 2000) test their values and try to make a better approximation of its empirical values by taking into account other characteristics like water stress as well 
%Modeling of water and solute extraction by plant roots can be used to predict the transpiration rates together with soil water and solute movement
%One of the greatest problems in agriculture is to predict the actual transpiration rates in order to calculate the amount of water to be applied in the field crops. The determination of this variable depends on a great number of properties such as climatic data and parameters of soil and plant.

%It needs to be clear that active and passive uptake only makes sense when it means movement of solute at the root surface. Then, active is the diffusive part and passive is the convective part. The whole transport in the soil profile is by convection and diffusion only.

%Analytical solutions are possible only for a few very simple systems, tend to be complicated and often are not understood by who needs to use them. Numerical solutions have the advantage that hey are not restricted to simple boundary conditions and a numerical technique converts a differential equation into a set of algebraic expressions that are easier to solve \cite[campbell].

\sec Water movement in the soil
%Describe the flow equation, its use in the continuity equation to form the Richards equation. Explain in details how each paameter affects the flux, how/with what they vary and how they are measured

%The flow equations that deal with the movement of water in the soil can be applied for saturated and non-saturated conditions. The structure of the soil is very complex, its pores forms tortuous paths and dead-ends for the water flow, turning impractical a microscopic study of its flow 
(A SHORT INTRODUCTION HERE)

Henry Darcy, in the middle of the 19th century, realized a series of experiments to study the flow of water through a saturated soil within a vertical column, which resulted in the equation
$$
Q=K_s A {\Delta H \over L} \eqmark
$$
%
known as Darcy's law. $Q$ [L$^3$T$^{-1}$] is the water flow, $A$ [L$^2$] is the cross-section area of the column, $\Delta H$ [L] is the difference between the hydraulic potentials at the beginning and the end of the column with length $L$ [L] and $K_s$ [LT$^{-1}$] is a proportionality factor called saturated hydraulic conductivity \cite[libardi2010].
The Darcy's law is valid for a homogeneous and isotropic soil with a constant cross-section area of the column through its length.
Although it is an empirical equation, it can be derived from the relationship between Bernoulli and Poiseuille's laws. For this derivation, see \citeonline[warrick]. 

The water flux density $q$ [LT$^{-1}$] (also called Darcian velocity) represents the volumetric water flow per unit area and can be generalized to represent the flow in the three spacial dimensions by its differential form \cite[hillel,warrick,libardi] (commonly used in macroscopic models). It also can represents the flow in radial coordinates (most used in microscopic models), as follows:
\label[water_flux]
$$
%q = -K_s \nabla H = -K_s \left({\partial H \over \partial x}+{\partial H \over \partial y}+{\partial H \over \partial z}\right) \eqmark
q = -K_s {dH \over dr} \eqmark
$$

The negative sign arises to determine the direction of the flux, from high to low hydraulic potentials. 

%Limitations of darcy's law
The Darcy's law is valid only for laminar flow. The flux (or velocity) $q$ eventually becomes nonlinear as the hydraulic gradient is too high or too low (Figure \ref[fig_Re]). These deviations can be either attributed to turbulent flow, for high gradients, or to the non-Newtonian behavior of water, for low gradients \cite[hillel,warrick]. Notwithstanding the limitations of Darcy's law, it can be applied in the vast majority of cases referring to water flow in soils.

\medskip
\label[fig_Re]
\picw=10cm \cinspic Re1.pdf
\caption/f {Darcy's law deviations at high and low fluxes. The continuous line is the actual flux and the dashed lines the Darcy's flux}
\medskip
{\tenpoint Source: Adapted from \citeonline[hillel]}

%Darcy's law for nonsaturated soils

%K-O-h Relations

%continuity equation
$$
{\partial\theta \over \partial t} = -{\partial q \over \partial r}
$$

%Richards equation
$$
{\partial\theta \over \partial t} = {\partial \over \partial r} \left(K(\theta){\partial h \over \partial r}\right)
$$
%Talk about micro and macroscopic models, compensation, reduction functions (FEDDES,Quirijn)...

\sec Solute availability and transport in soil
%Describe the flow equation, its use in the continuity equation to form the Convection-Difusion equation. Explain in details how each paameter affects the flux, how/with what they vary and how they are measured

% Talk about diffusion and massflow in the solute differential equation, buffer power...
% First, describe the availability and then, the transport

Solutes are present in the three phases of the soil. They are adsorbed on the soil colloids (solid), diluted in soil water (liquid) and vaporized in the soil air (gas). 
Solute transportation between the three phases is constantly occurring in the search for an equilibrium in the soil-plant-atmosphere dynamic system. Although the exchange between the gas phase to the others is important, most of the transport is done by the interaction between solid and liquid phases.
%In general, soils are predominantly negatively charged but often also contains colloids with positively charged sites, resulting from a large range of cations and anions species that coexist in a natural soil solution. 
As the concentration of ions in soil solution is diminished, due to root extraction or leaching, they are rapidly replenished by the ones that are adsorbed in soil colloids \cite[tinker]. 
Concentration of ions in soil solution is more affected by soil water content than the concentration of adsorbed ions. In a soil with calcium as the predominant cation in the soil solution, for example, a reduction by half of water content practically double its concentration in soil solution. The adsorbed concentration of calcium is slightly changed and it is usually neglected.
Roots are only able to take up solutes that are within the soil solution although some authors believe that growing roots that eventually intercept adsorbed solutes on soil colloids can absorb them without water as an intermediate \cite[marschner].
The relation between adsorbed and solution ion concentration is crucial in determining the ability of the soil to maintain the solution concentration and its mobility.

Solutes in the soil solution must be transported to the root surface to be taken up by the plants. The physic mechanisms involved in this process are diffusion and convection.
Similarly as with water, the diffusive solute flow ($F_d$) within the soil is driven by a gradient ($dC/dr$) between the root and the surrounding soil, being proportional to the solute diffusivity ($D_m$), as follows:
$$
F_d=-D_m {dC \over dr}
$$
%talk about diffusion coefficient

It is possible to analytically determine the diffusion of a solute in plain water at a given temperature. It depends on the hydrated radius of the ion.

The coefficient $D_m$ for the diffusive transport is most known as molecular diffusion. 
The diffusion process results from the random thermal motion of ions or molecules and depends on the temperature and the components of the molecule. 
As the process of diffusion is occurring in soil, some empirical models (BRESLER, 1973; PAPENDICK; CAMPBEL, 1973) can estimate it by relating to the one of water, as proposed by Millington and Kirk (1961):
$$
D_m=\xi D^0 = {\theta^{10 \over 3} \over \theta^2_s} D^0
$$

The diffusion coefficient is also affected by the convective flow, thus, to the molecular diffusion, is added a mechanical dispersion coefficient ($D_s$) which varies with the water flux and soil micropore size. An example of the estimation of the dispersion coefficient is given by (Bear, 1972):
$$
D_s=\tau {q \over \theta}
$$

Thus, for the overall diffusion processes, a called effective diffusion coefficient is defined as being the sum of both molecular diffision and molecular dispersion in the general equation of solute flux
$$
D=D_m+D_s
$$

The mechanical dispersion becomes more important to diffusion as high is the water flux

\medskip
\label[fig_D]
\picw=10cm \cinspic Ds.pdf
\caption/f {Molecular diffusion and mechanical dispersion coefficients as a function of water flux}
\medskip

The soil water content, the interaction of the solute with soil colloids and the distance it must overcome to arrive at the root surface are the main factors governing the diffusion mechanism \cite[wild].
The additional convective component is driven by mass flow of water proportional to the solute concentration, and together with the plant transpiration rate, determines the quantity of ions transported through this mechanism \cite[barber74].
The predominant mechanism for solute transport depends mainly on the interaction between water and solute uptake rates. 
At high transpiration rates, mass flow is the primary mechanism of transport. In a case of plant demand for solute is less than the solute flux towards the roots, solute would accumulate at root surface leading to a diffusion of solutes away from the roots (in the opposite direction to the mass flow) until an equilibrium is achieved. 
In the opposite case, the convective flow towards the roots would be insufficient to meet plant demand for solute and diffusion would become the complementary mechanism (or even more important than convection) of solute transport. 
It is clear to perceive that diffusion and convection occur simultaneously and, therefore, the solution for the governing equation of solute transport has to deal with both components as two synergic processes. 

%The determination of each contribution is often calculated roughly by measuring an individual contribution and comparing it with the total solute uptake. 

%The diffusion coefficient determines the velocity of solute movement at a given gradient and it is composed of molecular diffusion and mechanical dispersion coefficients.

%The former can be estimated from diffusion coefficient in water and depends on the water content. The latter is related to water flow and a factor of soil tortuosity. At high water fluxes, the mechanical dispersion is more important to the dispersion than to the molecular diffusion (Nye and Tinker, 1968). SHOW FIGURE
%The convective transport occurs with dissolved solute carried by mass flow of water towards the root. It is highly dependent on transpiration rate, which is the drive force of the hydraulic gradient.
%The sum of both convective and diffusive transports determines the solute flow in soil.

%But there is an additional component that affects the transport of solute: the convection which is the solute transport by mass flow of water, driven by the transpiration rate. In that way, some cases could occur with different scenarios, for example, if there transpiration rate is zero (at night or in very dry conditions), the solute flow is due to diffusion only, from in the direction of the gradient. In the long term, the concentration in the soil profile will be equal. 
%In the case of the water flow is high, leading a convective transport higher than the plant demand for the solute, an accumulation at root surface will occur and, finally, in a situation of the plant demand is attended by convection and diffusion, a depletion of solute will occur at root surface.

%The soil has a buffer power capacity in which it delivers solute to the soil as ii is depleted. The rate of reposition depends on the soil water content and on a adsorption constant, which is different for each soil type. The solute that are available to the plant is the solute in the soil water.Depending of the buffer capacity, the soil can maintain the concentration constant even with the absorption by the roots.

\sec Nutrient uptake by plant roots
%Explain MM equation and its use as a boundary condition. Describe how the parameters affects the flux, how they vary and how they are measured.

%Plants need to uptake water and nutrients from soil to
Once solutes are transported to the root surface, they are readily available to be taken up by the plants.
The rate of solute uptake by plant roots can be referred to be similar to that between enzyme and its substrate, described by the Michaelis-Menten (MM) equation \cite[michaelis]. The most well-known form of the equation is the following:
\label[eq_MM]
$$
I = {I_m C \over K_m+C} \eqmark
$$
%
where $I$ is the rate of solute uptake, $I_m$ is the maximum uptake rate, $C$ is the solute concentration in the external medium and $K_m$ the Michaelis-Menten constant. 
$I_m$ is found experimentally and $K_m$ is adjusted as the concentration at which $I_m$ assumes half of its value, and represents the affinity of the plant for the solute. 
The uptake rate following the MM kinetics is an asymptote which saturates with increasing external concentration (Figure \ref[orig_MM]). The equation shows that, for very low concentration values ($C \ll K_m$), the uptake rate is proportional to concentration whereas, for $C \gg K_m$, the solute uptake rate is maximal and independent of concentration.
\citeonline[johnson] revisited the original MM paper and, using modern computer techniques (inexistent by the time the equation was developed), they could get the same results as of the original paper---with no simplifying assumptions---showing how precise and careful was the measurements and equation development. They stressed that the constant found by Michaelis and Menten is $I_m/K_m$ rather than the widely referred $K_m$.

\medskip
\label[orig_MM]
\picw=13cm \cinspic orig_MM.pdf
\caption/f {Solute uptake rate as a function of external concentration following Michaelis-Menten kinetics with its more common parameters}
\medskip

\citeonline[epstein52] were the first to use the MM equation to represent the uptake of a solute and its concentration in external medium (for potassium and sodium solutions in barley roots) and it has been frequently used since then. 
It describes well the solute uptake for both anions \cite[epstein72,siddiqi,wang] and cations \cite[kochian,kelly,sadana,broadley,lux] in the low concentration range and, adding a linear component ($k$) to the equation, it can properly estimates the uptake rate also for high concentrations \cite[epstein72,kochian,borstlap,wang,vallejo,broadley].
Many authors agree that for low concentration in external medium, the uptake is driven by an active mechanism of the plant, as it occurs against the solute gradient between root and soil, known as Epstein's mechanism~I. 
For the high concentration range, solutes are freely transported from soil to roots by diffusion and occasional convection. This passive transport is known as Epstein's mechanism~II \cite[kochian,siddiqi]. 
Moreover, experiments have shown that a minimum concentration value ($C_{min}$) in which the uptake ceases to occur is often found \cite[mouat,dudal,machado].
Details on Epestein's mechanisms and its physiological processes, and on active and passive uptake, are found on \citeonline[epstein60] and \citeonline[fried].

% Limitiations of MM parameters measurements (21.pdf, pg. 163 - 2nd column)
The values of MM parameters are strongly dependent on the experimental methods used and vary with plant species, plant age, plant nutritional status, soil temperature and pH \cite[barber,shi]. Therefore, they have to be determined for each particular experimental situation.
Some types of experiments to determine the kinetic parameters $I_m$, $K_m$ and $C_{min}$ include hydroponically-grow plants \cite[barber] and the use of radioisotopes to estimate them directly from soil \cite[nye77]. The latter is more realistic since there is a large difference between a stirred nutrient solution and the complex and dynamic soil medium.
Measuring $C_{min}$ is particularly difficult \cite[seeling,lambers] because it occurs at very low concentration levels, being hard to be detected. \citeonline[seeling] shows that $C_{min}$ can be neglected for the cases of high $K_m$ values.

%Nevertheless, experiments have shown that the uptake rate can present a linear increase at high concentrations and a linear component must be added on MM equation \cite[epstein72,kochian,borstlap,wang,vallejo,broadley]. The solute can go inside the plant freely by diffusion and convection, so it is a passive mechanism called Epstein's mechanism II \cite[kochian,siddiqi].

Therefore, the MM equation can be often found in a modified form which considers both the linear component and the concentration where uptake is zero, as follows:
%by the addition of the other two terms of minimum concentration and the linear component, and used in many recent models is:
\label[eq_MM_plus]
$$
I = {I_m (C-C_{min}) \over K_m+C-C_{min}} + k (C-C_{min}) \eqmark
$$

The addition of the new parameters changes the solute uptake rate as shown in Figure \ref[MM_plus]. By the fact that the solute uptake rate acts in distinct ways for different solute concentration ranges, the modified MM equation may be considered a stepwise function. 

Both the original and the modified MM equation have been used in numerical \cite[nye,silberbush,simunek,bechtold] and analytical \cite[barber,roose] solute uptake models.
Authors do not agree on which cases one equation has to be chosen over the other, it all depends on the assumptions made for each particular model.  
During the history of the development of nutrient uptake models---using MM equation---some authors have used the modified version to estimate both passive and active uptake while others have chosen the original one to deal with active uptake only \cite[shi].

\medskip
\label[MM_plus]
\picw=13cm \cinspic MM_plus.pdf
\caption/f {Solute uptake rate as a function of external concentration. The saturable component is the first term of the right-hand side of Eq. \ref[eq_MM_plus], the linear component is the second term and the total uptake is Eq. \ref[eq_MM_plus] itself}
\medskip

There are other alternatives to describe the solute uptake rate function apart from MM equation. \citeonline[dalton] proposed a physical-mathematical model that include active uptake, mass flow and diffusion, and express solute uptake as being proportional to root water uptake. \citeonline[bouldin] described the uptake from low and high concentrations solutions by two linear equations, simplifying the process of solute uptake at root surface to two diffusion components.
\citeonline[nye77] determined the uptake rate as a linear component followed by a constant rate phase where the threshold value is dependent on solute concentration inside the plant.
Nevertheless, MM equation is yet the mostly used equation in root solute uptake models for being physically grounded and for its good agreement with experimental data.


%The separation active and passive is due to many physiological proccesses within the plant and is explained in more details in \cite[marschner].

\sec Transpiration reduction functions for water and solute
%Describe the reduction functions for water and solute

\sec Water and solute uptake models
%Describe the 3 main models (NU,CU and Cushman) to be used in the comparisson and show their weaknesses

