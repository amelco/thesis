\app Algorithm used to solve Equation \ref[eq_disc_water]

Applying Equation \ref[eq_disc_water] to each segment $i$, the values of $H_i^{j+1}$ is found by solving the tridiagonal matrix as follows

$$
\left [ \matrix{
 \beta_1 & \gamma_1 & & & &  \cr
 \alpha_2 & \beta_2 & \gamma_2 & & &  \cr
  & \alpha_3 & \beta_3 & \gamma_3 & &  \cr
  &  & \ddots & \ddots & \ddots &  \cr
  &  & & \alpha_{n-1} & \beta_{n-1} & \gamma_{n-1}  \cr
  &  & & & \alpha_{n} & \beta_{n}  \cr}  \right]
\left [ \matrix{
 H^{j+1}_1 \cr
 H^{j+1}_2 \cr
 H^{j+1}_3 \cr
 \vdots \cr
 H^{j+1}_{n-1} \cr
 H^{j+1}_n \cr} \right ]
=
\left [ \matrix{
 g_1 \cr
 g_2 \cr
 g_3 \cr
 \vdots \cr
 g_{n-1} \cr
 g_n \cr} \right ]
 \eqmark
$$
%
in which $H_n=h_n+{h_\pi}_n$ and ${h_\pi}_n$ is the osmotic head of the last time step; $\alpha_i$, $\beta_i$, $\gamma_i$ and $g_i$ are defined as described in the following.

1. Intermediate segments ($i=2$ to $i=n-1$):
$$
\eqalignno{
\alpha_i &= -{t^{j+1}-t^j \over r_i(r_i-r_{i-1})\Delta r_i} r_{i-1/2}K_{i-1/2}^j & \eqmark \cr
\beta_i &= {C_w}_i^{j+1,p-1}+{t^{j+1}-t^j \over r_i(r_i-r_{i-1})\Delta r_i} r_{i-1/2}K_{i-1/2}^j
   + {t^{j+1}-t^j \over r_i(r_{i+1}-r_{i})\Delta r_i} r_{i+1/2}K_{i+1/2}^j  & \eqmark \cr
\gamma_i &= -{t^{j+1}-t^j \over r_i(r_{i+1}-r_{i})\Delta r_i} r_{i+1/2}K_{i+1/2}^j & \eqmark \cr
g_i &= {C_w}_i^{j+1,p-1}H_i^{j+1,p-1} - \theta_i^{j+1,p-1}+\theta_i^j & \eqmark \cr
}
$$

2A. Root segment ($i=1$) with flux boundary condition, applied while $H_1 > $ \hlim, i.e., when transpiration meets potential transpiration demand:
$$
\eqalignno{
\beta_1 &= {C_w}_1^{j+1,p-1} + {t^{j+1}-t^j \over r_1(r_{2}-r_{1})\Delta r_1} r_{1+1/2}K_{1+1/2}^j  & \eqmark \cr
\gamma_1 &= -{t^{j+1}-t^j \over r_1(r_{2}-r_{1})\Delta r_1} r_{1+1/2}K_{1+1/2}^j & \eqmark \cr
g_1 &= {C_w}_1^{j+1,p-1}H_1^{j+1,p-1} - \theta_1^{j+1,p-1}+\theta_1^j - {r_0 q_0(t^{j+1}-t^j) \over r_1 \Delta r_1}& \eqmark \cr
}
$$
%
where $q_0 = \displaystyle{T_p \over 2 \pi r_o R z}$.

2B. Root segment ($i=1$) with head boundary condition, applied while $H_1 = $~\hlim, i.e., when transpiration is less than potential transpiration demand:
$$
\eqalignno{
\beta_1 &= {C_w}_1^{j+1,p-1} + {t^{j+1}-t^j \over r_1(r_{1}-r_{0})\Delta r_1} r_{1/2}K_{1/2}^j
  + {t^{j+1}-t^j \over r_1(r_{2}-r_{1})\Delta r_1} r_{1+1/2}K_{1+1/2}^j  & \eqmark \cr
\gamma_1 &= -{t^{j+1}-t^j \over r_1(r_{2}-r_{1})\Delta r_1} r_{1+1/2}K_{1+1/2}^j & \eqmark \cr
g_1 &= {C_w}_1^{j+1,p-1}H_1^{j+1,p-1} - \theta_1^{j+1,p-1}+\theta_1^j + {t^{j+1}-t^j \over r_1(r_{1}-r_{0})\Delta r_1} r_{1/2}K_{1/2}^j H_{lim} & \eqmark \cr
}
$$

3. Outer segment ($i=n$):

$$
\eqalignno{
\alpha_n &= -{t^{j+1}-t^j \over r_n(r_n-r_{n-1})\Delta r_n} r_{n-1/2}K_{n-1/2}^j & \eqmark \cr
\beta_n &= {C_w}_n^{j+1,p-1}+{t^{j+1}-t^j \over r_n(r_n-r_{n-1})\Delta r_n} r_{n-1/2}K_{n-1/2}^j & \eqmark \cr
g_n &= {C_w}_n^{j+1,p-1}H_n^{j+1,p-1} - \theta_n^{j+1,p-1}+\theta_n^j & \eqmark \cr
}
$$

The value of pressure head $h$ is calculated and added to $H$ using the value of osmotic head ($h_\pi$) from the previous time step. The values of $h_\pi$ are calculated and then the values of $H$ are updated in the next time step. The process repeats until the values of updated $H$ converge to $H$ from the previous time step. 
