\app Derivation of microscopic model root related equations

The derivation of Equation \ref[eq_R], here repeated
$$
R = {1 \over \pi {r_m}^2} \hbox{ ,}
$$
%
can be done by verifying that the root length density ($R$) is the sum of all root lengths ($z$) per volume of soil ($V_{soil}$). 
If we consider that the arrangement showed in Figure \ref[fig_rootzone] has $n$ roots, the root length ($L$) is simply $z_1+z_2+...+z_n$. 
Similarly, the soil surface area occupied by the plant ($A_p$) is the sum of the soil surface areas of the circles with radius $r_m$ ($A_s$), or $A_p = A_{s_1}+A_{s_2}+...+A_{s_n}$. 
Therefore, mathematically:
\label[eq_devR]
$$
R = {L \over V_{soil}} = {L \over A_p z} = {z_1+z_2+...+z_n \over (A_{s_1}+A_{s_2}+...+A_{s_n})z}. \eqmark
$$

All the cylinders have the same depth and the same radius, thus $z_1=z_2=...=z_n=z$ and $A_{s_1}=A_{s_2}=...=A_{s_n} = A_s$. 
Knowing that $A_s=\pi {r_m}^2$, Equation \ref[eq_devR] can be rewritten as
\label[eq_devR2]
$$
R = {n z \over n A_s z} = {1 \over A_s} \Rightarrow R = {1 \over \pi {r_m}^2} \qed \eqmark
$$

To derive Equation \ref[eq_rm2]
$$
r_m = \sqrt{A_p z \over \pi L} \hbox{ ,}
$$
we start by solving Equation \ref[eq_devR2] for $r_m$ 
\label[eq_devrm]
$$
{r_m}^2 = {1 \over \pi R}\, , \eqmark
$$
%
so that $r_m$ is a function of $R$.
Analysing Equation \ref[eq_devrm], and with the help of Figure \ref[fig_rootzone], it is easy to notice that an increase in root length density implies that there are more roots in the same soil volume and, therefore, the space between the roots is diminished. 
Hence, we can replace $R$ in Equation \ref[eq_devrm] by its relation between $L$, $z$ and $A_p$ from Equation \ref[eq_devR] to have:
\label[eq_devrm2]
$$
{r_m}^2 
= {1 \over \pi {L \over A_p z}} = {A_p z \over \pi L} \Rightarrow r_m = \sqrt{A_p z \over \pi L} \qed \eqmark
$$

With Equations \ref[eq_devrm] and \ref[eq_devrm2] we have basic relationships of the parameters that can be arranged to find expressions for any of them, as long we have enough measured parameters.
Thus, for a known value of $R$, $r_m$ can be calculated
%(Equations \ref[eq_rm] and \ref[eq_L]) 
using Equation \ref[eq_devrm], and $L$ by solving Equation \ref[eq_devrm2] for $L$. The former yields
$$
r_m={1 \over \sqrt{\pi R}} \,,
$$
which is the Equation \ref[eq_rm], and the later yields
$$
L = {A_p z \over \pi {r_m}^2} \,,
$$
which is the Equation \ref[eq_L].

As mentioned in Section \ref[sec_micro], when there is no root length density data available (or other parameter that can lead to it),
it is necessary to measure some root and soil characteristics of an experimental set to find parameters such as $L$, $r_m$ or $R$.
By finding one of them, it is possible to have the others through the relations presented in the previous equations of this appendix.
%$L$ can be estimated by measuring some root and soil characteristics. 
%By finding $L$, it is possible to have the other desired parameters $r_m$ and $R$.
%Equation \ref[eq_L2] is one of the possible ways to calculate $L$. 
%Other methods can be used to calculate $L$ or other parameter such as $R$.
Here, it is demonstrated the development of Equation \ref[eq_L2]:
$$
L = {d_s A_p z - m_s \over d_s \pi \overline{r_0}^2}
$$
%
where $d_s$ is the soil density, $m_s$ is the soil mass and $\overline{r_0}$ is the average root radius. 
All parameters are relatively easy to be measured. $\overline{r_0}$ can be troublesome but there are methodologies to measure it properly. $d_s$ and $m_s$ are routine in soil physics laboratories and $A_p$ is known. 
The total volume of a one-plant experimental parcel is the sum of soil and root volumes. 
Since it is difficult to measure the root volume, it can be estimated by difference between total and soil volumes. 
The first can be found by multiplying plant area by root depth and the later by the relation between soil volume, mass and density. 
Hence, volume of roots can be calculated as follows:
$$
V_t = V_r+V_s \Rightarrow V_r = V_t-V_s = A_p z - {m_s \over d_s} \eqmark
$$

Also, as roots are considered cylinders, their volume can be calculated as the volume of a cylinder:

$$
V_r = \pi \overline{r_0}^2 L \eqmark
$$

The unknown is $L$, all other parameters are measured. Relating the two $V_r$ equations, we can, thus, solve for $L$:
$$
A_p z - {m_s \over d_s} = \pi \overline{r_0}^2 L \Rightarrow L=\left(A_pz-{m_s \over d_s}\right) {1 \over \pi \overline{r_0}^2} \Rightarrow L = {d_s A_p z - m_s \over d_s \pi \overline{r_0}^2}
\qed \eqmark
$$



%The total root surface area ($A_r$, m$^2$) and volume ($V_r$, m$^2$) are given by the basic cylinder equations:
%$$
%\eqalignno{
%A_{r} &= 2 \pi r_0 L & \eqmark \cr
%V_r &= \pi {r_0}^2 L & \eqmark
%}
%$$
%
%For their easier measurement, the root related input data for the models are $r_0$, $z$ and the root density $R$ (m m$^{-3}$). 
%The other parameters can be estimated by finding proper relationships between the system variables. equations thus the equation for the determination of $r_m$ can be done knowing that the root density can be interpreted as 
%\label[eq_R]
%$$
%%R = { \hbox{sum of root lengths of all roots within the soil volume} \over \hbox{soil volume}}
%R = {L \over V_{soil}} = {L \over A_s L} = {1 \over A_s} = {1 \over \pi r_m^2} \eqmark
%$$
%%
%where $L$ \uhead is the total rooted soil depth (sum of the root lengths of all roots within the soil volume), $V_{soil}$ (m$^3$) is the soil volume and $A_s$ (m$^2$) is the area of the cylinder of radius $r_m$. The development of Equation \ref[eq_R] results in a relationship between $R$ and $r_m$, thus $r_m$ can be written as a function of $R$ as follows:
%\label[eq_rm]
%$$
%r_m = {1 \over \sqrt{\pi R}} \eqmark
%$$
%
%
