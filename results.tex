\cleardoublepage
\chap RESULTS AND DISCUSSION

The simulations were performed using the hydraulic parameters from the Dutch Staring series \cite[wosten] for three typical top soils, as listed in Table \ref[soils]. The general system parameters for the different scenarios are listed in Table \ref[general_param] and values for the Michaelis-Menten parameters in Table \ref[MMparam]. Values of root length density, salt content and relative transpiration were chosen to change, reflecting different possible scenarios that would occur in a practical situation. The chosen MM parameters were of K$^+$ solute.

\input tables/soils

\input tables/general_param

\input tables/MMparam

\sec Linear versus nonlinear comparison

This section describes how the linear (LU) and nonlinear (NLU) solutions simulate the transport of water and solutes in the system. The analysis of the results was made in order to choose one out of the two models in further simulations. The nonlinear solution uses the original MM equation but it takes longer to run due to an additional iterative process that has to be made. Another problem with NLU is that it is more susceptible to stabilization problems in the results. The linear model is a simplified version of the MM equation in which the solute uptake rate for small concentrations ($C<C_{lim}$) is smaller when compared to the original nonlinear equation. On the other hand, it has no stabilization problems and runs faster. Therefore, the objective of this section is to analyze the differences between the results of the two models and check if those differences are significant. For that, four different general scenarios were chosen (using the parameters listed in Table 2, with  loam soil) as listed below:

\begitems
%\style i
* Scenario 1: Medium root length density, High concentration and High potential transpiration (MrHcHt)
* Scenario 2: Medium root length density, High concentration and Low potential transpiration (MrHcLt)
* Scenario 3: Low root length density, High concentration and High potential transpiration (LrHcHt)
* Scenario 4: Medium root length density, Low concentration and High potential transpiration (MrLcHt)
\enditems


In all simulated scenarios, the difference between LU and NLU occurs only at values of solute concentration in soil water ($C$) below the threshold value \clim. This is expected because of the nature of the piecewise MM equation used in the model.
%{\bf (EXPLAIN THAT THE DIFFERENCE OCCURS ONLY FOR C<Clim IN METHODOLOGY)}
For both cases (LU and NLU), when solute concentration values are higher than \c2, all solute transport from soil to root is mostly driven by convection, therefore the uptake is passive only (active uptake is zero). With $C$ between the two threshold values (\c2 and \clim), the solute flux density is constant and NLU and LU are different only for values lower than $C_{lim}$ .

%Figures \ref[diff_t] and \ref[diff_r] show the differences between $C$ outputs of LU and NLU as a function of time and radial distance of axial center, respectively, for scenario 1.
The difference between C outputs of LU and NLU was calculated as
$$
\label[eq_diff]
d\!i\!f\!f\!.=C\!L\!U-C\!N\!L\!U \eqmark
$$
%
where $C\!L\!U$ and $C\!N\!L\!U$ are the concentrations for LU and for NLU, respectively.
When comparing differences in concentration between the two models, one has to be aware that $C\!N\!L\!U<C\!L\!U$ means that the uptake for NLU was greater than LU uptake since with a higher uptake, a higher amount of solute goes out from soil solution to inside the plant.
Therefore, according to Equation \ref[eq_diff], if $d\!i\!f\!f\!.<0$ then $C\!N\!L\!U>C\!L\!U$ and LU uptake is greater; if $d\!i\!f\!f\!.>0$ then $C\!N\!L\!U<C\!L\!U$ and NLU uptake is greater. Said that, we can see in Figure \ref[diff_t] that the uptake for NLU was greater then LU uptake at times when $C<C_{lim}$. This reflects a change also in the concentration profile for the latter times. Figure \ref[diff_r] shows the concentration profile at day 5 and the difference between CLU and CNLU through the profile. The high NLU uptake increased the concentration gradient causing a higher solute flux (most diffusive since water flux is very small) from soil towards the root, resulting in a slightly higher concentration for NLU at root surface ($d\!i\!f\!f\!.<0$).


\medskip
\label[diff_t]
\picw=13cm \cinspic diffs_t.pdf
\caption/f {Difference between the solute concentration in soil water at root surface ($C_0$) output for LU and NLU and $C_0$ as a function of time; and the relative difference}
\medskip

\medskip
\label[diff_r]
\picw=13cm \cinspic diffs_r.pdf
\caption/f {Difference between the solute concentration in soil water ($C$) output for LU and NLU and $C$ as a function of distance from axial center; and the relative difference}
\medskip

Nevertheless, the difference between LU and NLU is negligible. A difference of 80.9\% of concentration over time corresponds to only 0.318\% in the final concentration profile (for scenario 1) because the uptake at those times is really low. It can be seen at the cumulative uptake plot (Figure \ref[accumxt]) the insignificant effect of this difference.

The results of the relative accumulated error, according to Equation \ref[error_rel], were of 64.975\% and 0.739\%; 121.3\% and 0.941\%; 36.144\% and 0.027\% over time and distance for scenarios 2, 3 and 4, respectively.

\medskip
\label[accumxt]
\picw=13cm \cinspic accumxt.pdf
\caption/f {Cumulative solute uptake as a function of time for all scenarios. Dashed lines represents the nonlinear model}
\medskip

In addition, some part of the differences was due to stability problems with the numerical solution for NLU. The changing from equation 21 to equation 25 (change of boundary condition, from constant to nonlinear uptake rate) makes the numeric solution take some time to stabilize at the initial times. Many time and space steps combinations were used as an attempt to minimize the problem. Choosing a finer space discretization seems to decrease the stabilization problem but makes the simulation lasts longer. It needs to be found an optimal value for time and space step relation. Stabilization problems were not found in LU.

Since the differences between LU and NLU occurs for low concentration values and low solute flux, changes in relative transpiration are also negligible. The oscillation in the results due to the stabilization problem of the numerical solution is more likely to be noticed than the difference between the models. Figure \ref[diff_tr] shows the relative transpiration as a function of time and details the part where the oscillation occurs for each scenario. The absolute and relative differences are all due to oscillation problem.

\medskip
\label[diff_tr]
\picw=17cm \cinspic diffs_tr.pdf
\caption/f {Cumulative solute uptake as a function of time for all scenarios. Dashed lines represents the nonlinear model}
\medskip

\citeonline[roose2009] stated that, for numerical solutions of convection-dispersion equation, the convective part must use an explicit scheme because convection, unlike diffusion, occurs only in one direction thus the solution at the following time step depends only on the values within the domain of influence of the previous time step. This set bounds on time and space steps, with a condition of stability given by ${r_0 q_0 \Delta t \over D} < \Delta r$. As the proposed model uses a fully implicit scheme, that might be the cause of the stabilization problems.

%No significant difference between LU and NLU was verified in all output files (water and solute flux, concentration profile, heads and relative transpiration – Figures 1a, 2a, 3, 4, 5 and 6), but a more detailed analysis was made by analyzing the overall results (values of solute concentration for all output times -- Figure 9b) and low concentration results (values of solute concentration for output times where $C<C_{lim}$ -- Figures 8 and 9a).

%For low concentration (Figures 8 and 9a), it is noticeable that LU presents values of solute concentration higher than NLU. This is expected because in LE, solute uptake is always less than in NLU due to its linearization. Nevertheless, when the overall results are analyzed (Figure 9b), we can see that the difference is small and can be neglected. This conclusion (negligible difference) can be verified once more when analyzing the cumulative solute uptake (Figure 10). 

The conclusion, since no significant difference between LU and NLU was found, can be either to choose LU as it takes less time to run and has no stabilization problems, or to choose NLU except for the cases in which the stability problem is significantly high. Note that this is the conclusion for those specific scenarios as the results can be significantly different for different soil and solute types.


\sec Solute uptake models comparison

The scenario of this simulation is of loam soil, medium root length density, high potential transpiration and high initial concentration (Table 2). We compare all model types (no solute uptake -- NU; constant -- CU; linear -- LU and nonlinear -- NLU concentration dependent uptake rates). All simulations were made until the value of relative transpiration was equal or less than 0.001. The time step is dynamical (depends on the number of iterations for water and solute equations) and was set to vary between 0.1 and 2 seconds. The simulation for NU ended within near 3 days; for CU, LU and NLU, about 5 days.

In NU, salt is transported to the roots by convection, causing an accumulation of solutes at root surface. As water flux towards the root starts to decrease, salt is transported slower and carried away from the roots by diffusion (Figure 11). Because of the accumulation of salt in the root surface, the total head becomes limiting very fast and the transpiration is reduced faster than the other models (Figures 14 and 15).

In CU, as the salt uptake rate is constant (Figure 12), the concentration at root surface will decrease only if the uptake rate is larger than convection to the root surface. In the simulation, it happens in about half of the first day (Figure 11). This is very dependent on the uptake rate and water flux since for different conditions, the outcome could be different. Once the concentration at root surface is zero, the root behaves as a zero-sink, taking up solute at the same rate as which it arrives at the root, keeping the concentration there zero.

In LU and NLU, the concentration at root surface remains constant (Figure 11) until the convection to the root decreases as the water flux decreases (Figure 21). This behavior is really dependent of initial concentration and water flux values since, in this case, C0 at the beginning of simulation is greater than C2, thus the solute uptake equals the convection of solutes to the root. At around day 1, convection starts to decrease but the solute uptake is yet greater than \im . The solute uptake will become constant (and equal to \im) after concentration in root surface is less than \c2. This is clear in Figure 3a, where the osmotic head continues constant for a period of time after the beginning of the falling transpiration rate. At this point, active uptake starts since convection only is not capable to maintain solute uptake rate at \im. The concentration keeps decreasing at this constant rate until its value is less than \clim. It is assumed that, at this point, the uptake is not equal to the plant demand for solute (\im) due to the concentration dependence of the MM equation (Figure 0). The water flux and the concentration are too small, as so the active uptake that can not maintain the uptake rate at \im. Therefore, a second limiting condition occurs when $C<C_{lim}$ causing another fast decrease in transpiration (Figure 14).

The calculated concentrations $C_{lim}$ and \c2 depend on water flux and ion type (MM parameters $I_m$ and \km) meaning that the results can be quite different for other ion types and different values of initial water content.

Figure 12 shows the changes in solute flux at root surface for all models. At low concentrations (or at the second falling rate stage: $C<C_{lim}$), in LU and NLU, the solute flux decreases gradually over time (linear in relation to concentration but not linear in time) until the value of concentration is zero, where it will assume the zero-sink behavior.

The concentration profile through the distance from root axial center is shown in Figure 13. The different approaches (CU, LU and NLU) result in different final concentrations profiles. The concentration dependent models (LU and NLU) take up more solute from soil solution due to the higher uptake rate in the constant transpiration phase.

Figure 14 shows the relative transpiration as a function of time for the three model types. The proposed model is able to maintain the potential transpiration for a longer period of time due to the extraction by passive uptake only ($C>C_2$) that keeps the osmotic head constant, allowing pressure head to reach smaller values at the onset of the limiting hydraulic conditions, as can be seen in Figure 6.

Figure 15 shows that CU, LU and NLU have a more negative pressure head value for the onset of limiting hydraulic condition when compared to NU due to solute uptake that causes a increase in osmotic head (becomes less negative) and, in turn, decreasing pressure head. Thus, the first falling rate phase of relative transpiration extends in time. The solute uptake at the beginning of the simulation (for concentrations greater than \c2) caused a greater accumulation of solute in the plant and also influenced the final solute profile, in which LU and NLU have less solute left in the soil profile (Figure 13).

At the onset of the second falling rate phase ($C<C_{lim}$), water and solute fluxes decreases rapidly. In Figures 10 and 20 we can see that from day 4 to day 5, the fluxes are rapidly reduced, the water flux is near zero in the whole profile, meaning zero or really small convection. Thus, within this period, the transport of solute is made mainly by diffusion and the results of this diffusive transport can be visualized in Figures 17 and 18.

\medskip
\label[cxr]
\picw=13cm \cinspic cxt.pdf
\caption/f {Solute concentration in soil water at root surface as a function of time for no uptake (NU), constant (CU) and nonlinear (NLU) uptake models}
\medskip

\medskip
\label[cxt]
\picw=13cm \cinspic cxr.pdf
\caption/f {Solute concentration in soil water as a function of distance from axial center for no uptake (NU), constant (CU) and nonlinear (NLU) uptake models}
\medskip

\medskip
\label[fluxesxt]
\picw=13cm \cinspic fluxesxt.pdf
\caption/f {Solute and water fluxes at root surface as a function of time for no uptake (NU), constant (CU) and nonlinear (NLU) uptake models}
\medskip

\medskip
\label[Trs]
\picw=13cm \cinspic Trs.pdf
\caption/f {Relative transpiration as a function of time and pressure head for no uptake (NU), constant (CU) and nonlinear (NLU) uptake models}
\medskip




