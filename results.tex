\cleardoublepage

\def\diff{{\it diff}}

% COMMENTS FOR FURTHER DISCUSSION
%At low solute concentrations, the ion transport across membranes is limited by the diffusive flux of solute towards the membrane in case of high affinity uptake.Although mass flow towards the root caused by the transpiration stream is an additional solute transport mechanism to the root surface, its contribution to uptake at solute concentrations in the low micromolar range to uptake is estimated to be very low, often below 0.1% (cf.Tinker & Nye 2000) - Tinker P.B. & Nye P.H. (2000) Solute Movement in the Rhizosphere.Oxford University Press, New York, Oxford.

%IMPORTANT IN THE DISCUSSION, OR IN THE METHODOLOGY
%Uptake could be exclusively
%passive at CN higher than 0.4 mg N cm−3. This result
%supports the general viewpoint that plants do not ex-
%pend energy towards active nutrient uptake if the
%nutrient is sufficiently available, and that active
%uptake commences only when nutrient availability
%by passive uptake does not satisfy plant require-
%ments (Marschner 1995; Epstein and Bloom 2005;
%Šimůnek and Hopmans 2009). <== 142.pdf

\chap RESULTS AND DISCUSSION

\sec Linear (LU) versus nonlinear (NLU) solutions

%In all simulated scenarios, the difference between LU and NLU models occurs only at values of solute concentration in soil water ($C$) below the threshold value \clim. This is expected because of the nature of the piecewise MM equation used in the model.
%{\bf (EXPLAIN THAT THE DIFFERENCE OCCURS ONLY FOR C<Clim IN METHODOLOGY)}
%For both models, when solute concentration values at the root surface are higher than \c2, solute transport from soil to root is mostly driven by convection and, according to Equations \ref[eq_MM_mod] and \ref[eq_MM_linear], the uptake is passive only, with active uptake equal to zero. 
%For values of $C_0$ between the two threshold values (\c2 and \clim), the solute flux density is constant and NLU and LU are different only for $C$ values lower than $C_{lim}$ .

According to Equations \ref[eq_MM_mod] and \ref[eq_MM_linear], the difference between LU and NLU occurs only at the condition of concentration values at root surface below $C_{lim}$, due to the different considered boundary condition of each model. In this section, we will analyse whether the difference in the outputs of LU and NLU is significant to decide the situations that one model can be used over the another. For now on, the condition $C_0 < C_{lim}$ will be referred as `limiting concentration condition', or LCC, for short. 

%FIRST, TALK ONLY ABOUT THE RELATIVE DIFFERENCES. SHOW A TABLE OF VALUES. THEN GO TO THE STATISTICAL TEST TO SHOW THE SIGNIFICANCE. THEN SHOW THE DIFFERENCES IN Tr.

The differences between the two models were not significant when the whole time series is considered in calculations (from the initial to the final time step).
The reason is that the low value of $C_{lim}$ causes the LCC period to endure little time. 
In scenario 1, for instance, the duration of the LCC period is of about 1/30 of the total simulation. 
Moreover, due to the low value of $C_{lim}$, the LCC always occurs with very low solute uptake rates in all simulation scenarios. 
As consequence, the difference between the two models in all output variables are little noticeable at this time scale.
For example, the amount of solute extracted by the roots at the end of simulation are nearly the same for both models (Figure \ref[accumxt]). 
In addition, the LCC occurs at values of relative transpiration around 0.03 or less -- in a situation that the plant is already at high (water and/or osmotic) stress levels -- and close to the end of simulation (that stops when $Tr \leq 0.001$). 
%Because the difference between the models starts to occur at $T_r$ values around 0.03 or less, close to the end of simulation (that stops when $Tr \leq 0.001$), the time period in which the models to differ is small. 
A second reduction in the relative transpiration occurs and it can be noticed by an abrupt decrease in the transpiration rate (Figure \ref[diff_tr]).
A possible explanation is that, at this point, the solute uptake rate decreases, building a more negative value for the osmotic head and, consequently, reducing also the water uptake rate. It can be seen in Figure \ref[diff_tr] that the nonlinear equation causes a smoother $T_r$ reduction, which seems to be more appropriate for a natural phenomena.

\medskip
\label[accumxt]
\picw=13cm \cinspic accumxt.pdf
\caption/f {Cumulative solute uptake as a function of time for all scenarios. Dashed lines represent the nonlinear model}
\medskip

\medskip
\label[diff_tr]
\picw=17cm \cinspic diffs_tr.pdf
\caption/f {Relative transpiration as a function of time for all scenarios. Dashed lines represent the nonlinear model}
\medskip

There are two ways of making the difference between the two models be relevant: a) increase the value of $C_{lim}$ to make the LCC period be achieved faster, or b) look at that difference by a different time scale. 
We could not find a scenario that matched the first option. 
By analysing the Equation \ref[eq_clim2], it is noticeable that the maximum value for $C_{lim}$ is equal to $K_m$ and only if $q_0 \gg Km^2$. 
But, even if the value of $q_0$ attends this condition, Equation \ref[eq_c2] shows that $C_2$ would be higher and, since the uptake function is a piecewise equation, $C_0$ would have to be decreased until it reaches $C_{lim}$. 
At that time, the water flux would be low and the LCC period would be still small. 
However, by changing the scope and computing the differences between the two models starting from the time where LCC starts, a significant difference is noticed. 

Table \ref[tab_diff] shows the relative difference between the two models for concentration at root surface as function  of time ($C_0(t)$) and for concentration as a function of distance from axial center ($C(r)$) at the end of simulation, considering the shorter time scale (beginning from time where LCC starts). The Mann-whitney U test showed a significant difference for $C_0(t)$ but not for $C(r)$. The changes in root surface concentrations happen rapidly due to the decrease in solute uptake rate and it increase with time (Figure \ref[diff_t]), which is not the case for the concentration over the radial distance that the solute concentration for each segment is replenished by its neighbor, making the difference small with the increasing distance (Figure \ref[diff_r]). Scenario 4 never reached the LCC and it is presented in the results to show that the differences occurs only starting from this condition.


%The relative differences between the models, for simulation scenarios 1 to 4 at the LCC, are shown in Table \ref[tab_diff]. 
%Scenarios 1 and 2 showed similar results for $C_0(t)$, meaning that the potential transpiration has little impact on the computation of the relative difference. 
%Scenario 3 showed a greater difference because the limiting concentration condition was reached earlier than that of the other scenarios due to the lower initial concentration. 
%Nevertheless, for $C(r)$, the relative differences were of about 2.5\% for scenarios 1, 2 and 3. 
%This is due to the fact that at this condition, the flux of water and solutes towards the roots is higher near the roots and becomes small with the increasing distance.
%This gradient makes water and solute flow towards the roots. The higher the uptake, the higher the gradient and the solute transport. As NLU has a higher uptake rate, $C_0$ is smaller and the gradient is higher, resulting in little difference between LU and NLU.
%
\input tables/diff.tex

\medskip
\label[diff_t]
\picw=13cm \cinspic diffs_t.pdf
\caption/f {Difference between the solute concentration in soil water at root surface ($C_0$) for LU and NLU and $C_0$ as a function of time; and the relative difference -- Scenario 1}
\medskip

\medskip
\label[diff_r]
\picw=13cm \cinspic diffs_r.pdf
\caption/f {Difference between the solute concentration in soil water ($C$) at the end of simulation for LU and NLU and $C$ as a function of distance from axial center; and the relative difference -- Scenario 1}
\medskip

%All simulations had similar results then, to illustrate, we show the analysis of scenario 1 in Figures \ref[diff_t] and \ref[diff_r].

%The relative difference between both models, for the time selected outputs ($C_0(t)$ and $Ac(t)$), becomes significant only when calculated for this condition. 
%If the whole time series is analyzed, the difference is not significant since the condition of $C_0 < C_{lim}$ is achieved near the end of simulations and its contribution is small due to the low solute uptake rate at this time.

%PUT PLOTS TO EXPLAIN THIS

%This small flux can be observed in Figure \ref[diff_r]. At the end of simulation, the relative difference of LU and NLU was of 2.37\%. The relative difference between LU and NLU, in this case, ranged from XXX\% to XXX\% in the simulated scenarios.

%Figure~\ref[diff_t] shows that the difference between the estimated concentration at the root surface, for the linear (CL) and non-linear (CNL) models as a function of time and when $C_0 < C_{lim}$ for Scenario 1, is of around 43\%. 
%For the scenarios 2 to 4, the difference were of XXX\%, XXX\% and XXX\% respectively. 
%The Mann-Whitney U test for all scenarios when $C_0 < C_{lim}$, and the same output, showed a significant difference between the two models.
%When the same test is applied considering the whole time series, it shows non-significant difference.
%The list of the p-values for all cases and scenarios is presented in Table \ref[Utest].
%Figure~\ref[diff_r] shows the relative difference of concentration as a function of radial distance of 2.37\% and it can be noticed visually that the concentration profile does not change so much. 
%The U test corroborates with the visual analysis.
%The results for scenarios 2 to 4 were similar and it is not showed in the plots.

These results means that although there is indeed a difference between the two models, they are important only if looked in a shorter time scope. 
If the variable of interest arises from the whole time scope (accumulated uptake, for instance) or it is the concentration as a function of the radial distance, the differences are not important and both models can be used. 
In that case, it is preferred to use LU since a bigger time step can be used making the simulations faster.
On the other hand, considering that the metabolic response of the plant for different levels of concentration at root surface is immediate, or in situations where the value of $C_0$ is important, it is plausible to consider NLU for the simulations since it is the more realistic model.  
In this thesis we choose to use NLU since the values of $C_0$ are important to determine, precisely, the active and passive contribution to the solute uptake, which will be analysed in the next section.
% and, therefore, the differences between the two models outputs -- according to Equations  \ref[error_abs] and \ref[error_rel] -- were calculated for this time period only. 

%%For the Scenario 1, this situation occurred at nearly the fourth simulated day. 
%%It was observed that although $C_0$ values are distinct between the two models, it does not add a significant contribution to the final extracted solute amount. 
%CNL tends to be smaller than CL because the uptake rate of NLU is always higher than that of LU, in accordance to their govern equations.
%Figure~\ref[diff_r], however, shows that the final concentration profile is nearly identical, presenting a difference of 0.3\%, and that the difference in the concentration at the root surface does not lead to a significant additional uptake. 
%Moreover, Figure~\ref[accumxt] shows that the 
%Nevertheless, in all simulation scenarios, the total solute taken up in the end of simulation was similar in both models, even with the time interval of greater uptake of NLU.
%%The absolute and relative differences between the estimated solute concentration for the linear (CL) and nonlinear (CNL) model was calculated according to Equations \ref[error_abs] and \ref[error_rel], respectively.
%%From the \diff{} value resulted from Equations \ref[error_abs] and \ref[error_rel], it can be inferred which model had a greater uptake.
%%For instance, if \diff{}~>~0, then the solute concentration estimated by the linear model (CL) was greater than that of the non-linear model (CNL). 
%%As the only output of solute is caused by root extraction, a lower concentration indicates a higher uptake so that NLU taken up more solute for this case. 
%%The opposite is true for \diff{}~<~0, and for \diff{}~=~0 both models had the same uptake.
%
%%Figures \ref[diff_t] and \ref[diff_r] show the differences between $C$ outputs of LU and NLU as a function of time and radial distance of axial center, respectively, for scenario 1.
%%$$
%%\label[eq_diff]
%%d\!i\!f\!f\!=C\!L-C\!N\!L \eqmark
%%$$
%%%
%%where $C\!L$ and $C\!N\!L$ are the concentrations for LU and for NLU, respectively.
%When comparing differences in concentration between the two models, one has to be aware that $C\!N\!L<C\!L$ means that the uptake for NLU is greater than that of LU because, for a higher uptake, a higher amount of solute goes out from soil solution to inside the plant.
%Therefore if $d\!i\!f\!f\!<0$ then CNL > CL and LU uptake is greater; if $d\!i\!f\!f\!>0$ then $C\!N\!L<C\!L$ and NLU uptake is greater. Said that, we can see in Figure \ref[diff_t] that the uptake for NLU is greater then LU uptake at times when $C<C_{lim}$. This reflects a change also in the concentration profile for the latter times. Figure \ref[diff_r] shows the concentration profile at day 5 and the difference between $C\!L$ and $C\!N\!L$ through the profile. The higher NLU uptake increases the concentration gradient causing a higher solute flux (most diffusive since water flux is very small) from soil towards the root, resulting in a slightly higher concentration for NLU close to root surface ($d\!i\!f\!f\!.<0$).
%
%
%
%% This is the actual discussion of U test
%The Mann--Whitney U test for (Table \ref[Utest]) shows that, for scenarios 1, 2 and 4, the differences between $C\!L$ and $C\!N\!L$ are significant for concentration values at times where $C<C_{lim}$ and, for scenario 3, both models have similar results. INCOMPLETE (in development)
%%Although the differences in concentration through time and radial distance are significant
%
%\input tables/Utest
%
%Nevertheless, the difference between LU and NLU is negligible for cumulative uptake. A difference of 80.9\% of concentration over time (Figure \ref[diff_t]) corresponds to only 0.318\% in the final concentration profile (Figure \ref[diff_r]) because the uptake at times where $C<C_{lim}$ is really low. It can be seen at the cumulative uptake plot (Figure \ref[accumxt]) the insignificant effect of this difference (for both models, the cumulative solute uptake is nearly the same).
%
%Similar to Figures \ref[diff_t] and \ref[diff_r], the results of the relative accumulated error, according to Equation \ref[error_rel], were of 64.975\% and 0.739\%; 37.364.3\% and 0.041\%; 36.144\% and 0.027\% over time and distance for scenarios 2, 3 and 4, respectively.
%%STATISTICS FOR U TEST HERE
%
%%IMPROVE, POLISH
%%The Mann--Whitney U test results corroborates with the differences analysis. Assuming a confidence interval of 95\%, {\it p}-values below 0.05 indicates that the two models are different (one has greater values than the other), i.e. the hypothesis $H_0$ is rejected. Table \ref[Utest] shows that, for scenarios 2, 3 and 4, concentration $C_0$ is significantly different for LU and NLU, but when analysing the cumulative uptake (for the same period of time where $C<C_{lim}$) the difference is not significant.
%

It is important to cite here the issues with stabilization in the numerical solution of NLU.
%In addition, some part of the differences was due to stability problems with the numerical solution for NLU. 
The change from equation 21 to equation 25 (change of boundary condition, from constant to nonlinear uptake rate) makes the numerical solution take some time to stabilize at the initial times if the time and space steps are not chosen properly. 
Many different time and space steps combinations had to be tested, for each scenario, to eliminate the problem. 
We found that, as a general rule, a small time step (commonly less than 0.1 s) can be used when the default space step is used. 
In situations that a finner space step is necessary, smaller values of $\Delta t$ are required. 
As mentioned before, decreasing the time step leads to an increase of the number of calculatoins making the simulation takes longer to complete. 
In our simulation scenarios, the time that they took to complete ranged from 40 minutes to 1.5 days.
%It needs to be found an optimal value for time and space step relation. 
%Stabilization problems were not found in LU.

%Since the differences between LU and NLU occurs for low concentration values and low solute flux, changes in relative transpiration are also negligible. The oscillation in the results due to the stabilization problem of the numerical solution is more likely to be noticed than the difference between the models. Figure \ref[diff_tr] shows the relative transpiration as a function of time and details the part where the oscillation occurs for each scenario. The absolute and relative differences are all due to oscillation problem, since $T_r$ turns out to be the same after complete convergence.

\citeonline[roose2009] stated that, for numerical solutions of convection-dispersion equation, the convective part might use an explicit scheme because convection, unlike diffusion, occurs only in one direction thus the solution at the following time step depends only on the values within the domain of influence of the previous time step. This set bounds on time and space steps, with a condition of stability given by ${r_0 q_0 \Delta t \over D} < \Delta r$. As the proposed model uses a fully implicit scheme, that might be the cause of the stabilization problems.

%No significant difference between LU and NLU was verified in all output files (water and solute flux, concentration profile, heads and relative transpiration – Figures 1a, 2a, 3, 4, 5 and 6), but a more detailed analysis was made by analyzing the overall results (values of solute concentration for all output times -- Figure 9b) and low concentration results (values of solute concentration for output times where $C<C_{lim}$ -- Figures 8 and 9a).

%For low concentration (Figures 8 and 9a), it is noticeable that LU presents values of solute concentration higher than NLU. This is expected because in LE, solute uptake is always less than in NLU due to its linearization. Nevertheless, when the overall results are analyzed (Figure 9b), we can see that the difference is small and can be neglected. This conclusion (negligible difference) can be verified once more when analyzing the cumulative solute uptake (Figure 10). 

%The conclusion, since no significant difference between LU and NLU was found, can be either to choose LU as it takes less time to run and has no stabilization problems, or to choose NLU except for the cases in which the stability problem is significantly high. Note that this is the conclusion for those specific scenarios as the results can be significantly different for different soil and solute types.

\sec Model results

According to the analysis of the differences between the linear (LU) and nonlinear (NLU) solution for the solute uptake equation showed in the last section, the simulations of scenarios 1 to 7 (Table \ref[tab_scenarios]) were made using NLU. In this section, the results of the main output variables ($C_0(t)$, $C(r)$ and $T_r(t)$) are presented for each scenario, as well as the active and passive contributions to uptake.

%The model simulates the solute transport in soil according to the convection-dispersion equation \ref[eq_complete_solute] and boundary contitions of zero solute flux at the outer segment (Equation \ref[eq_bcrm]) and a concentration dependent solute uptake at root surface (inner segment, Equation \ref[eq_inner_bound]). 
%Those equations have the soil water content as a variable (the governing solute transport equation does not consider steady-state condition in respect to water flow), which is determined by the water flow equation. 
%The soil hydraulics functions and parameters determine how water flows towards the root and, consequently, the solute flow. 
%Therefore, different soil types will interact differently with water and solute flow. 
The simulation scenarios 1, 6 and 7 have the soil hydraulic properties as the only difference in the input parameters, being loam, sand and clay soil, respectively.
The concentration profiles were different for each soil type (Figure \ref[fig_C_r]).
Finer-texture soils offer less resistance to water transport and to convective solute flux.

Figure \ref[fig_C_r] shows the response of $C(r)$ to different input values of soil, root density, initial concentration and potencial transpiration. 

\medskip
\label[fig_C_r]
\picw=17cm \cinspic C_r.pdf
\caption/f {Responses of $C(r)$ to different values of the selected input parameters (a) soil, (b) root density, (c) initial concentration, and (d) potencial transpiration, simulated using the nonlinear uptake model (NLU)}
\medskip

\medskip
\label[fig_C_t]
\picw=17cm \cinspic C_t.pdf
\caption/f {Changes in $C_0(t)$ according to different values of the selected parameters}
\medskip

\medskip
\label[fig_Tr_t]
\picw=17cm \cinspic Tr_t.pdf
\caption/f {Changes in $T_r(t)$ according to different values of the selected parameters}
\medskip

\medskip
\label[fig_contrib_c]
\picw=17cm \cinspic all_contributionsxc.pdf
\caption/f {Still to put}
\medskip

\medskip
\label[fig_contrib_t]
\picw=17cm \cinspic all_contributionsxt.pdf
\caption/f {Still to put}
\medskip

\sec Solute uptake models comparison

In NU, salt is transported to the roots by convection, causing an accumulation of solutes at root surface. As water flux towards the root starts to decrease, salt is transported slower and carried away from the roots by diffusion (Figure \ref[cxt]). Because of the accumulation of salt in the root surface, the total head becomes limiting very fast and the transpiration is reduced faster than the other models (Figure \ref[Trs]).

In CU, as the salt uptake rate is constant (Figure \ref[fluxesxt]), the concentration at root surface will decrease only if the uptake rate is larger than convection to the root surface. In the simulation, it happens in about half of the first day (Figure \ref[cxt]). This is very dependent on the uptake rate and water flux since for different conditions, the outcome could be different. Once the concentration at root surface is zero, the root behaves as a zero-sink, taking up solute at the same rate as which it arrives at the root, keeping the concentration there zero.

In NLU, the concentration at root surface remains constant (Figure \ref[cxt]) until the convection to the root decreases as the water flux decreases (Figure \ref[fluxesxt]). This behavior is really dependent of initial concentration and water flux values since, in this case, $C_0$ at the beginning of simulation is greater than $C_2$, thus the solute uptake equals the convection of solutes to the root. At around day 1, convection starts to decrease but the solute uptake is yet greater than the plant demand (\im) due to convection. The solute uptake becomes constant (and equal to \im) after concentration in root surface is less than \c2. This is clear in XXXFigure 3a, where osmotic head continues constant for a period of time after the beginning of the falling transpiration rate. At this point, active uptake starts since convection only is not capable to maintain solute uptake rate at \im. The concentration keeps decreasing at this constant rate until its value is less than \clim. It is assumed that, at this point, the uptake is not equal to the plant demand for solute (\im) due to the concentration dependence of the MM equation (Figure \ref[fig_MM]). The water flux and the concentration are small as well as the active uptake, that can not maintain the uptake rate at \im. Therefore, a second limiting condition occurs when $C<C_{lim}$ causing another fast decrease in transpiration (Figure \ref[Trs]).
The calculated concentrations $C_{lim}$ (Eq. \ref[eq_clim]) and \c2 (Eq. \ref[eq_c2]) depend on water flux and ion type (MM parameters $I_m$ and \km) meaning that the results can be quite different for other ion types and different values of initial water content.

Figure \ref[fluxesxt] also shows the changes in solute flux at root surface for all models. At low concentrations (or at the second falling rate stage: $C<C_{lim}$), in NLU, the solute flux decreases gradually over time until the value of concentration is zero, where it will assume the zero-sink behavior.

The concentration profile through the distance from root axial center is shown in Figure \ref[cxr]. The different approaches (NU, CU and NLU) result in different final concentrations profiles. The concentration dependent model NLU takes up more solute from soil solution due to the higher uptake rate in the constant transpiration phase.

Figure \ref[Trs] shows the relative transpiration as a function of time for the three model types. The proposed model is able to maintain the potential transpiration for a longer period of time due to the extraction by passive uptake only ($C>C_2$) that keeps the osmotic head constant, allowing pressure head to reach smaller values at the onset of the limiting hydraulic conditions, as can be seen in Figure 6.

Figure \ref[Trs] shows that CU and NLU have a more negative pressure head value for the onset of limiting hydraulic condition when compared to NU due to solute uptake that causes a increase in osmotic head (becomes less negative) and, in turn, decreases pressure head. Thus, the first falling rate phase of relative transpiration extends in time. The solute uptake at the beginning of the simulation (for concentrations greater than \c2) caused a greater accumulation of solute in the plant and also influenced the final solute profile, in which LU and NLU have less solute left in the soil profile (Figure 13).

At the onset of the second falling rate phase ($C<C_{lim}$), water and solute fluxes decreases rapidly. In Figures XXX10 and 20 we can see that from day 4 to day 5, the fluxes are rapidly reduced, the water flux is near zero in the whole profile, meaning zero or really small convection. Thus, within this period, the transport of solute is made mainly by diffusion and the results of this diffusive transport can be visualized in Figures XXX17 and 18.

\medskip
\label[cxt]
\picw=13cm \cinspic cxt.pdf
\caption/f {Solute concentration in soil water at root surface as a function of time for no uptake (NU), constant (CU) and nonlinear (NLU) uptake models}
\medskip

\medskip
\label[cxr]
\picw=13cm \cinspic cxr.pdf
\caption/f {Solute concentration in soil water as a function of distance from axial center for no uptake (NU), constant (CU) and nonlinear (NLU) uptake models}
\medskip

\medskip
\label[fluxesxt]
\picw=13cm \cinspic fluxesxt.pdf
\caption/f {Solute and water fluxes at root surface as a function of time for no uptake (NU), constant (CU) and nonlinear (NLU) uptake models}
\medskip

\medskip
\label[Trs]
\picw=13cm \cinspic Trs.pdf
\caption/f {Relative transpiration as a function of time and pressure head for no uptake (NU), constant (CU) and nonlinear (NLU) uptake models}
\medskip

\sec Guidelines to upscale the proposed model

\cleardoublepage
\medskip
\label[fig_C0]
\picw=11cm \cinspic cxt1.pdf
\caption/f {Soil solution concentration at root surface as a function of time for constant (CU), zero (ZU) and nonlinear (NLU) uptake models}
\medskip

\medskip
\label[fig_C]
\picw=11cm \cinspic cxr1.pdf
\caption/f {Soil solution concentration as a function of distance from axial root center for constant (CU), zero (ZU) and nonlinear (NLU) uptake models}
\medskip

\medskip
\label[fig_accsol]
\picw=11cm \cinspic accumxt.pdf
\caption/f {Cumulative solute uptake as a function of time for constant (CU), zero (ZU) and nonlinear (NLU) uptake models}
\medskip

\vfill\break
\medskip
\label[fig_heads1]
\picw=11cm \cinspic headsxt.pdf
\caption/f {Pressure ($h$), osmotic ($h_\pi$) and total ($H$) heads as a function of time for constant (CU) and zero (ZU) uptake models}
\medskip

\medskip
\label[fig_heads2]
\picw=11cm \cinspic headsNLxt.pdf
\caption/f {Pressure ($h$), osmotic ($h_\pi$) and total ($H$) heads as a function of time for nonlinear (NLU) uptake model}
\medskip

\vfill\break
\medskip
\label[fig_qst]
\picw=11cm \cinspic qsxt1.pdf
\caption/f {Solute flux at root surface as a function of time for constant (CU), zero (ZU) and nonlinear (NLU) uptake models}
\medskip

\medskip
\label[fig_qst_detail]
\picw=11cm \cinspic qsxt1_detail.pdf
\caption/f {Detail of solute flux at root surface as a function of time, at times when $C_0<C_{lim}$, for constant (CU), zero (ZU) and nonlinear (NLU) uptake models}
\medskip

%\medskip
%%\label[Trs]
%\picw=11cm \cinspic qxt1.pdf
%%\caption/f {Relative transpiration as a function of time and pressure head for no uptake (NU), constant (CU) and nonlinear (NLU) uptake models}
%\medskip

\vfill\break
\medskip
\label[fig_Tr]
\picw=11cm \cinspic Trxt1.pdf
\caption/f {Relative transpiration as a function of time for constant (CU), zero (ZU) and nonlinear (NLU) uptake models}
\medskip

\medskip
\label[fig_Tr_detail]
\picw=11cm \cinspic Trxt1_detail.pdf
\caption/f {Detail of Relative transpiration as a function of time, at times when $C_0<C_{lim}$, for linear (LU) and nonlinear (NLU) uptake models}
\medskip

\vfill\break
\medskip
\label[fig_Im_sensit_curve]
\picw=11cm \cinspic Im_x_C0.pdf
\caption/f {Response of the time at the onset of concentration reduction at root surface for different values of $I_m$ parameter}
\medskip

\medskip
\label[fig_etas]
\picw=11cm \cinspic bar_n.pdf
\caption/f {Relative partial sensitivity ($\eta$) of selected model outputs to $I_m$ parameter. tC0, tTr, tend~-~times at the onset of falling concentration at root surface and transpiration and time at the end of simulation, respectively; hTr and hosTr~-~pressure and osmotic heads at the onset of falling transpiration; acc~-~total amount of solute taken up by the root at the end of simulation}
\medskip

\medskip
\label[sensitivity_all]
\picw=11cm \cinspic final_every_bar.pdf
\caption/f {Relative partial sensitivity of end time of simulation ($t_{end}$) and osmotic head ($h_{os}$) to selected plant and soil hydraulic sensitivity parameters for scenarios 1 to 7}
\medskip
