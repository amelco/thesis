\cleardoublepage

\def\diff{{\it diff}}
\def\tred {\localcolor\Red}
\def\tblue {\localcolor\Blue}

% COMMENTS FOR FURTHER DISCUSSION

%IMPORTANT IN THE DISCUSSION, OR IN THE METHODOLOGY
%Uptake could be exclusively
%passive at CN higher than 0.4 mg N cm−3. This result
%supports the general viewpoint that plants do not ex-
%pend energy towards active nutrient uptake if the
%nutrient is sufficiently available, and that active
%uptake commences only when nutrient availability
%by passive uptake does not satisfy plant require-
%ments (Marschner 1995; Epstein and Bloom 2005;
%Šimůnek and Hopmans 2009). <== 142.pdf

\chap RESULTS AND DISCUSSION

\label[result_luxnlu]
\sec Linear (LU) versus non-linear (NLU) solutions

%In all simulated scenarios, the difference between LU and NLU models occurs only at values of solute concentration in soil water ($C$) below the threshold value \clim. This is expected because of the nature of the piecewise MM equation used in the model.
%{\bf (EXPLAIN THAT THE DIFFERENCE OCCURS ONLY FOR C<Clim IN METHODOLOGY)}
%For both models, when solute concentration values at the root surface are higher than \c2, solute transport from soil to root is mostly driven by convection and, according to Equations \ref[eq_MM_mod] and \ref[eq_MM_linear], the uptake is passive only, with active uptake equal to zero. 
%For values of $C_0$ between the two threshold values (\c2 and \clim), the solute flux density is constant and NLU and LU are different only for $C$ values lower than $C_{lim}$ .

According to Equations \ref[eq_MM_mod] and \ref[eq_MM_linear], the difference between LU and NLU will only shows up when the solute concentration at the root surface is lower than $C_{lim}$, resulting in a different boundary condition for each model. 
This condition ($C_0 < C_{lim}$) will be referred to as `limiting concentration condition', abbreviated as LCC. 
In this section, we will analyze the predictions of LU and NLU and if these predictions are significantly different, in which case one model would be preferred over the other. 

%FIRST, TALK ONLY ABOUT THE RELATIVE DIFFERENCES. SHOW A TABLE OF VALUES. THEN GO TO THE STATISTICAL TEST TO SHOW THE SIGNIFICANCE. THEN SHOW THE DIFFERENCES IN Tr.

The differences between the two models were not significant when the complete time series is considered in calculations (from the initial to the final time step).
This is due to the fact that the low value of $C_{lim}$ causes LCC to occur only during a short fraction of time. 
In scenario 1, for instance, the duration of occurrence of LCC is about 3\% of the total simulation time. 
Moreover, due to the low value of $C_{lim}$, the LCC always coincides to very low solute uptake rates in all simulation scenarios. 
As a consequence, the difference in predictions for the two models are of minor relevance at this time scale.
Figure \ref[accumxt] gives an example of this: the cumulated amount of solute extracted by the roots until the end of the simulation is shown to be nearly the same for both models. 
In addition, the LCC occurs at values of relative transpiration ($T_r$, the ratio between actual and potential transpiration) around 0.03 or less -- a condition in which the plant is at high (water and/or osmotic) stress levels -- and close to the end of simulation 
{\tred (that stops when $Tr \leq 0.001$). {\it QM: Isso deveria estar tambem dos Material e metodos, e acho que nao esta?}}
%Because the difference between the models starts to occur at $T_r$ values around 0.03 or less, close to the end of simulation (that stops when $Tr \leq 0.001$), the time period in which the models to differ is small. 
{\tred A second reduction {\it QM: Mas voce nem explicou porque ocorreu a primeira reducaoo. Explique melhor isso. Ideia: indicar as reducoes na figura (numerar) e no texto explicar porque elas ocorrem}} 
in the relative transpiration occurs and it can be noticed by an abrupt decrease in the transpiration rate (Figure \ref[diff_tr]).
A possible explanation is that, at this point, 
{\tred the solute uptake rate decreases, resulting in a decreasing (more negative) value for the osmotic head {\it QM: Isso me lembra algo que descrevi tb no trabalho de 2009? E voce poderia citar/comparar aqui.}}
 and, consequently, reducing also the water uptake rate.
Therefore, this reduction depends on the value of $q_0$, which directly affects the $C_{lim}$ value (Equation \ref[eq_clim]), and abruptly reduces the relative transpiration due to the combined water and osmotic stress.
This limiting concentration, here only numerically computed, may be controlled by physical or physiologic processes and needs to be investigated.
The NLU causes a smoother $T_r$ reduction over time (Figure \ref[diff_tr]), apparently more appropriate for a natural phenomenon.

\medskip
\label[accumxt]
\picw=13cm \cinspic accumxt.pdf
\caption/f {Cumulative solute uptake as a function of time for scenario 1, predicted be the linear (solid line) and the non-linear (dashed line) model}
\medskip

\medskip
\label[diff_tr]
\picw=17cm \cinspic diffs_tr.pdf
\caption/f {Relative transpiration as a function of time for scenarios 1 to 4, predicted by the linear (solid line) and the non-linear (dashed line) model}
\medskip

%There are two ways of making the difference between the two models be relevant: 
Differences between both model predictions can become more  significant either by
a) increasing the value of $C_{lim}$ making LCC to be reached in a earlier stage, or b) looking at the results on a different time scale. 
{\tred We could not find a scenario that matched the first option. {\it QM: Nao entendi isso. Nao eh possivel um cenario assim?}}
{\tblue {\it My answer: The reason is that \clim{} value can not be chosen. It is numerically calculated by Equation \ref[eq_clim] and its value changes with $q_0$ variation.}}
The value of $C_{lim}$ will be less or equal $K_m$ if and only if $q_0 \gg Km^2$. 
But, even if the value of $q_0$ satisfies this condition, Equation \ref[eq_c2] shows that $C_2$ would be higher and, since the uptake function is a piecewise equation, $C_0$ would have to be decreased until it reaches $C_{lim}$ {\tblue for LCC to occur}.
Then, at occurrence of LCC, the water flux would be low and the LCC period still small, {\tblue which would, again, result in a non-significant difference between the models}. 
On the other hand, if only the predictions during the occurrence of LCC are computed, differences become significant. 

Table \ref[tab_diff] shows the relative difference between the model predictions for concentration at the root surface as function of time ($C_0(t)$), considering the period of LCC only (beginning from the time where LCC starts), and for the solute concentration as a function of distance from the axial center ($C(r)$) at the end of the simulation. 
The Mann-Whitney U test showed a significant difference for $C_0(t)$ but not for $C(r)$. 
The difference ({\tred {\it diff. QM: Nao poderia chamar isso pelo simbolo DELTA ou algo assim?}}) between LU and NLU increases with time (Figure \ref[diff_t]).
%due to the abrupt decrease of solute uptake rate at root surface (that causes the second $T_r$ reduction).
%The change in $C_0$ root surface concentrations happens rapidly due to the decrease in solute uptake rate and it increases with time 
Concentration as a function of radial distance (Figure \ref[diff_r]) shows a different behaviour, as solutes in each segment are replenished by the neighboring segment, making the difference smaller with increasing distance. 
Scenario 4 never reached the LCC and Table \ref[tab_diff] shows the differences only to occur when starting from a $C_0 < C_{lim}$ condition.


%The relative differences between the models, for simulation scenarios 1 to 4 at the LCC, are shown in Table \ref[tab_diff]. 
%Scenarios 1 and 2 showed similar results for $C_0(t)$, meaning that the potential transpiration has little impact on the computation of the relative difference. 
%Scenario 3 showed a greater difference because the limiting concentration condition was reached earlier than that of the other scenarios due to the lower initial concentration. 
%Nevertheless, for $C(r)$, the relative differences were of about 2.5\% for scenarios 1, 2 and 3. 
%This is due to the fact that at this condition, the flux of water and solutes towards the roots is higher near the roots and becomes small with the increasing distance.
%This gradient makes water and solute flow towards the roots. The higher the uptake, the higher the gradient and the solute transport. As NLU has a higher uptake rate, $C_0$ is smaller and the gradient is higher, resulting in little difference between LU and NLU.
%
\input tables/diff.tex

\medskip
\label[diff_t]
\picw=13cm \cinspic diffs_t.pdf
\caption/f {
{\tred (bottom) Solute concentration at the root surface as a function of time for scenario 1 during LCC occurrence predicted by the linear (solid line) and non-linear (dashed line) model;
(top) absolute difference between the two models and its relative difference value according to Equation \ref[error_C] {\it QM: Nao entendo a figura de cima. Ela eh um dividido pelo outro? Nao parece.}} {\tblue {\it My answer: a linha de cima eh a diferenca absoluta, e o valor em porcentagem, a relativa}}
}
\medskip

\medskip
\label[diff_r]
\picw=13cm \cinspic diffs_r.pdf
\caption/f {(bottom) Solute concentration as a function of distance from axial center for scenario 1 during LCC occurrence predicted by the linear (solid line) and non-linear (dashed line) model;
(top) absolute difference between the two models and its relative difference value according to Equation \ref[error_C] 
}
\medskip

%All simulations had similar results then, to illustrate, we show the analysis of scenario 1 in Figures \ref[diff_t] and \ref[diff_r].

%The relative difference between both models, for the time selected outputs ($C_0(t)$ and $Ac(t)$), becomes significant only when calculated for this condition. 
%If the whole time series is analyzed, the difference is not significant since the condition of $C_0 < C_{lim}$ is achieved near the end of simulations and its contribution is small due to the low solute uptake rate at this time.

%PUT PLOTS TO EXPLAIN THIS

%This small flux can be observed in Figure \ref[diff_r]. At the end of simulation, the relative difference of LU and NLU was of 2.37\%. The relative difference between LU and NLU, in this case, ranged from XXX\% to XXX\% in the simulated scenarios.

%Figure~\ref[diff_t] shows that the difference between the estimated concentration at the root surface, for the linear (CL) and non-linear (CNL) models as a function of time and when $C_0 < C_{lim}$ for Scenario 1, is of around 43\%. 
%For the scenarios 2 to 4, the difference were of XXX\%, XXX\% and XXX\% respectively. 
%The Mann-Whitney U test for all scenarios when $C_0 < C_{lim}$, and the same output, showed a significant difference between the two models.
%When the same test is applied considering the whole time series, it shows non-significant difference.
%The list of the p-values for all cases and scenarios is presented in Table \ref[Utest].
%Figure~\ref[diff_r] shows the relative difference of concentration as a function of radial distance of 2.37\% and it can be noticed visually that the concentration profile does not change so much. 
%The U test corroborates with the visual analysis.
%The results for scenarios 2 to 4 were similar and it is not showed in the plots.

These results indicate that the difference in predictions of both models are relevant only {\tred during LCC. {\it QM: Acho que eh isso? Descreva melhor isso, simplesmente dizer que depende da escala de tempo nao eh correto. Depende de qual parte do tempo voce esta analisando.}} 
For predictions that result from the entire simulation period, like accumulated uptake, or for concentration as a function of the radial distance, the differences are very small and both models perform equally. 
In these cases, LU may be preferred since it allows larger time steps making simulation to run faster.
On the other hand, by considering that the metabolic response of the plant for different levels of concentration at root surface is immediate -- only a slightly change in $C_0$ causes a instantaneous plant reaction -- or in situations where the value of $C_0$ during LCC is important, NLU could be preferred as the mechanistically most correct model.  
In this thesis we chose to use NLU since the values of $C_0$ are important to determine, precisely, the active and passive contributions to the solute uptake, which will be analyzed in the next section.
% and, therefore, the differences between the two models outputs -- according to Equations  \ref[error_abs] and \ref[error_rel] -- were calculated for this time period only. 

%%For the Scenario 1, this situation occurred at nearly the fourth simulated day. 
%%It was observed that although $C_0$ values are distinct between the two models, it does not add a significant contribution to the final extracted solute amount. 
%CNL tends to be smaller than CL because the uptake rate of NLU is always higher than that of LU, in accordance to their govern equations.
%Figure~\ref[diff_r], however, shows that the final concentration profile is nearly identical, presenting a difference of 0.3\%, and that the difference in the concentration at the root surface does not lead to a significant additional uptake. 
%Moreover, Figure~\ref[accumxt] shows that the 
%Nevertheless, in all simulation scenarios, the total solute taken up in the end of simulation was similar in both models, even with the time interval of greater uptake of NLU.
%%The absolute and relative differences between the estimated solute concentration for the linear (CL) and non-linear (CNL) model was calculated according to Equations \ref[error_abs] and \ref[error_rel], respectively.
%%From the \diff{} value resulted from Equations \ref[error_abs] and \ref[error_rel], it can be inferred which model had a greater uptake.
%%For instance, if \diff{}~>~0, then the solute concentration estimated by the linear model (CL) was greater than that of the non-linear model (CNL). 
%%As the only output of solute is caused by root extraction, a lower concentration indicates a higher uptake so that NLU taken up more solute for this case. 
%%The opposite is true for \diff{}~<~0, and for \diff{}~=~0 both models had the same uptake.
%
%%Figures \ref[diff_t] and \ref[diff_r] show the differences between $C$ outputs of LU and NLU as a function of time and radial distance of axial center, respectively, for scenario 1.
%%$$
%%\label[eq_diff]
%%d\!i\!f\!f\!=C\!L-C\!N\!L \eqmark
%%$$
%%%
%%where $C\!L$ and $C\!N\!L$ are the concentrations for LU and for NLU, respectively.
%When comparing differences in concentration between the two models, one has to be aware that $C\!N\!L<C\!L$ means that the uptake for NLU is greater than that of LU because, for a higher uptake, a higher amount of solute goes out from soil solution to inside the plant.
%Therefore if $d\!i\!f\!f\!<0$ then CNL > CL and LU uptake is greater; if $d\!i\!f\!f\!>0$ then $C\!N\!L<C\!L$ and NLU uptake is greater. Said that, we can see in Figure \ref[diff_t] that the uptake for NLU is greater then LU uptake at times when $C<C_{lim}$. This reflects a change also in the concentration profile for the latter times. Figure \ref[diff_r] shows the concentration profile at day 5 and the difference between $C\!L$ and $C\!N\!L$ through the profile. The higher NLU uptake increases the concentration gradient causing a higher solute flux (most diffusive since water flux is very small) from soil towards the root, resulting in a slightly higher concentration for NLU close to root surface ($d\!i\!f\!f\!.<0$).
%
%
%
%% This is the actual discussion of U test
%The Mann--Whitney U test for (Table \ref[Utest]) shows that, for scenarios 1, 2 and 4, the differences between $C\!L$ and $C\!N\!L$ are significant for concentration values at times where $C<C_{lim}$ and, for scenario 3, both models have similar results. INCOMPLETE (in development)
%%Although the differences in concentration through time and radial distance are significant
%
%\input tables/Utest
%
%Nevertheless, the difference between LU and NLU is negligible for cumulative uptake. A difference of 80.9\% of concentration over time (Figure \ref[diff_t]) corresponds to only 0.318\% in the final concentration profile (Figure \ref[diff_r]) because the uptake at times where $C<C_{lim}$ is really low. It can be seen at the cumulative uptake plot (Figure \ref[accumxt]) the insignificant effect of this difference (for both models, the cumulative solute uptake is nearly the same).
%
%Similar to Figures \ref[diff_t] and \ref[diff_r], the results of the relative accumulated error, according to Equation \ref[error_rel], were of 64.975\% and 0.739\%; 37.364.3\% and 0.041\%; 36.144\% and 0.027\% over time and distance for scenarios 2, 3 and 4, respectively.
%%STATISTICS FOR U TEST HERE
%
%%IMPROVE, POLISH
%%The Mann--Whitney U test results corroborates with the differences analysis. Assuming a confidence interval of 95\%, {\it p}-values below 0.05 indicates that the two models are different (one has greater values than the other), i.e. the hypothesis $H_0$ is rejected. Table \ref[Utest] shows that, for scenarios 2, 3 and 4, concentration $C_0$ is significantly different for LU and NLU, but when analysing the cumulative uptake (for the same period of time where $C<C_{lim}$) the difference is not significant.
%

It is important mention some issues with stabilization in the numerical solution of NLU.
%In addition, some part of the differences was due to stability problems with the numerical solution for NLU. 
The introduction of the non-linear uptake rate boundary condition Equation~\ref[eq_MM_mod] causes some numerical instability at the initial times if time and space steps are not chosen carefully.
Many different time and space step combinations had to be tested, for each scenario, until satisfying results were obtained. 
We found that, as a rule, a small time step (commonly smaller than 0.1 s) can be used when the {\tred default space step {\it QM: How much is this?}} is used. 
{\tred When a smaller space step is necessary, smaller values of $\Delta t$ are required. {\it QM: Aqui vc deveria entrar um pouco na teoria, isso tem a ver com o criterio de estabilidade de Von Neumann. Veja eq. 11 em \begtt https://en.wikipedia.org/wiki/Von_Neumann_stability_analysis \endtt Onde alfa eh a difusividade. Daria para aplicar isso no seu caso? De qq forma, mostra que, quanto menor dt, menor dx.}}
As mentioned before, decreasing the time step leads to an increase of the number of calculations requiring more computing time. 
In our simulation scenarios, the time to complete a simulation run ranged from 40 minutes to 1.5 days using a {\tred Intel Core i5 [specifications].}
%It needs to be found an optimal value for time and space step relation. 
%Stabilization problems were not found in LU.

%Since the differences between LU and NLU occurs for low concentration values and low solute flux, changes in relative transpiration are also negligible. The oscillation in the results due to the stabilization problem of the numerical solution is more likely to be noticed than the difference between the models. Figure \ref[diff_tr] shows the relative transpiration as a function of time and details the part where the oscillation occurs for each scenario. The absolute and relative differences are all due to oscillation problem, since $T_r$ turns out to be the same after complete convergence.

{\tred \citeonline[roose2009] stated that, for numerical solutions of convection-dispersion equation, the convective part might use an explicit scheme because convection, unlike diffusion, occurs only in one direction thus the solution at the following time step depends only on the values within the domain of influence of the previous time step. This set bounds on time and space steps, with a condition of stability given by ${r_0 q_0 \Delta t \over D} < \Delta r$. As the proposed model uses a fully implicit scheme, that might be the cause of the stabilization problems. 
{\it QM: Elaborar mais isso a luz do comentario no paragrafo anterior}}

%No significant difference between LU and NLU was verified in all output files (water and solute flux, concentration profile, heads and relative transpiration – Figures 1a, 2a, 3, 4, 5 and 6), but a more detailed analysis was made by analyzing the overall results (values of solute concentration for all output times -- Figure 9b) and low concentration results (values of solute concentration for output times where $C<C_{lim}$ -- Figures 8 and 9a).

%For low concentration (Figures 8 and 9a), it is noticeable that LU presents values of solute concentration higher than NLU. This is expected because in LE, solute uptake is always less than in NLU due to its linearization. Nevertheless, when the overall results are analyzed (Figure 9b), we can see that the difference is small and can be neglected. This conclusion (negligible difference) can be verified once more when analyzing the cumulative solute uptake (Figure 10). 

%The conclusion, since no significant difference between LU and NLU was found, can be either to choose LU as it takes less time to run and has no stabilization problems, or to choose NLU except for the cases in which the stability problem is significantly high. Note that this is the conclusion for those specific scenarios as the results can be significantly different for different soil and solute types.

%%%%%%%%%%%%%%%%%%%%%%%%%%%%%%%%%%%%%%%%%%%%%%%%%%%%%%%%%%%%%%%%%%%%%%%%%%%%%%%%%%%%%%%%%%%%%%%%%%%%%%%%%%%%%%%%%%%%%%%%%%%%%%%%%%%%%%%%%%%%%%%%%%%%%%%%%%%%%%%%%%%%%%%%%%%%%
%%%%%%%%%%%%%%%%%%%%%%%%%%%%%%%%%%%%%%%%%%%%%%%%%%%%%%%%%%%%%%%%%%%%%%%%%%%%%%%%%%%%%%%%%%%%%%%%%%%%%%%%%%%%%%%%%%%%%%%%%%%%%%%%%%%%%%%%%%%%%%%%%%%%%%%%%%%%%%%%%%%%%%%%%%%%%
%%%%%%%%%%%%%%%%%%%%%%%%%%%%%%%%%%%%%%%%%%%%%%%%%%%%%%%%%%%%%%%%%%%%%%%%%%%%%%%%%%%%%%%%%%%%%%%%%%%%%%%%%%%%%%%%%%%%%%%%%%%%%%%%%%%%%%%%%%%%%%%%%%%%%%%%%%%%%%%%%%%%%%%%%%%%%

\sec {\tred Model results}

{\tred{\it QM: Esse item ficou muito longo, poderia ser subdividido em subitens 5.2.1, 5.2.2, etc., para organizar mais essa discussao.}}

After analysis of solute uptake predictions by the linear (LU) and non-linear (NLU) uptake model discussed the previous section, the simulations of scenarios 1 to 8 (Table \ref[tab_scenarios]) were performed using the NLU model. 
In this section, the results of the main output variables ($C_0(t)$, $C(r)$ and $T_r(t)$) are presented for each scenario, as well as the predicted active and passive contributions to solute uptake.
Simulation results for these scenarios are shown in Figures \ref[fig_C_t]
to
%, \ref[fig_C_r] and 
\ref[fig_Tr_t]. 
{\tred All simulations started at pressure head ($h_{ini}$) of $-1$~m {\it QM: Isso deve constar tb do M\&M, nao sei se consta}}, so that both wet and dry soil conditions occurred in the simulated period.

%The general model predictions can be done by analysing scenario 1, chosen to be the default scenario (continuous line in Figures \ref[fig_C_t], \ref[fig_C_r] and \ref[fig_Tr_t]).
According to the piecewise Equation \ref[eq_MM_mod], the solute uptake rate occurs according to three phases. 
They will be abbreviated LUP -- linear uptake phase when $C_0>C_2$, CUP -- constant uptake phase when $C_{lim}<C_0<C_2$, and NUP -- non-linear uptake phase when $C_0<C_{lim}$.
{\tred For the simulation scenarios, the set of input parameters $h_{ini}$ and $C_{ini}$ were chosen for the simulations to start always starts at LUP, when relative transpiration  $T_r=1$ and solute uptake occurs by mass flow of the solution only (passive uptake).{\it QM: Isso tambem parece mais M\&M. Nao tem problema repetir aqui, mas deve estar no M\&M tb.}}
In this phase, the water content and concentration in bulk soil deplete at the same rate, resulting in a period of constant solute concentration.
Results from these simulations are shown in Figure \ref[fig_C_t] for scenarios involving different textures (different soil hydraulic properties), different root length densities, different initial solute concentrations and different potential transpiration rates.
For different soil types, although the initial concentration in bulk soil is the same (10 mol m$^{-3}$), initially, the predicted concentrations at the root surface are different, as shown in Figure \ref[fig_C_t](a). 
This occurs due to the fact that the initial condition refers to a fixed pressure head, resulting in different water contents for each soil in accordance to their hydraulic parameters.

\medskip
\label[fig_C_t]
\picw=17cm \cinspic C_t.pdf
\caption/f {{\tred Solute concentration in soil water as a function of time predicted by the non-linear uptake model (NLU) for scenarios differing in (a) soil hydraulic parameters (according to texture class), (b) root length density, (c) initial solute concentration, and (d) potential transpiration rate {\it QM: Essa figura ficaria melhor colocando a mesma escala em todos os 4 graficos}}}
\medskip

{\tred The constant uptake phase (CUP) is characterized by the depletion of $C_0$ at a constant rate, which starts some time after the onset of the $T_r$ falling rate phase.{\it QM: Isso nao da para entender assim. Da para ver isso na figura? Deve explicar melhor. Talvez mais uma figura?Pois nessa fig. 9 da para ver quando comeca o falling rate de Tr?}}
%, when the total head reaches its limit value $H_{lim}$. 
The reduction in $q_0$ (reflected in $T_r$) increases $C_2$ value according to Equation \ref[eq_c2], therefore, the time that CUP initiates depends on the rate that $q_0$ decreases,
%According to the results showed in Figure \ref[fig_C_t], 
%the rate of decreasing $q_0$ is modified by
which changes with soil type, root density, initial concentration and potential transpiration (Figure \ref[fig_C_t]).
%(WE CAN SEE IT IN THE DIFFERENT SOIL TYPES)
{\tred Moreover, during the CUP, the solute uptake by mass flow does not attend completely the plant demand, {\it QM: Como da para ver isso na figura?}}
thus, according to the model assumptions, an active uptake is developed and it becomes higher as passive uptake 
%(driven by mass flow of water) 
becomes lower in order to maintain the constant uptake rate.
This constant demand, filled by active and passive uptake, is maintained until $C_0$ reaches $C_{lim}$ which, in turn, increases as $q_0$ decreases, according to Equation \ref[eq_clim].
Active uptake reaches its maximum value when $C_0=C_{lim}$  and the uptake is then limited by the concentration in the soil in further time steps, starting the non-linear uptake phase (NUP).
{\tred It can be seen in Figure \ref[fig_contrib_t] {\it QM: Separar melhor a discussao das figuras. Assim fica pouco claro. Discuta primeiro a fig. 9, depois a 10; ou as duas juntas. Essa parte nao ficou claro.}} that the time of the occurrence of the maximum value for active uptake depends on how fast passive uptake 
%(caused by the depletion of $C_0$ and reduction of $q_0$) 
decreases, which is 
%Passive uptake decreases 
as fast as $q_0$ and $C_0$ decreases depending on the rate of solute uptake \im{} (according to Equation \ref[eq_MM_mod] and model assumptions).
Therefore, the soil type, root density, initial concentration, and potential transpiration also affects the time for the maximum active uptake value and, consequently, the time at which NUP starts.

\def\desca{Bold line represents the total solute uptake (active+passive), the thin line represents active contributions to the solute uptake, and the dashed line represents the passive contribution. }
\def\descb{Simulated scenarios according to Table \ref[tab_scenarios]}

\medskip
\label[fig_contrib_t]
\picw=17cm \cinspic all_contributionsxt.pdf
\caption/f {Active and passive contributions to the solute uptake as a function of time predicted by NLU. \desca \descb}
\medskip

\medskip
\label[fig_contrib_c]
\picw=17cm \cinspic all_contributionsxc.pdf
\caption/f {Active, passive and 
{\tred total {\it QM: Sugestao: na figura, escrever active, passive, e total (e nao: qs0). O eixo-Y, retirar qs0 da descricao, qs0 e somente uma das tres linhas}} 
uptake ($q_{s_0}$) as a function of solute concentration. \desca \descb}
\medskip

An important feature of the non-linear uptake phase (NUP) is a second falling rate phase for $T_r$ (as shown in Figure \ref[diff_tr]). 
The total hydraulic head $H$ is not allowed to decrease below  $H_{lim}$ and, as the solute uptake rate ($q_{s_0}$) decreases due to the limitation imposed by $C_{lim}$ (Equation~\ref[eq_MM_mod]), a decrease in the water uptake rate ($q_0$) is predicted in order to maintain $H$ at its limiting value.
This stage is of very short duration as it starts with very low values of $q_0$ and $T_r$, near the end simulations at $T_r=0.001$, when $q_0$ is considered negligible.
In this phase, however, the active component of uptake dominates, being the major component of the solute uptake. 
%According to \citeonline[tinker], at low concentrations, the ion transport across membranes is limited by the diffusive flux of solute towards the membrane in case of high affinity uptake 
%(low $K_m$ values) 
%and the contribution of the solute uptake by mass flow of water (through the transpiration stream) is estimated to be very low, often below 0.1\%, which corroborates the results.

Predictions of active and passive contributions as a function of time and concentration in soil solution are shown respectively in Figures \ref[fig_contrib_t] and \ref[fig_contrib_c], according to Equation \ref[eq_MM_mod] and the model assumptions.
From the onset of simulations (wet condition, $T_r=1$), although scenarios start in LUP, a constant (instead of a linearly decreasing) uptake rate is developed. 
This is due to the fact that the model considers a constant potential transpiration rate over time, therefore, the solute uptake is determined by (constant) water flow $q_0$ and solute concentration $C_0$ (Equation \ref[eq_case3]), also constant because passive uptake by mass flow is the dominating process.
A plant controlled process allowing the reduction of solute uptake (for example, a {\tred reflection coefficient {\it QM: Acho que voce ainda nao introduziu/mencionou esse termo, poderia explicar um pouco mais. Ou citar algum trabalho que faz isso}}) might be a modeling alternative to describe controlled uptake in this phase.
In our model such a feature was not included and, according to model assumptions, the rate of solute uptake at LUP is constant and its value depends on the potential transpiration, soil hydraulic properties, root length density, and initial solute concentration.
It does not depend on ion type, since $q_{s_0}$ values were equally predicted (Figure \ref[fig_contrib_t]).

%The length of the two subsequent phases (CUP and NUP)
Different predictions for the active and passive contributions for solute uptake as a function of solute concentration are shown in Figure \ref[fig_contrib_c].
Modifying $T_p$ from the low to the high rate cause a change in the length of the phases (LUP, CUP and NUP), but no alteration in the active and passive partitioning was predicted.
This result is straightforward since the potential transpiration affects the rate of the potential water flux and water uptake which, on their turn, affect the rate of solute uptake.
Keeping soil hydraulic properties, root length density, ion type and solute uptake parameters the same, similar results are expected, although on a different time scale.
The most significant alteration in the partitioning between active and passive uptake is predicted when changing hydraulic properties (especially for the clay soil) and root length density values (Figure \ref[fig_contrib_c]).

In the clay soil, {\tred the water flux is smaller due to its hydraulic properties {\it QM: Entao a Tr nesse solo tb eh menor? Explicar melhor isso.}} causing a small solute transport and uptake by mass flow. 
The diffusive transport in the root surface (active uptake), quickly becomes dominant (Figure \ref[fig_contrib_c], scenario 7).
Consequently, $C_0$ is slowly depleted in CUP whereas $\theta$ rapidly reduces to very low values, causing a fast reduction of $q_0$ and $T_r$.
When $T_r\approx0$ (end of simulation), 
{\tred $C_0$ equals 17.88~mol~m$^{-3}$ and is higher than $C_{lim}$ 
%(5.97~mol~m$^{-3}$) 
(Figure \ref[fig_C_r](a)), and the active contribution ends in its maximum value  and the NUP does not occur {\it QM: Nao muito facil para entender}} (Figures \ref[fig_contrib_t] and \ref[fig_contrib_c], scenario 7).

At high root densities, more water is extracted from the same soil volume during a given time period, causing a delay in the onset of $T_r$ reduction. 
The higher {\tred root length density results in a steeper falling rate phase {\it QM: Isso esta em um dos meus papers tb (2006? 2008? , que poderia ser citado aqui}} (Figure \ref[fig_Tr_t]). 
Therefore, active uptake becomes more important with increasing $R$, as can be seen in Figure \ref[fig_contrib_c] (scenarios 1, 4 and 5).

The final concentration profile and relative transpiration are important modeling results. 
The former is essential to determine the gradient of solute concentration and fluxes. 
Additionally, it can be used to compute the average solute concentration in a soil layer, important for model upscaling since it is required for one-dimensional macroscopic models.
The latter is used to predict water and osmotic stress, which can be related to biomass accumulation and yield predictions.
{\tred By default, the concentration profile is set to be stored in a file at the end of each simulated day and in the end of simulation, but the user can select any time interval to it. {\it QM: Isso nao eh cientificamente relevante.}}
Figure \ref[fig_C_r] shows that, at the end of simulations, the concentration profiles and profile average solute concentration are affected by all selected parameters, except potential transpiration.
As expected, a change in $T_p$ causes changes only in the time at which $C_0$ reaches limiting values ($C_2$ and \clim) due to the decrease in water and solute fluxes. 
Corroborating with these results, Figure \ref[fig_contrib_c] shows the same amount of active and passive uptake contributions, according to the actual concentration in the soil, for both scenarios 1 ($T_p=6$ mm) and 2 ($T_p=3$ mm),
whereas the $T_r$ falling rate phase onsets earlier when $T_p$ is increased (Figure \ref[fig_Tr_t]).
In other words, these simulations show that a reduction of $T_p$ results in the same predicted solute concentration profiles, but at {\tred a different time scale {\it QM: Provavelmente, se voce expressar o resultado em funcao do (tempo x Tp) as curvas ficarao igual?}}.

Soil hydraulic properties and root length density change the final concentration profile respectively due both to the different soil hydraulic properties that affect water movement and to the greater water and solute uptake per soil volume unit (see earlier discussion).
Similarly, they affect the steepness of the $T_r$ falling rate phase, also modifying the length of the phases LUP, CUP and NUP.
The initial concentration, for obvious reasons, changes the final concentration profile as well.
It slightly changes the onset of falling {\tred $T_r$ rate phase due to the osmotic hydraulic head component. {\it QM: Aqui poderia citar meu 2010 paper.}}
The highest variation can be observed in CUP and NUP.
With low values of $C_0$ ($h_\pi$ closer to zero), $H_{lim}$ is achieved faster because there is less solute in soil to be taken up.
This causes a steeper $T_r$ reduction curve, shortening the time needed to $C_0$ be smaller than \clim.

\medskip
\label[fig_C_r]
\picw=17cm \cinspic C_r.pdf
\caption/f {Solute concentration in soil water as a function of distance from axial center, at the end of simulations, for different scenarios regarding (a) hydraulic properties, (b) root length density, (c) initial solute concentration, and (d) potencial transpiration. Thin lines indicates the average concentration}
\medskip

\medskip
\label[fig_Tr_t]
\picw=17cm \cinspic Tr_t.pdf
\caption/f {Relative transpiration as a function of time, predicted for different scenarios referring to (a) hydraulic properties, (b) root length density, (c) initial solute concentration, and (d) potential transpiration}
\medskip

%The $C_0$ value at the end of simulation is never absolutely zero due to the asymptotic behaviour of $q_{s_0}$ function in response to decreasing $C_0$ values (Equation \ref[eq_case1]). 
%Even so, $C_0$ becomes so close to zero that it can be consider zero and, consequently, solute uptake also zero.

Figures \ref[fig_contrib_t] and \ref[fig_contrib_c], referring to scenarios 1 to 7, show that with the same values of $I_m$ and $K_m$ (same solute), different sets of active and passive contributions can be observed just due to changes in soil, root and atmospheric parameters. 
{\tred All of these changes occurred in drying out simulations, with constant atmospheric demand. {\it QM: Esse tipo de observacao confunde. Todas as suas simulacoes nao se referem a esse tipo de cenario?}}
{\tred 
Different than simulated scenarios, real systems soil water contents changes due to rain or irrigation, and the solute concentration may be altered by the addition of nutrients to the soil.
The atmospheric demand is far to be constant. {\it QM: Esse tipo de coisa poderia ser discutido numa secao especifica, onde discute a relevancia dos cenarios simulados comparados com a realidade}}
The variation of all model parameters due to the natural changes that occur in a real system would generate a wide range of different responses of root uptake and its passive and active contributions, being hard to be predicted even with a more complex model.
This reveals the system complexity to be simulated and how much is still missing to come to a more complete understanding of the dynamics of these phenomena.

%
{\tred 
The partitioning of the solute uptake in active and passive processes, as simulated by the model, may increase insight and lead to further studies. 
It reveals about the metabolic partitioning of energy used by the plant that regulates solute uptake, which would be useful to predict its stress status and separates osmotic and ionic stressors.
To improve results, the uptake model should be coupled to a more complex plant model, which deals with metabolic processes as a function of the solute concentration within plant cells (nutritional status) and time.
Separating osmotic and ionic stress would lead to more detailed predictions of crop stress, growth and yield.
Besides, it also could be useful to water and nutritional management.
{\it QM: Subitem ``Suggestions for further results''?}}
%Obviously, the model assumptions could be more complex and, as is known, the assumptions are extremelly important to the model predictions.
%Poor assumptions leads to poor predictions and a complex system like this in study (SPA) are often far from reality.
%Luckily, several computational/physical/mathematical models are being developed, each one with its own characteristics and limitations, in order to best approximate nature and its phenomenon.


{\tred Concluding, the model was shown to be able to quantify the active and passive contributions to the solute uptake, which can be used to distinguish osmotic and ionic stressors in further works.
Soil hydraulic properties, root length density, initial concentration and potential transpiration are factors that change the time that the concentration at the root surface starts to decrease and the time that the active uptake is maximum. {\it QM: OK, mas isso poderia simplesmente estar na secao ``conclusions''}}


%%%%%%%%%%%%%%%%%%%%%%%%%%%%%%%%%%%%%%%%%%%%%%%%%%%%%%%%%%%%%%%%%%%%%%%%%%%%%%%%%%%%%%%%%%%%%%%%%%%%%%%%%%%%%%%%%%%%%%%%%%%%%%%%%%%%%%%%%%%%%%%%%%%%%%%%%%%%%%%%%%%%%%%%%%%%%
%%%%%%%%%%%%%%%%%%%%%%%%%%%%%%%%%%%%%%%%%%%%%%%%%%%%%%%%%%%%%%%%%%%%%%%%%%%%%%%%%%%%%%%%%%%%%%%%%%%%%%%%%%%%%%%%%%%%%%%%%%%%%%%%%%%%%%%%%%%%%%%%%%%%%%%%%%%%%%%%%%%%%%%%%%%%%
%%%%%%%%%%%%%%%%%%%%%%%%%%%%%%%%%%%%%%%%%%%%%%%%%%%%%%%%%%%%%%%%%%%%%%%%%%%%%%%%%%%%%%%%%%%%%%%%%%%%%%%%%%%%%%%%%%%%%%%%%%%%%%%%%%%%%%%%%%%%%%%%%%%%%%%%%%%%%%%%%%%%%%%%%%%%%

\sec Sensitivity analysis 

To better analyze the impact of the variation of the model input parameters, a sensitivity analysis is presented in this section.

The relative partial sensitivity $\eta$ is the fraction of the relative change in parameter $P$ that will propagate in the prediction $Y$ \cite[quirijn_sens].
A value of $\eta=0.5$ indicates that a 1\% increase of $P$ results in 0.5\% increase in the prediction $Y$.
The value of $\eta$ found through a variation of a parameter is likely to change with variation of another parameter, 
%Thus, it is profoundly difficult to make a sensitivity analysis that deals with all combinations of parameters variation of a complex model. 
making a sensitivity analysis that deals with all combinations of parameters variation of a complex model being profoundly difficult to be done. 
Therefore, the analyses are commonly restricted to a fixed set of parameters. 

Despite the solution to the convection-dispersion equation be the concentration in soil solution as a function of time and distance ($C(r,t)$), the model can give a wide range of another predictions.
For instance: relative transpiration, pressure and osmotic potentials, water and solute uptake rates, and accumulated solute uptake.
Some of these predictions were selected to check how changes in soil and ion/plant parameters would affect them. 
The selected predictions $Y$ were end time of simulation ($t_{end}$), osmotic head ($h_{os}$), time at the onset of the $T_r$ falling rate phase ($t_{T_r}$), time at the onset of constant uptake phase ($t_{C_0}$) and accumulated solute uptake ($acc$);
and $P$ were the ion/plant ($I_m$ and $K_m$) and the soil hydraulic ($\alpha$, $n$, $K_s$, $\lambda$, $\theta_r$ and $\theta_s$) parameters. 

Sensitivities to ion/plant parameters were high for $h_{os}$ and presented an inverse response (negative values) to $I_m$ variation (Figure \ref[sensitivity_MM_rest]).
This is due to the fact that with higher values of $I_m$ there are higher uptake rates of solute in both phases of constant and non-linear uptake, diminishing the value of concentration at root surface in the end of simulation (less negative values of osmotic head $h_\pi$).
On the contrary, higher values of $K_m$ make higher values of $C_{lim}$ that causes an increase of the non-linear uptake phase and a decrease in the constant uptake phase.
With less time of potential solute uptake, the final concentration becomes higher due to a lower amount of extracted solute.   
Thus, in general, a variation of 1\% in $K_m$ and $I_m$ cause, respectively, a decrease of 1.2\% and a increase of 1\%, approximately, in $h_{os}$. 
The exception was scenario 7 that showed low $\eta$ values for both parameters.
The reason is the small variation in $C_0$ during the simulation due to the low water and solute fluxes of the clay soil, explained in last section.

\medskip
\label[sensitivity_MM_rest]
\picw=15cm \cinspic final_everybar_MM_rest.pdf
\caption/f {Relative partial sensitivity of osmotic head at root surface in the end of simulation ($h_{os}$) and accumulated solute uptake ($acc$) to MM equation parameters $I_m$ and $K_m$ for scenarios 1 to 7}
\medskip

Figure \ref[sensitivity_MM_rest] shows that, despite the sensitivity of $h_{os}$, $acc$ presented very low (close to zero) sensitivity to $K_m$ and $I_m$.
This is due to the fact that $I_m$ and $K_m$ just effectively act in the solute uptake rate in the constant and non-linear uptake phases, since all the uptake in the linear uptake phase is due to mass flow of water. 
The period of constant and non-linear phases are relatively short when compared with the linear phase period, causing a low influence of $I_m$ and $K_m$ in the accumulated solute uptake. 

Figure \ref[sensitivity_MM_time] shows low sensitivity of the time related predictions to $I_m$ and no sensitivity to $K_m$.
As the model deals with constant potential transpiration only, the time at the onset of the $T_r$ falling rate phase is the exact time when $q_0$ decreases.
It is a phase that $I_m$ and $K_m$ does not exerts any influence, which explains the zero sensitivity of $t_{T_r}$ to those parameters. 

\medskip
\label[sensitivity_MM_time]
\picw=15cm \cinspic final_everybar_MM_time.pdf
\caption/f {Relative partial sensitivity of end time of simulation ($t_{end}$), time at the onset of the $T_r$ falling rate phase ($t_{T_r}$) and time at the onset of constant uptake phase ($t_{C_0}$) to MM equation parameters $I_m$ and $K_m$ for scenarios 1 to 7}
\medskip

The predictions for $t_{end}$ and $t_{C_0}$ presented low sensitivity to $I_m$ and no sensitivity to $K_m$.
$I_m$ is directly related to the solute uptake rate (as it defines the plant demand) and $K_m$ to the $C_{lim}$ value. 
Thus, higher values of $I_m$ causes higher uptake rates, faster decrease of water and solute contents and, therefore, a faster $T_r$ falling rate phase resulting in a sooner end of simulation.
%The no influence of $K_m$ over the time predictions is due to... EXPLAIN
%It is hard to be analyzed in another way than empirically testing.

Figures \ref[sensitivity_soil_time], \ref[sensitivity_soil_rest] and \ref[sensitivity_n] show the sensitivities to the soil hydraulic properties.
%It is hard to discuss the sensitivities of the model outputs to each parameter because of the great number of parameters and outputs being tested.
Although the sensitivity results are presented separately for each parameter, output and scenario, the interactions between all parameters and simulated processes occurs sinergetically and concomitantly.
Due to the relative nature of the sensitivity (there is no absolute value of $\eta$ for one prediction with a variation of different parameters), an individual analysis can give an insight of the importance of a parameter to the model, but it truthly serves only for that particular simulation.
Therefore, due to the high number of soil hydraulic parameters, the discussion will focus only on those with the greatest $\eta$ values.
A more detailed and profound analysis with these sensitivity results can be done in the future, with empirical tests that might corroborate the results.

The parameter $n$ showed the highest (absolute) values of $\eta$ among the soil hydraulic properties, mainly for $h_{os}$ in the end of simulation, reaching the value of $-$23 for scenario 2 (Figure~\ref[sensitivity_n]).
Similar result was found by \citeonline[quirijn_sens], where the yield predictions of the evaluated model had higher sensitivity for the $n$ parameter.
%, also with one order of magnitude higher.
Figure \ref[fig_wrcs](a) shows how the water retention curve, following the \citeonline[genuchten80] model, changes with the variation of 1\% in $n$ parameter.
It shows that the same value of water content is reached with higher (more negative) values of pressure head and it affects both water retention and hydraulic conductivity. 
An increase in this parameter causes a more steep decrease in those hydraulic functions.
Increasing $n$ increases the solute uptake rate because the same initial pressure head originates lower values of water content and, consequently, higher solute concentration values.
Therefore, due to the higher solute uptake rate, the limiting value $H_{lim}$ is reached at later times, elongating LUP and resulting in less negative values of solute concentration in soil water ($h_\pi$), which is what occurs in the simulations with loam soil (scenarios 1,2,3,4 and 5).
Nevertheless, despite the higher change in the water retention curve for the clay soil does not affect the $\eta$ value of scenario 7 in the same proportion.
%(its variation parameter is the soil type: clay).
Due to the interactions between water retention and hydraulic conductivity curves, the overall effect is hard to be properly analyzed without an empirical test.
However, the cause for the lower value of $\eta$ in scenario 7 might reside on the fact that the solute uptake rate in this scenario is very low and, even if the variation of 1\% in $n$ causes a bigger change on the retention curve (and consequently in the concentration values), it does not strongly affect the solute uptake rate and, consequently, the change in $h_\pi$ (or $h_{os}$) at the end of simulation is small.

\medskip
\label[sensitivity_n]
\picw=15cm \cinspic final_large.pdf
\caption/f {Relative partial sensitivity of end time of simulation ($t_{end}$), osmotic head ($h_{os}$), time at the onset of the $T_r$ falling rate phase ($t_{T_r}$), time at the onset of constant uptake phase ($t_{C_0}$) and accumulated solute uptake ($acc$) to soil hydraulic parameter $n$ for scenarios 1 to 7}
\medskip


For the time selected predictions ($t_{end}$, $t_{T_r}$ and $t_{C_0}$), the parameters $\theta_s$ and $\lambda$ presented the highest (absolute) $\eta$ values.
Increasing values of $\theta_s$ increases the available water content and increases at about the same proportion ($\approx$1\%) $t_{T_r}$, $t_{end}$ and $t_{C_0}$ due to the highest opportunity of water uptake with the same rate of $T_p$.
The parameter of the hydraulic conductivity function $\lambda$ causes an inverse response in $t_{end}$ prediction.
Higher values of $\lambda$ causes higher hydraulic conductivities with the same value of water content, which increases the water flux and uptake and decreases the time that the root surface reaches the limiting value $H_{lim}$, reducing the time of $T_r$ and $C_0$ falling phases, consequently reducing $t_{end}$.
Thus, sensitivities to $\lambda$ and $\theta_s$ are expected to be negative and positive, respectively, for the selected time predictions, corroborating the results (Figure \ref[sensitivity_soil_time]).
The higher sensitivity to $\theta_s$ can be explained due to the fact that the model has a powerful (possibly overestimated) solute extraction at LUP, either due to the lack of a plant regulated process for solute uptake at this phase or because the model does not consider a regulation in the solute uptake according to solute concentration inside the plant.

%The model sensitivity for the parameter $\theta_r$ was the lowest in the selected time predictions.
%An increase in $\theta_r$ decreases the available water content decreasing the time of $T_r$ and $C_0$ falling phases, consequently reducing $t_{end}$. 
%%It also reduces the minimum value of $h$ (becomes more negative) increasing the maximum value of $h_\pi$ in the limiting total head ($H=-150$) m.
%Thus, sensitivities to $\theta_r$ and $\theta_s$ are expected to be negative and positive, respectively, for the selected time predictions, corroborating the results (Figure \ref[sensitivity_soil_time]).
%%Depending on the potential transpiration, initial concentration and other parameters combinations, $h$
%In a general way, the selected time predictions showed to be much more sensitive to $\theta_s$ (disregarding $n$), being $\lambda$ the second in some scenarios.
%A possible explanation would be that the model has a powerful (possibly overestimated) solute extraction at LUP, due to the lack of a plant regulation process for the uptake.
%In that way, an increase in $\theta_s$ is more significant than an equal increase in $\theta_r$, as for low water contents the water uptake and solute transport by mass flow are smaller.

%{\localcolor\Red
%The variation of $\alpha$ parameter caused more changes in $t_{end}$ and $h_{os}$ model predictions.
%
%$\lambda$ and $K_s$ presented low capacity of alter the predictions, when compared to the other parameters.
%
%The accumulated solute uptake presented low sensitivity to all tested parameters, with exception of $n$.
%}

\medskip
\label[sensitivity_soil_time]
\picw=15cm \cinspic final_every_bar_soil_time.pdf
\caption/f {Relative partial sensitivity of end time of simulation ($t_{end}$), time at the onset of the $T_r$ falling rate phase ($t_{T_r}$) and time at the onset of constant uptake phase ($t_{C_0}$) to soil hydraulic parameters $\theta_r$, $\theta_s$, $\alpha$, $K_s$ and $\lambda$ for scenarios 1 to 7}
\medskip

For the solute related predictions,
%($h_{os}$ and $acc$),
$h_{os}$ showed high sensitivities to $\theta_r$, $\theta_s$ and $\alpha$ (Figure \ref[sensitivity_soil_rest]).
The sign of sensitivity of $h_{os}$ to $\theta_r$ and $\theta_s$ is straightforward.
As explained before, an increase in $\theta_r$ decreases the available water content, shortening the duration of LUP phase and $t_{T_r}$.
As a result, the period of highest solute uptake rate is reduced causing, in the end of simulation, a higher solute concentration at the root surface (more negative $h_{os}$).
On the other hand, an increase in $\theta_s$ causes a increase in the available water content, elongating LUP which reflects in lower solute concentration at the root surface (less negative $h_{os}$).
As the change in the retention curve is similar to the variation of $n$, the effects are also similar.
The end of simulation presents less negative values of $h_\pi$ due to the higher solute uptake rate.
The parameter $\alpha$ affects the water retention curve by changing the air entry pressure head.
An increase in its value makes that the same water content corresponds to a less negative pressure head (Figure \ref[fig_wrcs](b)).

Accumulated solute uptake at the end of simulation ($acc$) presented the lowest sensitivity to all parameters for almost all scenarios. 
As the majority of the solute uptake is done in LUP (first phase, where $C > C_2$), $acc$ is more affected when the duration of this phase is altered.
The sensitivity of $t_{T_r}$ serves as an indicator to the shortening/elongation of this phase, in a way that $acc$ responds more strongly when $t_{T_r}$ is also affected.
%, as Figures \ref[sensitivity_soil_time] to \ref[sensitivity_n] show.



\medskip
\label[sensitivity_soil_rest]
\picw=15cm \cinspic final_every_bar_soil_rest.pdf
\caption/f {Relative partial sensitivity of osmotic head at root surface in the end of simulation ($h_{os}$) and accumulated solute uptake ($acc$) to soil hydraulic parameters $\theta_r$, $\theta_s$, $\alpha$, $K_s$ and $\lambda$ for scenarios 1 to 7}
\medskip

\medskip
\label[fig_wrcs]
%\picw=12cm \cinspic wrcs.pdf
\picw=17cm \cinspic wrcs_changes.pdf
%\caption/f {Theoretical water retention curves, according to Equation \ref[eq_theta], for the selected soils from Table \ref[soils]. The lower value of $h$ matches the initial value for all simulations $h_{ini}=-1$~m. Dashed lines represent the variation of 1\% in $n$ parameter}
\caption/f {
Theoretical water retention curves, according to Equation \ref[eq_theta], 
(a) for the selected soils from Table \ref[soils] with dashed lines representing a variation of +1\% in parameter $n$; and
(b) for the loam soil with dashed line representing a variation of +50\% in parameter $\alpha$;
%(c) hydraulic conductivity function, according to Equation \ref[eq_K], for the loam soil with dashed line representing a variation of +50\% in parameter $\lambda$.
Values of $h$ ranges from the initial value for all simulations $h_{ini}=-1$~m to $-$150 m}
\medskip

With this simple sensitivity analysis, we can conclude that the parameters that need more carefulness to measure are: $\theta_r$, $\theta_s$, $\alpha$, $I_m$ and $K_m$ which affects more strongly the solute concentration at root surface in the end of simulation, $\theta_s$ which strongly affects the time at the limiting values are reached, and $n$ which strongly affects all selected predictions, mainly $h_{os}$.



{\tred
%Note that we could have choose one scenario to do the sensitivity analysis. 
%In that way we would have not analyse a wide range of possibilities of sensitivity in the model predictions.
%The same prediction can have positive or negative sensitivity to a variation of a parameter depending on the chosen scenario.
%In other words, it is proven that the value of $\eta$ calculated through a variation of a parameter can change with the variation of another parameter and shows the complexity of the model and, therefore, of its sensitivity analysis.
}

%%%%%%%%%%%%%%%%%%%%%%%%%%%%%%%%%%%%%%%%%%%%%%%%%%%%%%%%%%%%%%%%%%%%%%%%%%%%%%%%%%%%%%%%%%%%%%%%%%%%%%%%%%%%%%%%%%%%%%%%%%%%%%%%%%%%%%%%%%%%%%%%%%%%%%%%%%%%%%%%%%%%%%%%%%%%%
%%%%%%%%%%%%%%%%%%%%%%%%%%%%%%%%%%%%%%%%%%%%%%%%%%%%%%%%%%%%%%%%%%%%%%%%%%%%%%%%%%%%%%%%%%%%%%%%%%%%%%%%%%%%%%%%%%%%%%%%%%%%%%%%%%%%%%%%%%%%%%%%%%%%%%%%%%%%%%%%%%%%%%%%%%%%%
%%%%%%%%%%%%%%%%%%%%%%%%%%%%%%%%%%%%%%%%%%%%%%%%%%%%%%%%%%%%%%%%%%%%%%%%%%%%%%%%%%%%%%%%%%%%%%%%%%%%%%%%%%%%%%%%%%%%%%%%%%%%%%%%%%%%%%%%%%%%%%%%%%%%%%%%%%%%%%%%%%%%%%%%%%%%%

\sec Solute uptake models comparison

%The incorporation of solute uptake in the model of \citeonline[liersolute] shows to be very complex according to the results that has been shown. 
%The differences are shown in this section, together with the constant uptake model developed by \citeonline[willigen1] to 
%{\localcolor\Red SECTION INTRODUCTION}

As the model developed in this thesis will serve as an addition to the model of \citeonline[liersolute], in which does not considers solute uptake, a comparisson between the two models is necessary.
Additionally, the results are compared with a model of constant solute uptake (based on the work of \citeonline[willigen1]) to show the differences between constant and zero uptake approaches.

Figure \ref[fig_C0] shows the differences in the solute concentration at root surface as a function of time for the three different models ZU, CU and NLU. 
In ZU, salt is transported to the roots by convection, causing an accumulation of solutes at root surface. 
As water flux towards the root starts to decrease, salt is transported slower and carried away from the roots by diffusion. 
CU presents the same behaviour with a slightly lower $C_0$ value, due to the small solute uptake value. 
%The difference between ZU and CU concentrations increases as the water flux towards the root decreases, making the transport by diffusion away from the root becomes more important than by convection towards the roots.
As higher is the value of the constant uptake rate and as lower is $T_p$ in CU, less solute will accumulate in the root surface and more similar CU and NLU will become, as can be seen in the gray line of Figure \ref[fig_C0].

\def\descc{Gray line is CU with the constant of solute uptake rate 10 times higher}

\medskip
\label[fig_C0]
%\picw=11cm \cinspic cxt1.pdf
\picw=13cm \cinspic cxt_comp.pdf
\caption/f {Soil solution concentration at root surface as a function of time for constant (CU), zero (ZU) and non-linear (NLU) uptake models. \descc}
\medskip

For ZU and CU, $C_0$ 
%the concentration 
starts to diminish (less negative values of $h_\pi$) as soon as $T_r<1$, because $H$ at root surface reached the limiting value of $-$150 m (Figure~\ref[fig_heads1]).
%due to $H_0=-150$ m (Figure \ref[fig_heads1]).
In CU, as the solute uptake does not depend on the solute concentration in soil (it is a constant), the solute transport by convection becomes less important than the movement by diffusion as the water flux towards the root diminishes.
Thus, the solute depletion at the root surface is due both to the uptake and to the diffusive transport away from the roots.
For the case of NLU (Figure \ref[fig_heads2]), an accumulation period is not developed due to the high solute uptake rate in LUP (as can be seen in Figure \ref[fig_qst_detail], all solute carried by mass flow of water is taken up).
The time needed to $C_2$ increases until it becomes higher than $C_0$ (explained in the last section) generates a delay between the onset of the $T_r$ falling rate phase and the time that $C_0$ starts to diminish.
Moreover, the higher uptake of NLU causes lower values of osmotic head ($h_\pi$) and generates a longer period of potential transpiration rate ($T_r=1$).
For the case of CU and ZU, $T_r$ starts to diminish around the 21th day whilst for NLU it is about the 24th day.

\medskip
\label[fig_heads1]
\picw=13cm \cinspic headsxt.pdf
\caption/f {Pressure ($h$), osmotic ($h_\pi$) and total ($H$) heads at the root surface as a function of time for constant (CU) and zero (ZU) uptake models, for scenario 1}
\medskip

\medskip
\label[fig_heads2]
\picw=13cm \cinspic headsNLxt.pdf
\caption/f {Pressure ($h$), osmotic ($h_\pi$) and total ($H$) heads at the root surface as a function of time for non-linear (NLU) uptake model, for scenario 1}
\medskip

Figure \ref[fig_qst_detail] shows the solute uptake rate (solute flux at root surface $q_{s_0}$) as a function of time for each model.
It shows the three uptake phases for NLU (LUP, CUP and NUP), the constant uptake rate for CU, and the zero uptake rate for ZU, according to their respective equations.
Once more, it is noticeable that at the first uptake phase (LUP) the solute uptake rate is higher than the other two phases due to the constant potential transpirations rate and the lack of a plant controlled mechanism of solute uptake regulation.
Nevertheless, this result is compatible with the model assumption 1, that the plant does not control the solute taken up by mass flow of water.
This result must be compared with experimental data of solute uptake of non-stressed plants to elucidate whether a plant regulated mechanism of uptake must be added to the model. 
 
\medskip
\label[fig_qst_detail]
\picw=11cm \cinspic qsxt1_detail.pdf
\caption/f {Detail of solute flux at root surface as a function of time for constant (CU), zero (ZU) and non-linear (NLU) uptake models, for scenario 1}
\medskip


The result of the total amount of solute extracted by the plant is shown in Figure \ref[fig_accsol].
It can be understand as the amount of solute accumulated inside the plant.
The high uptake rate at LUP, in NLU model, causes a fast accumulation of solute inside the plant due to passive uptake.
According to the model assumptions, the plant does not regulate passive uptake (or the uptake caused by mass flow of water), which, for the case of NLU model in all simulated scenarios, is the greatest contribution to the uptake.
The ZU model obviously does not accumulate any solute and LU model accumulates it linearly, at a constant rate.

\medskip
\label[fig_accsol]
\picw=11cm \cinspic accumxt_models.pdf
\caption/f {Cumulative solute uptake as a function of time for constant (CU), zero (ZU) and non-linear (NLU) uptake models, for scenario 1}
\medskip

%{\tred
%Based on the assumption that the uptake is not regulated by a concentration inside the plant
%, it can be plausibly inferred that the governing mechanism of the toxic effect is the passive uptake.
%}
%{\tred
%If the concentration inside the plant is already at its limiting value (meaning that, for that specific time period, the plant does not need that solute anymore), it can stop the active mechanism (as it is plant regulated by metabolic processes) but it is not allowed to stop the passive process, that would be the closure of stomata to stop the water uptake.
%The plant then has to choose what is physiologically more positive to it: either close the stomata to not enter in a toxicity process or extract more solute and be affected by the toxic effects of the solute excess inside the cells. 
%Therefore, as a further model improvement, the solute concentration inside the plant has to be considered so it can regulate the uptake.
%}


%{\tred
%Thus, in a further improvement of the model to separate ionic and osmotic stress, it can be assumed that the plant will enter in a stage of toxicity by means of passive uptake since active uptake is plant regulated.
%In the current proposed model, in a situation that the plant is in osmotic stress, it is in CUP with $T_r < 1$ and with passive and active uptake contributions working.
%Once the plant concentration is accounted into the model and the concentration inside the plant already reached the critical level (value the above it, the solute will start to produce toxicity effects), the plant can eliminate the active uptake (because it is plant regulated) but can not eliminate passive uptake.
%Therefore, in this situation, all solute that is extracted by the plant will contribute to a unavoidable accumulation of solute inside the plant, causing the toxicity effects of the ionic stress.
%}
%
%{\tred
%When solute uptake is considered, the value of $H$ diminishes (becomes more negative) slower either because $h_\pi$ becomes less negative or more negative at a smaller rate, which causes that the limiting value of $H$ be achieved on a later time, according to the solute uptake rate.
%Figure \ref[fig_Tr] shows that the onset of the falling $T_r$ phase, for NLU model, occurs in a later time when compared with ZU or CU (which has a smaller solute uptake rate) models.
%}
%
%\medskip
%\label[fig_Tr]
%\picw=11cm \cinspic Trxt_comp.pdf
%\caption/f {Relative transpiration as a function of time for constant (CU), zero (ZU) and non-linear (NLU) uptake models, for scenario 1}
%\medskip

As explained before, the LU model might develop a period of accumulation of solute near the root surface depending on the potential transpiration rate and solute uptake rate values.
This period is then followed by a period of diffusion of solute away from the roots which, along with the solute uptake, are responsible for the depletion of solute at the root surface.
This behaviour causes a different solute concentration profile in the end of simulation when compared with the proposed solute uptake model (NLU).
It can be seen in figure \ref[fig_C] (in the grey line that represents the CU model with a uptake rate 10 times higher) that the diffusion of solute away from the roots generates a higher concentration somewhere between $r_0$ and $r_m$ in CU model, as opposed to NLU model, which has no solute accumulation period and its concentration profile diminishes when approaching $r_0$.
It is also shown that the models that consider solute uptake (CU and NLU) present a smaller concentration at $r_m$ due to the higher solute transport towards the root generated by the gradient caused by the solute uptake at $r_0$.

\medskip
\label[fig_C]
%\picw=11cm \cinspic cxr1.pdf
\picw=11cm \cinspic cxr_comp.pdf
\caption/f {Soil solution concentration as a function of distance from axial root center in the end of simulation for constant (CU), zero (ZU) and non-linear (NLU) uptake models. \descc}
\medskip

%\medskip
%\label[fig_qst]
%\picw=11cm \cinspic qxt1.pdf
%\caption/f {Water flux at root surface as a function of time for constant (CU), zero (ZU) and non-linear (NLU) uptake models, for scenario 1}
%\medskip

%\vfill\break
%\medskip
%%\label[Trs]
%\picw=11cm \cinspic qxt1.pdf
%%\caption/f {Relative transpiration as a function of time and pressure head for no uptake (NU), constant (CU) and non-linear (NLU) uptake models}
%\medskip

%\medskip
%\label[fig_Tr_detail]
%\picw=11cm \cinspic Trxt1_detail.pdf
%\caption/f {Detail of Relative transpiration as a function of time, at times when $C_0<C_{lim}$, for linear (LU) and non-linear (NLU) uptake models}
%\medskip

%\vfill\break
%\medskip
%\label[fig_Im_sensit_curve]
%\picw=11cm \cinspic Im_x_C0.pdf
%\caption/f {Response of the time at the onset of concentration reduction at root surface for different values of $I_m$ parameter}
%\medskip

%In NU, salt is transported to the roots by convection, causing an accumulation of solutes at root surface. As water flux towards the root starts to decrease, salt is transported slower and carried away from the roots by diffusion (Figure \ref[cxt]). Because of the accumulation of salt in the root surface, the total head becomes limiting very fast and the transpiration is reduced faster than the other models (Figure \ref[Trs]).
%
%In CU, as the salt uptake rate is constant (Figure \ref[fluxesxt]), the concentration at root surface will decrease only if the uptake rate is larger than convection to the root surface. In the simulation, it happens in about half of the first day (Figure \ref[cxt]). This is very dependent on the uptake rate and water flux since for different conditions, the outcome could be different. Once the concentration at root surface is zero, the root behaves as a zero-sink, taking up solute at the same rate as which it arrives at the root, keeping the concentration there zero.
%
%In NLU, the concentration at root surface remains constant (Figure \ref[cxt]) until the convection to the root decreases as the water flux decreases (Figure \ref[fluxesxt]). This behavior is really dependent of initial concentration and water flux values since, in this case, $C_0$ at the beginning of simulation is greater than $C_2$, thus the solute uptake equals the convection of solutes to the root. At around day 1, convection starts to decrease but the solute uptake is yet greater than the plant demand (\im) due to convection. The solute uptake becomes constant (and equal to \im) after concentration in root surface is less than \c2. This is clear in XXXFigure 3a, where osmotic head continues constant for a period of time after the beginning of the falling transpiration rate. At this point, active uptake starts since convection only is not capable to maintain solute uptake rate at \im. The concentration keeps decreasing at this constant rate until its value is less than \clim. It is assumed that, at this point, the uptake is not equal to the plant demand for solute (\im) due to the concentration dependence of the MM equation (Figure \ref[fig_MM]). The water flux and the concentration are small as well as the active uptake, that can not maintain the uptake rate at \im. Therefore, a second limiting condition occurs when $C<C_{lim}$ causing another fast decrease in transpiration (Figure \ref[Trs]).
%The calculated concentrations $C_{lim}$ (Eq. \ref[eq_clim]) and \c2 (Eq. \ref[eq_c2]) depend on water flux and ion type (MM parameters $I_m$ and \km) meaning that the results can be quite different for other ion types and different values of initial water content.
%
%Figure \ref[fluxesxt] also shows the changes in solute flux at root surface for all models. At low concentrations (or at the second falling rate stage: $C<C_{lim}$), in NLU, the solute flux decreases gradually over time until the value of concentration is zero, where it will assume the zero-sink behavior.
%
%The concentration profile through the distance from root axial center is shown in Figure \ref[cxr]. The different approaches (NU, CU and NLU) result in different final concentrations profiles. The concentration dependent model NLU takes up more solute from soil solution due to the higher uptake rate in the constant transpiration phase.
%
%Figure \ref[Trs] shows the relative transpiration as a function of time for the three model types. The proposed model is able to maintain the potential transpiration for a longer period of time due to the extraction by passive uptake only ($C>C_2$) that keeps the osmotic head constant, allowing pressure head to reach smaller values at the onset of the limiting hydraulic conditions, as can be seen in Figure 6.
%
%Figure \ref[Trs] shows that CU and NLU have a more negative pressure head value for the onset of limiting hydraulic condition when compared to NU due to solute uptake that causes a increase in osmotic head (becomes less negative) and, in turn, decreases pressure head. Thus, the first falling rate phase of relative transpiration extends in time. The solute uptake at the beginning of the simulation (for concentrations greater than \c2) caused a greater accumulation of solute in the plant and also influenced the final solute profile, in which LU and NLU have less solute left in the soil profile (Figure 13).
%
%At the onset of the second falling rate phase ($C<C_{lim}$), water and solute fluxes decreases rapidly. In Figures XXX10 and 20 we can see that from day 4 to day 5, the fluxes are rapidly reduced, the water flux is near zero in the whole profile, meaning zero or really small convection. Thus, within this period, the transport of solute is made mainly by diffusion and the results of this diffusive transport can be visualized in Figures XXX17 and 18.

%\medskip
%\label[cxt]
%\picw=13cm \cinspic cxt.pdf
%\caption/f {Solute concentration in soil water at root surface as a function of time for no uptake (NU), constant (CU) and non-linear (NLU) uptake models}
%\medskip
%
%\medskip
%\label[cxr]
%\picw=13cm \cinspic cxr.pdf
%\caption/f {Solute concentration in soil water as a function of distance from axial center for no uptake (NU), constant (CU) and non-linear (NLU) uptake models}
%\medskip
%
%\medskip
%\label[fluxesxt]
%\picw=13cm \cinspic fluxesxt.pdf
%\caption/f {Solute and water fluxes at root surface as a function of time for no uptake (NU), constant (CU) and non-linear (NLU) uptake models}
%\medskip
%
%\medskip
%\label[Trs]
%\picw=13cm \cinspic Trs.pdf
%\caption/f {Relative transpiration as a function of time and pressure head for no uptake (NU), constant (CU) and non-linear (NLU) uptake models}
%\medskip

%{\tred 
%The effects of the MM-based uptake model applied in the convection-dispersion equation are noticed in very low concentration values.
%This is due to the selected model assumptions but they are acceptable.
%}
%{\tred
%The addition of a solute uptake function to the already existent model of solute transport was able to improve it, giving more realistic results to the simulations.
%Moreover, it is possible to, in further developments, add a more realistic plant concentration model to improve the proposed model, in a way that the uptake will also depend on the concentration inside the plant, and not only on concentration in soil.
%In that way, the active and passive partitioning would be essential to determine the stress status of the plant, separating osmotic and toxicity effects on biomass production and yield.
%Irrigation and nutrition management would then be improved.
%}

%%%%%%%%%%%%%%%%%%%%%%%%%%%%%%%%%%%%%%%%%%%%%%%%%%%%%%%%%%%%%%%%%%%%%%%%%%%%%%%%%%%%%%%%%%%%%%%%%%%%%%%%%%%%%%%%%%%%%%%%%%%%%%%%%%%%%%%%%%%%%%%%%%%%%%%%%%%%%%%%%%%%%%%%%%%%%
%%%%%%%%%%%%%%%%%%%%%%%%%%%%%%%%%%%%%%%%%%%%%%%%%%%%%%%%%%%%%%%%%%%%%%%%%%%%%%%%%%%%%%%%%%%%%%%%%%%%%%%%%%%%%%%%%%%%%%%%%%%%%%%%%%%%%%%%%%%%%%%%%%%%%%%%%%%%%%%%%%%%%%%%%%%%%
%%%%%%%%%%%%%%%%%%%%%%%%%%%%%%%%%%%%%%%%%%%%%%%%%%%%%%%%%%%%%%%%%%%%%%%%%%%%%%%%%%%%%%%%%%%%%%%%%%%%%%%%%%%%%%%%%%%%%%%%%%%%%%%%%%%%%%%%%%%%%%%%%%%%%%%%%%%%%%%%%%%%%%%%%%%%%

\sec Analytical comparison

The proposed numerical model is compared with the analytical model of \citeonline[cushman], in Figure \ref[fig_analytic], with input parameters of scenario 1.
The analytical model has the limitation to deal with constant water flux and dispersion coefficient (steady-state condition).
Therefore, averages for the simulated water flux and dispersion coefficient of scenario 1 were calculated to serve as inputs in the analytical model.
%The maximum rate of solute uptake of both models were the same ($J_{max}=I_m$), as well as the root absorbing power was equal to the plant affinity to the solute ($k=K_m$).
The input time to evaluate the concentration profile in the analytical model was the same of the numerical model for scenario 1 ($t_{end}=30$ days).

\medskip
\label[fig_analytic]
\picw=13cm \cinspic analytic_cushman.pdf
\caption/f {Solute concentration profile of the analytical and numerical models, with parameters of scenario 1, at time $t=30$ days. The grey line is the numerical result of the linear model }
\medskip

The differences between the two models is mostly due to the consideration of steady-state condition in respect to water flow in the analytical model as opposed to the proposed numerical model that deals with transient water and solute fluxes.
It is shown in the discussion of Section \ref[result_luxnlu] that there is no significant difference between the results of concentration profile of LU and NLU models.
Nevertheless, Figure \ref[fig_analytic] also shows the concentration profile for the LU model, considering that the analytical model has a linear solute uptake rate as the boundary condition at the root surface.
With the assumption of linearity (Equation \ref[eq_cond_linear]), the boundary condition equation has 1 ($k$) instead of 2 ($J_{max}$ and $k$) parameters, which can overestimate the uptake at high concentrations.

Despite the differences in the assumptions and input values of each model, the numerical model shows good agreement with the analytical solution by \citeonline[cushman], which is plausible as they are simulating the same phenomenon over a almost identical scenario.
The proposed numerical model can then be used as an alternative to the analytical model, with the advantage of simulate transient water and solute fluxes, giving the possibility to predict the solute uptake of a wider range of scenarios.

%%%%%%%%%%%%%%%%%%%%%%%%%%%%%%%%%%%%%%%%%%%%%%%%%%%%%%%%%%%%%%%%%%%%%%%%%%%%%%%%%%%%%%%%%%%%%%%%%%%%%%%%%%%%%%%%%%%%%%%%%%%%%%%%%%%%%%%%%%%%%%%%%%%%%%%%%%%%%%%%%%%%%%%%%%%%%
%%%%%%%%%%%%%%%%%%%%%%%%%%%%%%%%%%%%%%%%%%%%%%%%%%%%%%%%%%%%%%%%%%%%%%%%%%%%%%%%%%%%%%%%%%%%%%%%%%%%%%%%%%%%%%%%%%%%%%%%%%%%%%%%%%%%%%%%%%%%%%%%%%%%%%%%%%%%%%%%%%%%%%%%%%%%%
%%%%%%%%%%%%%%%%%%%%%%%%%%%%%%%%%%%%%%%%%%%%%%%%%%%%%%%%%%%%%%%%%%%%%%%%%%%%%%%%%%%%%%%%%%%%%%%%%%%%%%%%%%%%%%%%%%%%%%%%%%%%%%%%%%%%%%%%%%%%%%%%%%%%%%%%%%%%%%%%%%%%%%%%%%%%%

%
%
%\sec Guidelines to upscale the proposed model
%
%{\tred
%The proposed model simulates the transport and uptake of solutes from the soil to a single root and, because of the considered geometry (Figure \ref[root_zone]), it can be easily upscaled to consider a single plant.
%There are more complex models that simulates crop production at field scale and, since the proposed model deals with and quantify active and passive uptakes, it can be used to simulate osmotic and ionic stressors that cause a yield decrease.
%For this, the present model has to be upscaled, or goes from a micro to a macroscopic scale.
%There are 1D, 2D and 3D macroscopic models.
%Here it will be shown how to upscale to a 1D model as for the other dimensions is just do everything again.
%
%One dimensional models usually consider vertical fluxes. 
%The gradients are vertical and all movements and exchanges are vertical, between the layers of the model.
%We will focus in the soil layers.
%The proposed model simulates horizontal fluxes and therefore horizontal gradients.
%Although the model calculates the horizontal gradients and fluxes between all segments and each concentration for them, a representative average soil concentration has to be computed.
%As the calculations takes some time to run, as showed in the previous sections, the number of segments must be the smallest as possible.
%As we saw in FIGURE XXX when the concentration in the root surface is not important, we can use the LU and the generated concentration profile can be used to calculate the average soil concentration.
%As LU is more flexible to deal with time and space steps, it is the chosen model to use in the upscaling.
%More studies must be done to determine the ideal space steps to use for each time step scale, since macroscopic models can use time steps that range from seconds to days.
%}

\cleardoublepage

