%% \appendix file
\appendixpage
% Definition of qued erat demonstrandum (that which was to be demonstrated) abbreviation 
\def\qed{\,\,\,\,\,\,\,\,\hbox{q.e.d.}}

\label[ap_micro]
\input ap_micro

\cleardoublepage
\label[ap_water]
\input ap_water

\cleardoublepage
\label[ap_clim]
\input ap_clim


\cleardoublepage
\label[ap_l_nl_fluxes]
\app Finding whether the solute flux for the linear model can be greater than from the nonlinear model

The subscripts $L$ and $NL$ are relative to linear and nonlinear models, respectively. From Equations \ref[eq_MM_mod] and \ref[eq_MM_linear] we have:
$$
\eqalignno{
q_{sL}&=-(\alpha + q_{0L})C_{0L} \cr
q_{sNL}&=-\left({I_m \over (K_m+C_{0NL})} + q_{0NL}\right)C_{0NL}
}
$$

We will check if there is any situation where $q_{sL}>q_{sNL}$. Then
\label[cond_1]
$$
(\alpha + q_{0L})C_{0L} > \left({I_m \over K_m+C_{0NL}} + q_{0NL}\right)C_{0NL} \eqmark
%\;\;\Rightarrow\;\;
%\alpha > {I_m \over K_m+C_0} \eqmark
$$

For Equation \ref[cond_1] to be true, there must exist at least one situation in which the inequality is true. From Equations \ref[eq_beta] and \ref[eq_linear], we know that:
\label[new_alpha]
$$
\alpha = \beta-q_{0L} = {I_m \over C_{limL}}-q_{0L} \eqmark
$$

Substituting Equation \ref[new_alpha] into \ref[cond_1], we have:
$$
{I_m \over C_{limL}}C_{0L} > \left({I_m \over K_m+C_{0NL}} + q_{0NL}\right)C_{0NL} \eqmark
$$

By knowing that the solute fluxes for linear and nonlinear models are different only for values of $C_0$ lower than \clim, we can set the value of $C_0$ to \clim \ and the equation is still valid. Thus:
$$
{I_m \over C_{limL}}C_{limL} > \left({I_m \over K_m+C_{limNL}} + q_{0NL}\right)C_{limNL}
%{I_m \over C_{lim}} > {I_m \over K_m+C_{lim}}+q_0 \eqmark
$$

With some algebraic operations, we have:
\label[cond_2]
$$
{I_m \over C_{limNL}} > \left({I_m \over K_m+C_{limNL}} + q_{0NL}\right) \eqmark
$$

Since $I_m$ and $K_m$ are constants and greater than zero, and $C_{limNL}$ and $q_{0NL}$ are also greater than zero, it can be easily seen that ${I_m \over C_{limNL}}$ is always greater than ${I_m \over K_m+C_{limNL}}$. 
Therefore, Equation \ref[cond_2] is true for all situations, except for the case:
$$
q_{0NL} > {I_m \over C_{limNL}} - {I_m \over K_m+C_{limNL}}
$$

Simplifying:
\label[cond_3]
$$
q_{0NL} > {I_m K_m \over C_{limNL}(K_m+C_{limNL})} \eqmark
$$

We can solve Equation \ref[cond_3] for $C_{limNL}$ to find the value in which the inequality is true:
\label[cond_clim]
$$
C_{limNL}^2 + K_m C_{limNL} - {I_m K_m \over q_{0NL}} > 0
\;\;\Rightarrow\;\;
C_{limNL} > -{K_m \pm \left( {K_m}^2-4{I_m K_m \over q_{0NL}} \right)^{1 \over 2} \over 2} \eqmark
$$

Equation \ref[cond_clim] is the same as Equation \ref[eq_clim]. Therefore, the condition \ref[cond_1] will never be satisfied since the affirmative  $C_{limNL}>C_{limNL}$ is logicaly wrong.

It has been proved mathematically what was already possible to be noticed in Figure \ref[fig_MM], that nonlinear uptake is always greater than linear and, therefore, that the linearization of the MM equation can be considered reasonable.

\cleardoublepage
\label[ap_flowchart]
\input ap_flowcharts

%\cleardoublepage
%\input code
