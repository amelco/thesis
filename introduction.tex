\cleardoublepage

\def\quic{Quirijn's COMMENT: }

\chap INTRODUCTION

Crop growth is directly related to plant transpiration, and the closer the cumulative transpiration over a growing season is to its potential value, the higher will be the crop yield. 
Any stress occurring during crop development results in stomata closure and transpiration reduction, affecting productivity. 
Therefore, knowing how plants respond to abiotic stresses like those related to water and salt, and predicting and quantifying them, is important not only to improve the understanding of plant-soil interactions, but also to propose better crop management practices.
The interpretation of experimental data to analyze the combined water and salt stress on transpiration and yield has been shown to be difficult due to the great range of possible interactions between the factors determining the behavior of the soil-plant-atmosphere (SPA) system.
%that can be observed in the field.
Modeling has been shown to be an elucidative manner to analyze the involved processes and mechanisms, providing insight in the interaction of water and salt stress.

Analytical models describing transport of nutrients in soil towards plant roots usually consider steady state conditions with respect to water flow to deal with the high non-linearity of soil hydraulic functions. 
Several simplifications (assumptions) are needed regarding the uptake of solutes by the roots, most of them also imposed by the non-linearity of the influx rate function. 
Consequently, although analytical models describe the processes involved in transport and uptake of solutes, they are only capable of simulating water and solute flow just for specific boundary conditions.
%(simplified scenarios that are hardly XXX even in full agreement with real conditions). 
Therefore, applying these models in situations that do not exactly correspond to their boundary condition may lead to a rough approximation but may also result in erroneous predictions.
Many of the available analytical solutions include special math functions (Bessels, Airys or infinite series, for example) that need, at some point, numerical algorithms to compute results.
For the case of the convection-diffusion equation, even the fully analytical solutions are restricted by numerical procedures, although with computationally efficient and reliable results.

As a substitute to analytical solutions, numerical modeling allows more flexibility when dealing with non-linear equations, being an alternative to better cope with diverse boundary conditions. 
The functions can be solved considering transient conditions for water and solute flow but with some pullbacks regarding numerical stability and more processing to perform calculations.
In general, numerical models use empirical functions that relate osmotic stress to some electric conductivity of the soil solution. 
The parameters of these empirical models depend on soil, plant and atmospheric conditions in a range covered by the experiments used to generate data for model calibration. 
Using these models out of the measured range is not recommended and, in these cases, a new parameter calibration should be done.
Physical/mechanistic models for the solute transport equations describe the involved processes in a wider range of situations since it is less dependent on experimental data, giving more reliable results.

%describe liersolute

The objective of this thesis is to present a modification of the model of root water uptake and solute transport proposed by \citeonline[liersolute].
This modification allows the model to take into account plant solute uptake.
%develop a
To do so, a numerical mechanistic solution for the equation of convection-dispersion will be developed that considers transient flow of water and solute, as well as root competition.
A soil concentration dependent solute uptake function as boundary condition at the root surface was assumed.
In this way, the new model allows prediction of active and passive contributions to the solute uptake, which can be used to separate ionic and osmotic stresses by considering solute concentration inside the plant. 
The proposed model is compared with the original model, with a constant solute uptake numerical model and with an analytical model that uses a steady state condition for water content. 
%It is also shown how the model can be upscaled to be used in macroscopic scale hydrological models.
%{\localcolor \Red \quic It is not clear the advantages of separate active and passive.}
