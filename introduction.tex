\cleardoublepage

\def\quic{Quirijn's COMMENT: }

\chap INTRODUCTION

Crop growth is directly related to plant transpiration, so that the closer the transpiration is from its potential value, the higher will be the crop yield. 
Any stress that occurs during crop development causes transpiration reduction, affecting productivity. 
Therefore, knowing how plants respond to abiotic stress such as of water and salt, and predicting and quantifying them, is important not only to improve the understanding of plant-soil mechanisms, but also to propose better crop management practices.
The use of experimental data to analyze the combined water and salt stress on relative transpiration and relative yield has been shown to be difficult to examine due to the great range of possible interactions between the actors in the soil-plant-atmosphere (SPA) system.
%that can be observed in the field.
Modeling has been shown to be an elucidative manner to analyze the involved processes and mechanisms, giving a new insight in how water and salt stress interact.

Analytical models of transport of nutrients in soil towards plant roots usually consider steady-state conditions with respect to water flow to deal with the high nonlinearity of soil hydraulic functions. 
Several simplifications (assumptions) are made regarding the uptake of solutes by the roots, most of them also imposed by the nonlinearity of the influx rate function. 
Consequently, although analytical models describe the processes involved in transport and uptake of solutes, they are only capable of simulating water and solute flow just for specific boundary conditions.
%(simplified scenarios that are hardly XXX even in full agreement with real conditions). 
Therefore, their application for situations they do not fully represent can lead to only a rough approximation.
Even the available analytical solutions include special math functions (Bessels, Airys or infinite series, for example) that need, at some point, numerical algorithms to compute results.
Thus, for the case of convection-diffusion equation, even the fully analytical solutions are restricted by numerical procedures although they have yet fast and reliable results.

Numerical modeling, in turn, has more flexibility when dealing with nonlinear equations, being an alternative to better cope with boundary conditions. 
The functions can be solved considering transient conditions for water and solute flow but with some pull-backs regarding numerical stability and more time demanded to perform calculations.
In general, numerical models use empirical functions in the determination of osmotic stress, relating it to the electric conductivity of the soil solution. 
The parameters of these empirical models depend on soil, plant and atmospheric conditions in a range covered by the experiments used to generate data for model calibration. 
One must be aware that the use of such models for extrapolated scenarios can result in prediction errors and that, commonly, a new calibration of the parameters needs to be done.
A model that uses a physical/mechanistic approach for the solute transport equations can describe the involved processes in a wider range of situations since it is less dependent on experimental data, giving more reliable results.

%describe liersolute

In this thesis, the objective is to 
implement a modification in the model of root water uptake and solute transport proposed by \citeonline[liersolute] in order to take into account the solute uptake.
%develop a
A numerical mechanistic solution for the equation of convection--dispersion, considering transient flow for water and solute, as well as root competition, and assuming a soil concentration dependent solute uptake function as boundary condition at the root surface was developed.
It has the advantage of quantifying active and passive contributions to the solute uptake, which can be used to separate ionic and osmotic stresses when solute concentration inside the plant is considered. 
The proposed model is compared with 
the original model,
%a zero solute uptake and 
with a constant solute uptake numerical model and with an analytical model that uses steady-state condition for water content. 
It is also shown how the model can be upscaled to be used in macroscopic scale hydrological models.
%{\localcolor \Red \quic It is not clear the advantages of separate active and passive.}
